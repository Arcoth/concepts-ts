

\setcounter{chapter}{4}
\rSec0[expr]{Expressions}

Modify paragraph 8 to include a reference to \grammarterm{requires-expression}{}s.

\begin{quote}
\pnum
In some contexts, unevaluated operands appear
(\added{\ref{expr.prim.req}, }\cxxref{expr.typeid}, \cxxref{expr.sizeof}, 
\cxxref{expr.unary.noexcept}).
\end{quote}

%%
%% Primary Expressions
%%
\rSec1[expr.prim]{Primary expressions}

%%
%% General (primary expressions)
%%
\rSec2[expr.prim.general]{General}

In this section, add the
\grammarterm{requires-expression} to the rule for 
\grammarterm{primary-expression}.

\begin{quote}
\begin{bnf}
\nontermdef{primary-expression}\br
    literal\br
    \terminal{this}\br
    \terminal{(} expression \terminal{)}\br
    id-expression\br
    lambda-expression\br
    \added{fold-expression}\br
    \added{requires-expression}
\end{bnf}
\end{quote}

In paragraph 8, add \tcode{auto} and \grammarterm{constrained-type-name} 
to \grammarterm{nested-name-specifier}:

\begin{quote}
\setcounter{Paras}{7}
\pnum\vspace{-\the\baselineskip} % Adjust to hide the blank line.

\begin{bnf}
\nontermdef{nested-name-specifier}\br
    \terminal{::}\br
    type-name \terminal{::}\br
    namespace-name \terminal{::}\br
    decltype-specifier \terminal{::}\br
    \added{\terminal{auto} \terminal{::}}\br
    \added{constrained-type-name \terminal{::}}\br
    nested-name-specifier identifier \terminal{::}\br
    nested-name-specifier \terminal{template}\opt \terminal{::}
\end{bnf}
\end{quote}

Add a new paragraph after paragraph 11:

\begin{quote}
\begin{addedblock}
\setcounter{Paras}{11}
\pnum
In a \grammarterm{nested-name-specifier} of the form \tcode{auto::} or
\tcode{C::}, where \tcode{C} is a \grammarterm{constrained-type-name},
that \grammarterm{nested-name-specifier} designates a placeholder that 
will be replaced later according to the rules for placeholder deduction in
\ref{dcl.spec.auto}.
%
If the designated placeholder is not a placeholder type, the program is
ill-formed.
%
\enternote
A \grammarterm{constrained-type-specifier} can designate a placeholder for
a non-type or template (\ref{dcl.spec.auto.constr}).
\exitnote
%
The replacement type deduced for a placeholder shall be a class or
enumeration type.
% 
\enterexample
\begin{codeblock}
template<typename T> concept bool C = sizeof(T) == sizeof(int);
template<int N> concept bool D = true;

struct S1 { int n; };
struct S2 { char c; };
struct S3 { struct X { using Y = int; }; };

int auto::* p1 = &S1::n; // \tcode{auto} deduced as \tcode{S1}
int D::* p2 = &S1::n;    // error: \tcode{D} does not designate a placeholder type
int C::* p3 = &S1::n;    // OK: \tcode{C} deduced as \tcode{S1}
char C::* p4 = &S2::c;   // error: deduction fails because constraints are not satisfied

void f(typename auto::X::Y);
f(S1()); // error: \tcode{auto} cannot be deduced from \tcode{S1()}
f<S3>(0); // OK
\end{codeblock}
In the declaration of \tcode{f}, the placeholder appears in a non-deduced 
context (\cxxref{temp.deduct.type}). It may be replaced later through the
explicit specification of template arguments.
\exitexample
\end{addedblock}
\end{quote}

Add a new paragraph after paragraph 13:

% TODO: The wording "non-overloaded function declaration" is not right.
\begin{quote}
\begin{addedblock}
\setcounter{Paras}{13}
\pnum
A program that refers explicitly or implicitly to a function with associated 
constraints that are not satisfied (\ref{temp.constr}), other than to declare 
it, is ill-formed.
% 
\enterexample
\begin{codeblock}
void f(int) requires false;

f(0);                      // error: cannot call \tcode{f}
void (*p1)(int) = f;       // error: cannot take the address of \tcode{f}
decltype(f)* p2 = nullptr; // error: the type \tcode{decltype(f)} is invalid
\end{codeblock}
In each case the associated constraints of \tcode{f} are not satisfied. In the 
declaration of \tcode{p2}, those constraints are required to be satisfied even 
though \tcode{f} is an unevaluated operand (Clause~\ref{expr}).
\exitexample
\end{addedblock}
\end{quote}


%%
%% Lambda expressions
%%
\rSec2[expr.prim.lambda]{Lambda expressions}

Insert the following paragraph after paragraph 4 to define the
term ``generic lambda''. 

\begin{quote}
\begin{addedblock}
\setcounter{Paras}{4}
\pnum
A \defn{generic lambda} is a \grammarterm{lambda-expression} where one
or more placeholders (\ref{dcl.spec.auto}) appear in the parameter-type-list
of the \grammarterm{lambda-declarator}.
\end{addedblock}
\end{quote}

Modify paragraph 5 so that the meaning of a generic lambda is defined 
in terms of its abbreviated member function template call operator.

\begin{quote}
The closure type for a non-generic
\grammarterm{lambda-expression} has a public inline 
function call operator (\cxxref{over.call})
whose parameters and return type are described by the 
\grammarterm{lambda-expression}'s 
\grammarterm{parameter-declaration-clause} and 
\grammarterm{trailing-return-type}, respectively.
%
\removed{
For a generic lambda, the closure type has a public inline function call
operator member template (\cxxref{temp.mem}) whose 
\grammarterm{template-parameter-list} consists of 
one invented type \grammarterm{template-parameter} 
for each occurrence of \tcode{auto} in the lambda's 
\grammarterm{parameter-declaration-clause}, in order 
of appearance.
%
The invented type \grammarterm{template-parameter} is 
a parameter pack if the corresponding 
\grammarterm{parameter-declaration declares} a 
function parameter pack (\ref{dcl.fct}). 
%
The return type and function parameters of the function call operator 
template are derived from the 
\grammarterm{lambda-expression}'s 
\grammarterm{trailing-return-type} and 
\grammarterm{parameter-declaration-clause} by 
replacing each occurrence of \tcode{auto} in the 
\grammarterm{decl-specifier}{}s of the 
\grammarterm{parameter-declaration-clause}
with the name of the corresponding invented template-parameter.
}
%
\added{
The closure type for a generic lambda has a public inline function call 
operator member template that is an abbreviated function template whose 
parameters and return type are derived from the
\grammarterm{lambda-expression}'s 
\grammarterm{parameter-declaration-clause} and 
\grammarterm{trailing-return-type} according to the
rules in (\ref{dcl.fct}).
}
\end{quote}

Add the following example after those in paragraph 5 in the 
\Cpp Standard.

\begin{quote}
\begin{addedblock}
\enterexample
\begin{codeblock}
template<typename T> concept bool C = true;

auto gl = [](C& a, C* b) { a = *b; }; // OK: denotes a generic lambda

struct Fun {
  auto operator()(C& a, C* b) const { a = *b; }
} fun;
\end{codeblock}
\tcode{C} is a 
\grammarterm{constrained-type-specifier},
signifying that the lambda is generic. The generic lambda \tcode{gl}
and the function object \tcode{fun} have equivalent behavior when 
called with the same arguments.
\exitexample
\end{addedblock}
\end{quote}

%%
%% Fold expressions
%%
\rSec2[expr.prim.fold]{Fold expressions}

Add this section after 5.1.2.

\begin{quote}

\pnum
A fold expression performs a fold of a template parameter
pack~(\ref{temp.variadic}) over a binary operator.

\begin{bnf}
\nontermdef{fold-expression}\br
    \terminal{(} cast-expression fold-operator \terminal{...} \terminal{)}\br
    \terminal{(} \terminal{...} fold-operator cast-expression \terminal{)}\br
    \terminal{(} cast-expression fold-operator \terminal{...} fold-operator cast-expression \terminal{)}
\end{bnf}

\begin{bnf}
\nontermdef{fold-operator} \textnormal{one of}\br
    \terminal{+ }\quad\terminal{- }\quad\terminal{* }\quad\terminal{/ }\quad\terminal{\% }\quad\terminal{\^{} }\quad\terminal{\& }\quad\terminal{| }\quad\terminal{\shl\ }\quad\terminal{\shr }\br
    \terminal{+=}\quad\terminal{-=}\quad\terminal{*=}\quad\terminal{/=}\quad\terminal{\%=}\quad\terminal{\^{}=}\quad\terminal{\&=}\quad\terminal{|=}\quad\terminal{\shl=}\quad\terminal{\shr=}\quad\terminal{=}\br
    \terminal{==}\quad\terminal{!=}\quad\terminal{< }\quad\terminal{> }\quad\terminal{<=}\quad\terminal{>=}\quad\terminal{\&\&}\quad\terminal{||}\quad\terminal{,  }\quad\terminal{.* }\quad\terminal{->*}
\end{bnf}

\pnum
An expression of the form
\tcode{(...} \placeholder{op} \tcode{e)}
where \placeholder{op} is a \grammarterm{fold-operator}
is called a \defn{unary left fold}.
An expression of the form
\tcode{(e} \placeholder{op} \tcode{...)}
where \placeholder{op} is a \grammarterm{fold-operator}
is called a \defn{unary right fold}.
Unary left folds and unary right folds
are collectively called \defnx{unary folds}{unary fold}.
In a unary fold,
the \grammarterm{cast-expression}
shall contain an unexpanded parameter pack~(\ref{temp.variadic}).

\pnum
An expression of the form
\tcode{(e1} \placeholder{op1} \tcode{...} \placeholder{op2} \tcode{e2)}
where \placeholder{op1} and \placeholder{op2} are \grammarterm{fold-operator}{s}
is called a \defn{binary fold}.
In a binary fold,
\placeholder{op1} and \placeholder{op2}
shall be the same \grammarterm{fold-operator},
and either \tcode{e1}
shall contain an unexpanded parameter pack
or \tcode{e2}
shall contain an unexpanded parameter pack,
but not both.
If \tcode{e2} contains an unexpanded parameter pack,
the expression is called a \defn{binary left fold}.
If \tcode{e1} contains an unexpanded parameter pack,
the expression is called a \defn{binary right fold}.
\enterexample
\begin{codeblock}
template<typename ...Args>
bool f(Args ...args) {
  return (true && ... && args); // OK
}

template<typename ...Args>
bool f(Args ...args) {
  return (args + ... + args); // error: both operands contain unexpanded parameter packs
}
\end{codeblock}
\exitexample

\end{quote}


%%
%% Requires expressions
%%
\rSec2[expr.prim.req]{Requires expressions}

Add this section to 5.1.

\begin{quote}

\pnum
A \grammarterm{requires-expression} provides a concise way to express 
requirements on template arguments. 
% 
A requirement is one that can be checked by name lookup
(\cxxref{basic.lookup}) or by checking properties of types and expressions.

\begin{bnf}
\nontermdef{requires-expression}\br
    \terminal{requires} requirement-parameter-list\opt requirement-body

\nontermdef{requirement-parameter-list}\br
    \terminal{(} parameter-declaration-clause\opt~\terminal{)}
  
\nontermdef{requirement-body}\br
    \terminal{\{} requirement-seq \terminal{\}}

\nontermdef{requirement-seq}\br
    requirement\br
    requirement-seq requirement

\nontermdef{requirement}\br
    simple-requirement\br
    type-requirement\br
    compound-requirement\br
    nested-requirement
\end{bnf}

\pnum
A \grammarterm{requires-expression} defines a constraint (\ref{temp.constr})
based on its parameters (if any) and its nested \grammarterm{requirement}{s}.

\pnum
A \grammarterm{requires-expression} has type \tcode{bool} and is an 
unevaluated expression (\ref{expr}).
\enternote
A \grammarterm{requires-expression} is transformed into a constraint 
(\ref{temp.constr.decl}) in order to determine if it is 
satisfied (\ref{temp.constr}).
\exitnote

\pnum
A \grammarterm{requires-expression} shall appear
only within a concept definition (\ref{dcl.spec.concept}),
or within the \grammarterm{requires-clause} of a
\grammarterm{template-declaration} 
(Clause~\ref{temp}) or function declaration (\ref{dcl.fct}).
% 
\enterexample
A common use of \grammarterm{requires-expression}{}s is to define
requirements in concepts such as the one below:
\begin{codeblock}
template<typename T>
  concept bool R() {
    return requires (T i) {
      typename T::type;
      {*i} -> const T::type&;
    };
  }
\end{codeblock}
A \grammarterm{requires-expression} can also be used in a 
\grammarterm{requires-clause} as a way of writing ad hoc 
constraints on template arguments such as the one below:
\begin{codeblock}
template<typename T>
  requires requires (T x) { x + x; }
    T add(T a, T b) { return a + b; }
\end{codeblock}
The first \tcode{requires} introduces the 
\grammarterm{requires-clause}, and the second
introduces the \grammarterm{requires-expression}.
\exitexample
\enternote
Such requirements can also be written by defining them within
a concept.
\begin{codeblock}
template<typename T>
  concept bool C = requires (T x) { x + x; };

template<typename T> requires C<T> 
  T add(T a, T b) { return a + b; }
\end{codeblock}
\exitnote

\pnum
A \grammarterm{requires-expression} may introduce local parameters using a
\grammarterm{parameter-declaration-clause}
(\ref{dcl.fct}). 
%
A local parameter of a \grammarterm{requires-expression} shall not have a 
default argument.
%
Each name introduced by a local parameter is in scope from the point
of its declaration until the closing brace of the
\grammarterm{requirement-body}.
%
These parameters have no linkage, storage, or lifetime; they are only used
as notation for the purpose of defining \grammarterm{requirement}{}s.
%
The \grammarterm{parameter-declaration-clause} of a 
\grammarterm{requirement-parameter-list} shall
not terminate with an ellipsis.
\enterexample
\begin{codeblock}
template<typename T>
  concept bool C1() { 
    requires(T t, ...) { t; }; // error: terminates with an ellipsis
  }

template<typename T>
  concept bool C2() { 
    requires(T t, void (*p)(T*, ...)) // OK: the parameter-declaration-clause of
    { p(t); };                        // the requires-expression does not terminate 
  }                                   // with an ellipsis
\end{codeblock}
\exitexample

\pnum
The \grammarterm{requirement-body} is comprised of 
a sequence of \grammarterm{requirement}{}s.
%
These \grammarterm{requirement}{}s may refer to local 
parameters, template parameters, and any other declarations visible from the 
enclosing context. 
%
Each \grammarterm{requirement} appends a constraint (\ref{temp.constr}) to 
the conjunction of constraints defined by the
\grammarterm{requires-expression}. Constraints are appended in the order
in which they are written.

\pnum
The substitution of template arguments into a \grammarterm{requires-expression} 
may result in the formation of invalid types or expressions in its
requirements. In such cases, the constraints corresponding to those
requirements are not satisfied; it does not cause the program to be ill-formed.
%
If the substitution of template arguments into a \grammarterm{requirement} 
would always result in a substitution failure, the program is ill-formed; 
no diagnostic required.
%
\enterexample
\begin{codeblock}
template<typename T> concept bool C =
  requires {
    new int[-(int)sizeof(T)]; // ill-formed, no diagnostic required
  };
\end{codeblock}
\exitexample


%%
%% Simple requirements
%%
\rSec3[expr.prim.req.simple]{Simple requirements}

\begin{bnf}
\nontermdef{simple-requirement}\br
    expression \terminal{;}
\end{bnf}

\pnum
A \grammarterm{simple-requirement} introduces an expression 
constraint (\ref{temp.constr.expr}) for its 
\grammarterm{expression}.
\enternote
An expression constraint asserts the validity of an expression.
\exitnote

\enterexample
\begin{codeblock}
template<typename T> concept bool C =
  requires (T a, T b) {
    a + b;  // an expression constraint for \tcode{a + b}
  };
\end{codeblock}
\exitexample


%%
%% Type requirements
%%
\rSec3[expr.prim.req.type]{Type requirements}

\begin{bnf}
\nontermdef{type-requirement}\br
    \terminal{typename} nested-name-specifier\opt type-name \terminal{;}
\end{bnf}

\pnum
A \grammarterm{type-requirement} introduces a type 
constraint (\ref{temp.constr.type}) for the type named by its
optional \grammarterm{nested-name-specifier} and \grammarterm{type-name}.
%
\enternote
A type requirement asserts the validity of an associated
type, either as a member type, a class template specialization,
or an alias template. It is not used to specify requirements for
arbitrary \grammarterm{type-specifier}{}s.
\exitnote
%
\enterexample
\begin{codeblock}
template<typename T> struct S { };
template<typename T> using Ref = T&;

template<typename T> concept bool C =
  requires () {
    typename T::inner; // required nested member name
    typename S<T>;     // required class template specialization
    typename Ref<T>;   // required alias template substitution
  };
\end{codeblock}
\exitexample


%%
%% Compound requirements
%%
\rSec3[expr.prim.req.compound]{Compound requirements}
      
\begin{bnf}
\nontermdef{compound-requirement}\br
  \terminal{\{} expression \terminal{\}} \terminal{noexcept}\opt 
    trailing-return-type\opt~\terminal{;}
\end{bnf}

\pnum
A \grammarterm{compound-requirement} introduces 
a conjunction of one or more constraints for the
\grammarterm{expression} \tcode{E}. The order in which
those constraints are introduced is:
%
\begin{itemize}
\item the \grammarterm{compound-requirement} introduces an expression 
constraint for \tcode{E} (\ref{temp.constr.expr});

\item if the \tcode{noexcept} specifier is present, the 
\grammarterm{compound-requirement} appends an exception constraint for 
\tcode{E} (\ref{temp.constr.noexcept});

\item if the \grammarterm{trailing-return-type} is present, the
\grammarterm{compound-requirement} appends one or more constraints derived 
from the type \tcode{T} named by the \grammarterm{trailing-return-type}:

\begin{itemize}
\item if \tcode{T} contains one or more placeholders (\ref{dcl.spec.auto}), 
the requirement appends a deduction constraint (\ref{temp.constr.deduct}) 
of \tcode{E} against the type \tcode{T}.

\item otherwise, the requirement appends two constraints:
a type constraint on the formation of \tcode{T} (\ref{temp.constr.type}) and
an implicit conversion constraint from \tcode{E} to \tcode{T}
(\ref{temp.constr.conv}).
\end{itemize}
\end{itemize}
%
\enterexample
\begin{codeblock}
template<typename T> concept bool C1 =
  requires(T x) {
    {x++};
  };
\end{codeblock}
The \grammarterm{compound-requirement} in \tcode{C1} 
introduces an expression constraint for \tcode{x++}.
It is equivalent to a \grammarterm{simple-requirement}
with the same \grammarterm{expression}.

\begin{codeblock}
template<typename T> concept bool C2 =
  requires(T x) {
    {*x} -> typename T::inner;
  };
\end{codeblock}

The \grammarterm{compound-requirement} in \tcode{C2} 
introduces three constraints: an expression constraint for \tcode{*x}, 
a type constraint for \tcode{typename T::inner}, and a conversion 
constraint requiring \tcode{*x} to be implicitly convertible to
\tcode{typename T::inner}.

\begin{codeblock}
template<typename T> concept bool C3 =
  requires(T x) {
    {g(x)} noexcept;
  };
\end{codeblock}

The \grammarterm{compound-requirement} in \tcode{C3} 
introduces two constraints: an expression constraint for \tcode{g(x)} 
and an exception constraint for \tcode{g(x)}.

\begin{codeblock}
template<typename T> concept bool C() { return true; }

template<typename T> concept bool C5 =
  requires(T x) {
    {f(x)} -> const C&;
  };
\end{codeblock}

The \grammarterm{compound-requirement} in \tcode{C5} 
introduces two constraints: an expression constraint for \tcode{f(x)}, 
and a deduction constraint requiring that overload resolution succeeds for the
call \tcode{g(f(x))} where \tcode{g} is the following
invented abbreviated function template.
\begin{codeblock}
void g(const C&);
\end{codeblock}
\exitexample

%%
%% Nested requirements
%%
\rSec3[expr.prim.req.nested]{Nested requirements}

\begin{bnf}
\nontermdef{nested-requirement}\br
    requires-clause \terminal{;}
  \end{bnf}

\pnum
A \grammarterm{nested-requirement} can be used
to specify additional constraints in terms of local parameters.
%
A \grammarterm{nested-requirement} appends a predicate constraint 
(\ref{temp.constr.pred}) for its \grammarterm{constraint-expression} 
to the conjunction of constraints introduced by its enclosing
\grammarterm{requires-expression}.
% 
\enterexample
\begin{codeblock}
template<typename T> concept bool C() { return sizeof(T) == 1; }

template<typename T> concept bool D =
  requires (T t) {
    requires C<decltype (+t)>();
  };
\end{codeblock}
The \grammarterm{nested-requirement} appends the predicate constraint 
\tcode{sizeof(decltype (+t)) == 1} (\ref{temp.constr.pred}).
\exitexample

\end{quote}
