

\setcounter{chapter}{4}
\rSec0[expr]{Expressions}

Modify paragraph 8 to include a reference to \grammarterm{requires-expression}{}s.

\begin{quote}
\pnum
In some contexts, unevaluated operands appear
(\added{\ref{expr.prim.req}, }\cxxref{expr.typeid}, \cxxref{expr.sizeof}, 
\cxxref{expr.unary.noexcept}).
\end{quote}

%%
%% Primary Expressions
%%
\rSec1[expr.prim]{Primary expressions}

%%
%% General (primary expressions)
%%
\rSec2[expr.prim.general]{General}

In this section, add the
\grammarterm{requires-expression} to the rule for 
\grammarterm{primary-expression}.

\begin{quote}
\begin{bnf}
\nontermdef{primary-expression}\br
    \added{requires-expression}
\end{bnf}
\end{quote}

In paragraph 8, add \tcode{auto} to \grammarterm{nested-name-specifier}:

\begin{quote}
\setcounter{Paras}{7}
\pnum
\begin{bnf}
\nontermdef{nested-name-specifier}\br
    \terminal{auto} \terminal{::}
\end{bnf}
\end{quote}

Add a new paragraph after paragraph 11:

\begin{quote}
\setcounter{Paras}{11}
\pnum
In a \grammarterm{nested-name-specifier} of the form \tcode{auto::}, 
the \tcode{auto} specifier is a placeholder for a type to be deduced
(\ref{dcl.spec.auto}).
\end{quote}

Add a new paragraph after paragraph 13:

% TODO: The wording "non-overloaded function declaration" is not right.
\begin{quote}
\setcounter{Paras}{13}
\pnum
If an \grammarterm{id-expression} denotes a non-overloaded function 
declaration that was declared with a \grammarterm{requires-clause}
(\ref{dcl.fct}), its associated constraints shall be satisfied
(\ref{temp.constr}).
\enterexample
\begin{codeblock}
void f(int) requires false;

f(0);               // error: constraints not satisfied
void (*p)(int) = f; // error: constraints not satisfied
\end{codeblock}
\exitexample
\end{quote}

%   <!-- ==================
%        Lambda expressions
%        ================== -->

%%
%% Lambda expressions
%%
\rSec2[expr.prim.lambda]{Lambda expressions}

Insert the following paragraph after paragraph 4 to define the
term ``generic lambda''. 

\begin{quote}
\setcounter{Paras}{4}
\pnum
A \defn{generic lambda} is a \grammarterm{lambda-expression} where either the
\tcode{auto} \grammarterm{type-specifier} (\ref{dcl.spec.auto})
or a \grammarterm{constrained-type-specifier} (\ref{dcl.spec.constr})
appears in a parameter type of the \grammarterm{lambda-declarator}.
\end{quote}

Modify paragraph 5 so that the meaning of a generic lambda is defined 
in terms of its abbreviated member function template call operator.

\begin{quote}
The closure type for a non-generic
\grammarterm{lambda-expression} has a public inline 
function call operator (\cxxref{over.call})
whose parameters and return type are described by the 
\grammarterm{lambda-expression}'s 
\grammarterm{parameter-declaration-clause} and 
\grammarterm{trailing-return-type}, respectively.
%
\removed{
For a generic lambda, the closure type has a public inline function call
operator member template (\cxxref{temp.mem}) whose 
\grammarterm{template-parameter-list} consists of 
one invented type \grammarterm{template-parameter} 
for each occurrence of \tcode{auto} in the lambda's 
\grammarterm{parameter-declaration-clause}, in order 
of appearance.
%
The invented type \grammarterm{template-parameter} is 
a parameter pack if the corresponding 
\grammarterm{parameter-declaration declares} a 
function parameter pack (\ref{dcl.fct}). 
%
The return type and function parameters of the function call operator 
template are derived from the 
\grammarterm{lambda-expression}'s 
\grammarterm{trailing-return-type} and 
\grammarterm{parameter-declaration-clause} by 
replacing each occurrence of \tcode{auto} in the 
\grammarterm{decl-specifier}{}s of the 
\grammarterm{parameter-declaration-clause}
with the name of the corresponding invented template-parameter.
}
%
\added{
For a generic lambda, the function call operator (\cxxref{over.call}) 
is an abbreviated function template (\ref{dcl.fct}).
}
\end{quote}

Add the following example after those in paragraph 5 in the 
\Cpp Standard.

\begin{quote}
\begin{addedblock}
\enterexample
\begin{codeblock}
template<typename T> concept bool C = true;

auto gl = [](C& a, C* b) { a = *b }; // OK: denotes a generic lambda

struct Fun {
  auto operator()(C& a, C* b) const { return a = *b; }
} fun;
\end{codeblock}
\tcode{C} is a 
\grammarterm{constrained-type-specifier},
signifying that the lambda is generic. The generic lambda \tcode{gl}
and the function object \tcode{fun} have equivalent behavior when 
called with the same arguments.
\exitexample
\end{addedblock}
\end{quote}

%%
%% Requires expressions
%%
\rSec2[expr.prim.req]{Requires expressions}

Add this section to 5.1.

\begin{quote}

\pnum
A \grammarterm{requires-expression} provides a concise way to express 
syntactic requirements on template arguments.
A syntactic requirement is one that can be checked by name lookup 
(\cxxref{basic.lookup}) or by checking properties of types and expressions.

\begin{bnf}
\nontermdef{requires-expression}\br
    \terminal{requires} requirement-parameter-list\opt requirement-body

\nontermdef{requirement-parameter-list}\br
    \terminal{(} parameter-declaration-clause\opt \terminal{)}
  
\nontermdef{requirement-body}\br
    \terminal{\{} requirement-list \terminal{\}}

\nontermdef{requirement-list}\br
    requirement\br
    requirement-list requirement

\nontermdef{requirement}\br
    simple-requirement\br
    type-requirement\br
    compound-requirement\br
    nested-requirement
\end{bnf}

\pnum
A \grammarterm{requires-expression} defines a conjunction of the constraints 
(\ref{temp.constr}) introduced by its \grammarterm{requirement}{}s.

\pnum
A \grammarterm{requires-expression} has type \tcode{bool} and is an 
unevaluated operand (\ref{expr}).

\pnum
A \grammarterm{requires-expression} shall appear
only within a concept definition (\ref{dcl.spec.constr}),
or a \grammarterm{requires-clause} following a
\grammarterm{template-parameter-list} 
(\ref{temp}) or function declaration (\ref{dcl.fct}).

% FIXME: The example names an undeclared entity.
\enterexample
A common use of \grammarterm{requires-expression}{}s is to define
syntactic requirements in concepts such as the one below:
\begin{codeblock}
template<typename T>
  concept bool R() {
    return requires (T i) {
      typename A<T>;
      {*i} -> const A<T>&;
    };
  }
\end{codeblock}
A \grammarterm{requires-expression} can also be used in a 
\grammarterm{requires-clause} templates as a way of writing ad hoc 
constraints on template arguments such as the one below:
\begin{codeblock}
template<typename T>
  requires requires (T x) { x + x; }
    T add(T a, T b) { return a + b; }
\end{codeblock}
The first \tcode{requires} introduces the 
\grammarterm{requires-clause}, and the second
introduces the \grammarterm{requires-expression}.
\exitexample
\enternote
Such requirements can also be written using by defining them within
a concept.
\begin{codeblock}
template<typename T>
  concept bool C = requires (T x) { x + x; };

template<typename T> requires C<T> 
  T add(T a, T b) { return a + b; }
\end{codeblock}
\exitnote

\pnum
A \grammarterm{requires-expression} may introduce local parameters using a
\grammarterm{parameter-declaration-clause}
(\ref{dcl.fct}). 
%
A local parameter of a \grammarterm{requires-expression} shall not have a 
default argument.
%
Each name introduced by a local parameter is in scope from the point
of its declaration until the closing brace of the
\grammarterm{requirement-body}.
%
These parameters have no linkage, storage, or lifetime; they are only used
as notation for the purpose of defining \grammarterm{requirement}{}s.
%
The \grammarterm{requirement-parameter-list} shall
not include an ellipsis.

\pnum
The \grammarterm{requirement-body} is comprised of 
a sequence of \grammarterm{requirement}{}s.
%
These \grammarterm{requirement}{}s may refer to local 
parameters, template parameters, and any other declarations visible from the 
enclosing context. 
%
Each \grammarterm{requirement} appends one or more atomic constraints
(\ref{temp.constr}) to the conjunction of constraints defined by the 
\grammarterm{requires-expression}{}s.

\pnum
The substitution of template arguments into a \grammarterm{requirement} may 
result in an invalid type or expression, in which case the constraint is not
satisfied; it does not cause the program to be ill-formed.
%
\enternote
But if the substitution of template arguments into
a \grammarterm{requirement} would always result in
a substitution failure, the program is ill-formed; no diagnostic
required (\ref{temp.res}).
\exitnote
%
\enterexample
\begin{codeblock}
template<typename T> concept bool C =
  requires () {
    new int[(int)-sizeof(T)]; // ill-formed, no diagnostic required
  };
\end{codeblock}
\exitexample


%%
%% Simple requirements
%%
\rSec3[expr.prim.req.simple]{Simple requirements}

\begin{bnf}
\nontermdef{simple-requirement}
    expression \terminal{;}
\end{bnf}

\pnum
A \grammarterm{simple-requirement} introduces an expression 
constraint (\ref{temp.constr.atom.expr}) for its 
\grammarterm{expression}.

\enterexample
\begin{codeblock}
template<typename T> concept bool C =
  requires (T a, T b) {
    a + b;  // a simple requirement
  };
\end{codeblock}
\exitexample


%%
%% Type requirements
%%
\rSec3[expr.prim.req.type]{Type requirements}

\begin{bnf}
\nontermdef{type-requirement}\br
    typename-specifier \terminal{;}
\end{bnf}

\pnum
A \grammarterm{type-requirement} introduces type 
constraint (\ref{temp.constr.atom.type}) for the type named by its
\grammarterm{typename-specifier}.
%
\enternote
A type requirement asserts the validity of an associated
type, either as a member type, a class template specialization,
or an alias template. It is not used to specify requirements for
arbitrary \grammarterm{type-specifiers}.
\exitnote
%
\enterexample
\begin{codeblock}
template<typename T> struct S { };
template<typename T> using Ref = T&;

template<typename T> concept bool C =
  requires () {
    typename T::inner; // required nested type name
    typename S<T>;     // required class template specialization
    typename Ref<T>;   // required alias template substitution
  };
\end{codeblock}
\exitexample


%%
%% Compound requirements
%%
\rSec3[expr.prim.req.compound]{Compound requirements}
      
\begin{bnf}
\nontermdef{compound-requirement}\br
  \terminal{\{} expression \terminal{\}} \terminal{noexcept}\opt 
    trailing-return-type\opt~\terminal{;}
\end{bnf}

\pnum
A \grammarterm{compound-requirement} introduces 
a conjunction of one or more atomic constraints for the
\grammarterm{expression} \tcode{E}:
%
\begin{itemize}
\item the \grammarterm{compound-requirement} introduces an expression 
constraint for \tcode{E} (\ref{temp.constr.atom.expr});

\item if the \tcode{noexcept} specifier is present, the 
\grammarterm{compound-requirement} appends an exception constraint for 
\tcode{E} (\ref{temp.constr.atom.noexcept});

\item if the \grammarterm{trailing-return-type} is given and its 
type \tcode{T} contains no placeholders (\ref{dcl.spec.auto}, 
\ref{dcl.spec.constr}), the requirement appends two constraints to the 
conjunction of constraints:
a type constraint on the formation of \tcode{T} (\ref{temp.constr.atom.type}) and
an implicit conversion constraint from \tcode{E} to \tcode{T}
(\ref{temp.constr.atom.conv}).
%
Otherwise, if \tcode{T} contains placeholders, an argument deduction 
constraint (\ref{temp.constr.atom.deduct}) of \tcode{E} against the 
type \tcode{T} is appended to the conjunction of constraints.
\end{itemize}
%
\enterexample
\begin{codeblock}
template<typename T> concept bool C1 =
  requires(T x) {
    {x++};
  };
\end{codeblock}
The \grammarterm{compound-requirement} in \tcode{C1} 
introduces an expression constraint on \tcode{x++}.
It is equivalent to a \grammarterm{simple-requirement}
with the same \grammarterm{expression}.

\begin{codeblock}
template<typename T> concept bool C2 =
  requires(T x) {
    {*x} -> typename T::inner;
  };
\end{codeblock}

The \grammarterm{compound-requirement} in \tcode{C2} 
introduces three constraints: an expression constraint for \tcode{*x}, 
a type constraint for \tcode{typename T::inner}, and a conversion 
constraint requiring \tcode{*x} to be implicitly convertible to
\tcode{typename T::inner}.

\begin{codeblock}
template<typename T> concept bool C3 =
  requires(T x) {
    {g(x)} noexcept;
  };
\end{codeblock}

The \grammarterm{compound-requirement} in \tcode{C3} 
introduces two constraints: an expression constraint for \tcode{g(x)} 
and an exception constraint on \tcode{g(x)}.

\begin{codeblock}
template<typename T> concept bool C() { return true; }

template<typename T> concept bool C5 =
  requires(T x) {
    {f(x)} -> const C&;
  };
\end{codeblock}

The \grammarterm{compound-requirement} in \tcode{C5} 
introduces two constraints: expression constraint for \tcode{f(x)}, 
and a deduction constraint requiring that overload resolution succeeds for the
call \tcode{g(f(x))} where \tcode{g} is the following
invented abbreviated function template.
\begin{codeblock}
void g(const C&);
\end{codeblock}
\exitexample

%%
%% Nested requirements
%%
\rSec3[expr.prim.req.nested]{Nested requirements}

\begin{bnf}
\nontermdef{nested-requirement}\br
    requires-clause \terminal{;}
  \end{bnf}

\pnum
A \grammarterm{nested-requirement} can be used
to specify additional constraints in terms of local parameters.
%
A \grammarterm{nested-requirement} appends the normalized form 
(\ref{temp.constr.expr}) of its \grammarterm{constraint-expression} 
to the conjunction of constraints introduced by its enclosing
\grammarterm{requires-expression}.

\enterexample
\begin{codeblock}
template<typename T> concept bool C() { return sizeof(T) == 1; }

template<typename T> concept bool D =
  requires (T t) {
    requires C<decltype (+t)>();
  };
\end{codeblock}
The \grammarterm{nested-requirement} appends the predicate constraint 
\tcode{sizeof(decltype (+t)) == 1} (\ref{temp.constr.atom.pred}).
\exitexample

\end{quote}