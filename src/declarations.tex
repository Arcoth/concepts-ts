
%%
%% Declarations
%%
\setcounter{chapter}{6}
\rSec0[dcl.dcl]{Declarations}


%%
%% Specifiers
%%
\rSec1[dcl.spec]{Specifiers}

Extend the \grammarterm{decl-specifier} production
to include the \tcode{concept} specifier.

\begin{quote}
\begin{bnf}
\nontermdef{decl-specifier}\br
    \added{\terminal{concept}}
\end{bnf}
\end{quote}

%%
%% Type specifiers
%%
\setcounter{subsection}{5}
\rSec2[dcl.type]{Type specifiers}


%%
%% Simple type specifiers
%%
\setcounter{subsubsection}{1}
\rSec3[dcl.type.simple]{Simple type specifiers}
        
Add \grammarterm{constrained-type-specifier}
to the grammar for \grammarterm{simple-type-specifier}{}s.

\begin{quote}
\begin{bnf}
\nontermdef{simple-type-specifier}\br
    \added{constrained-type-specifier}
\end{bnf}
\end{quote}

%%
%% Auto specifier
%%
\setcounter{subsubsection}{3}
\rSec3[dcl.spec.auto]{\tcode{auto} specifier}

Modify paragraph 1 to extend the use of \tcode{auto} to
designate abbreviated functions.

\begin{quote}
\setcounter{Paras}{0}
\pnum
The \tcode{auto} and \tcode{decltype(auto)} 
\grammarterm{type-specifier}s designate a 
placeholder type that will be replaced later, either by deduction from 
an initializer or by explicit specification with a 
\grammarterm{trailing-return-type}. The 
\tcode{auto} \grammarterm{type-specifier}
is also used to signify that a lambda is a generic lambda\added{ or
that a function declaration is an abbreviated function}.
% 
\enternote
The use of the \tcode{auto} 
\grammarterm{type-specifier} in a non-deduced 
context (\cxxref{temp.deduct.type}) will cause the 
deduction of a value for that placeholder type to fail, resulting
in an ill-formed program.
\exitnote
% 
\enterexample
\begin{codeblock}
struct N {
  template<typename T> struct Wrap;
  template<typename T> static Wrap<T> make_wrap(T);
};
template<typename T, typename U> struct Pair;
template<typename T, typename U> Pair<T, U> make_pair(T, U);

template<int N> struct Size { void f(int) { }  };
Size<0> s;
bool g(char, double);

void (auto::*)(auto) p = &Size<0>::f;   // OK
N::Wrap<auto> a = N::make_wrap(0.0);    // OK
Pair<auto, auto> p = make_pair(0, 'a'); // OK
auto::Wrap<int> x = N::make_wrap(0);    // error: failed to deduce value for auto
Size<sizeof(auto)> y = s;               // error: failed to deduce value for auto
\end{codeblock}
\exitexample
\end{quote}

Modfify paragraph 2, allowing multiple occurrences of \tcode{auto}
in those contexts where it is valid.

\begin{quote}
\setcounter{Paras}{1}
\pnum
\removed{The}\added{A} placeholder type can appear with a function 
declarator in the 
\grammarterm{decl-specifier-seq}, 
\grammarterm{type-specifier-seq},
\grammarterm{conversion-function-id}, or
\grammarterm{trailing-return-type},
in any context where such a declarator is valid. 

If the function declarator includes a 
\grammarterm{trailing-return-type}
(\ref{dcl.fct}),
that specifies the declared return type of the function.
% 
If the declared return type of the function contains a placeholder 
type, the return type of the function is deduced from return 
statements in the body of the function, if any.
\end{quote}

Modify paragraph 3 to allow the use of \tcode{auto} within the 
parameter type of a lambda or function.

\begin{quote}
\setcounter{Paras}{2}
\pnum
If the \tcode{auto} \grammarterm{type-specifier}
appears
\removed{as one of the \grammarterm{decl-specifier}s
in the \grammarterm{decl-specifier-seq} of
a \grammarterm{parameter-declaration}}
\added{in a parameter type} 
of a
\grammarterm{lambda-expression}, the lambda is
a generic lambda (\ref{expr.prim.lambda}).
% 
\enterexample
\begin{codeblock}
auto glambda = [](int i, auto a) { return i; }; // OK: a generic lambda
\end{codeblock}
\exitexample
% 
\begin{addedblock}
Similarly, if the \tcode{auto} \grammarterm{type-specifier}
appears in a parameter type of a function declarator, that is an 
abbreviated function (\ref{dcl.fct}).
% 
\enterexample
\begin{codeblock}
void f(const auto&, int); // OK: an abbreviated function
\end{codeblock}
\exitexample
\end{addedblock}
\end{quote}

Add the following after paragraph 3 to allow the use of 
\tcode{auto} in the 
\grammarterm{trailing-return-type} of a 
\grammarterm{compound-requirement}.

\begin{quote}
\setcounter{Paras}{2}
\pnum
If the \tcode{auto} \grammarterm{type-specifier} 
appears in the \grammarterm{trailing-return-type} 
of a \grammarterm{compound-requirement} in
a \grammarterm{requires-expression}
(\ref{expr.prim.req.compound}), that
return type introduces an argument deduction constraint
(\ref{temp.constr.atom.deduct}).
% 
\enterexample
\begin{codeblock}
template<typename T> concept bool C() {
  return requires (T i) { 
    {*i} -> const auto&; // OK: introduces an argument deduction constraint
  };
}
\end{codeblock}
\exitexample
\end{quote}

Modify paragraph 4 to allow multiple uses of \tcode{auto} within
certain declarations. Note that the examples in the original text are 
unchanged and therefore omitted.

\begin{quote}
\setcounter{Paras}{3}
\pnum
The type of a variable declared using \tcode{auto} or 
\tcode{decltype(auto)} is deduced from its initializer. This use 
is allowed when declaring variables 
in a block 
(\cxxref{stmt.block}), 
in namespace scope
(\cxxref{basic.scope.namespace}), and 
in a \grammarterm{for-init-statement} 
(\cxxref{stmt.for}).
% 
\removed{
\tcode{auto} or \tcode{decltype(auto)} shall appear as one 
of the \grammarterm{decl-specifier}s in the 
\grammarterm{decl-specifier-seq}
}
% 
\added{
Either \tcode{auto} shall appear in the 
\grammarterm{decl-specifier-seq}, or
\tcode{decltype(auto)} shall appear as one of the
\grammarterm{decl-specifier}s of the
\grammarterm{decl-specifier-seq}.
}
% 
\removed{and the}\added{The} \grammarterm{decl-specifier-seq} 
shall be followed by one or more \grammarterm{init-declarator}s,
each of which shall have a non-empty initializer.
% 
In an initializer of the form
\begin{codeblock}
( expression-list )
\end{codeblock}
the \grammarterm{expression-list} shall be a 
single \grammarterm{assignment-expression}.
\end{quote}

Update the rules in paragraph 7 to allow deduction of multiple
occurrences of \tcode{auto} in a declaration.

\begin{quote}
\setcounter{Paras}{6}
\pnum
When a variable declared using a placeholder type is initialized, or 
a \tcode{return} statement occurs in a function declared with a 
return type that contains a placeholder type, the deduced return type 
or variable type is determined from the type of its initializer. 
% 
In the case of a return with no operand, the initializer is
considered to be \tcode{void()}. 
% 
Let \tcode{T} be the declared type of the variable or return 
type of the function. 
% 
\removed{If the placeholder is the \tcode{auto} 
\grammarterm{type-specifier},}
\added{If \tcode{T} contains any occurrences of the \tcode{auto}
\grammarterm{type-specifier},}
the deduced type is determined using the rules for template argument
deduction. 
% 
If the deduction is for a return statement and the initializer is a 
\grammarterm{braced-init-list} 
(\cxxref{dcl.init.list}), the program is 
ill-formed. 
% 
\removed{
Otherwise, obtain \tcode{P} from \tcode{T} by replacing the 
occurrences of \tcode{auto} with either a new invented type 
template parameter \tcode{U} or, if the initializer is a 
\grammarterm{braced-init-list}, with 
\tcode{std::initializer_list<U>}.
}
% 
\begin{addedblock}
Otherwise, obtain \tcode{P} from \tcode{T} as follows:

\begin{itemize}
\item replace each occurrence of \tcode{auto} in the variable type 
with a new invented type template parameter, or 

\item when the initializer is a \grammarterm{braced-init-list}
and \tcode{auto} is a \grammarterm{decl-specifier}
of the \grammarterm{decl-specifier-seq} of
the variable declaration, replace that occurrence of \tcode{auto} 
with \tcode{std::initializer_list<U>}
where \tcode{U} is an invented template type parameter.
\end{itemize}
\end{addedblock}
% 
Deduce a value for 
\removed{\tcode{U}}
\added{each invented template type parameter in \tcode{P}}
using the rules of template argument 
deduction from a function call 
(\cxxref{temp.deduct.call}),
where \tcode{P} is a function template parameter type and the 
initializer is the corresponding argument. 
% 
If the deduction fails, the declaration is ill-formed. 
% 
Otherwise, the type deduced for the variable or return type is obtained 
by substituting the deduced 
\removed{\tcode{U}}
\added{values for each invented template parameter} 
into \tcode{P}.
% 

% \enterexample
% \begin{codeblock}
% \added{template<typename T> struct Vec { };
% template<typename T> Vec<T> make_vec(std::initalizer_list<T>) { return Vec<T>{}; } }

% auto x1 = { 1, 2 };                 // \added{OK:} decltype(x1) is std::initializer_list<int>
% auto x2 = { 1, 2.0 };               // error: cannot deduce element type
% \end{codeblock}
% \begin{addedblock}
% \begin{codeblock}
% auto& x3 = *x1.begin();             // OK: decltype(x3) is int\&
% const auto* p = &x3;                // OK: decltype(p) is const int*
% Vec<auto> v1 = make_vec({1, 2, 3}); // OK: decltype(v1) is Vec<int>
% Vec<auto> v2 = {1, 2, 3};           // error: cannot deduce element type
% \end{codeblock}
% \end{addedblock}
% \exitexample

% NOTE: added/removed does not work in examples. The markup in this
% example is lost...
\enterexample
\begin{codeblock}
template<typename T> struct Vec { };
template<typename T> Vec<T> make_vec(std::initalizer_list<T>) { return Vec<T>{}; } }

auto x1 = { 1, 2 };                 // OK: \tcode{decltype(x1)} is \tcode{std::initializer_list<int>}
auto x2 = { 1, 2.0 };               // error: cannot deduce element type
auto& x3 = *x1.begin();             // OK: \tcode{decltype(x3)} is \tcode{int\&}
const auto* p = &x3;                // OK: \tcode{decltype(p)} is \tcode{const int*}
Vec<auto> v1 = make_vec({1, 2, 3}); // OK: \tcode{decltype(v1)} is \tcode{Vec<int>}
Vec<auto> v2 = {1, 2, 3};           // error: cannot deduce element type
\end{codeblock}
\exitexample
% 
\enterexample
\begin{codeblock}
const auto& i = expr;
\end{codeblock}
The type of \tcode{i} is the deduced type of the parameter 
\tcode{u} in the call \tcode{f(expr)} of the following 
invented function template:
\begin{codeblock}
template <class U> void f(const U& u);
\end{codeblock}
\exitexample
% 
\begin{addedblock}
\enterexample
Similarly, the type of \tcode{p} in the following program
\begin{codeblock}
template<typename F, typename S> struct Pair;
template<typename T, typename U> Pair<T, U> make_pair(T, U);

struct S { void mfn(bool); } s;
int fn(char, double);

Pair<auto (*)(auto, auto), auto (auto::*)(auto)> p = make_pair(fn, &S::mfn);
\end{codeblock}
is the deduced type of the parameter \tcode{x} in the call of 
\tcode{g(make_pair(fn, \&S::mfn))} of the following invented function 
template:
\begin{codeblock}
template<class T1, class T2, class T2, class T3, class T4, class T5, class T6> 
  void g(Pair< T1(*)(T2, T3), T4 (T5::*)(T6)>);
\end{codeblock}
\exitexample
\end{addedblock}
\end{quote}


%%
%% Constrained type specifiers
%%
\rSec3[dcl.spec.constr]{Constrained type specifiers}

Add this section to \cxxref{dcl.type}.

\pnum
Like the \tcode{auto} \grammarterm{type-specifier}
(\ref{dcl.spec.auto}),
a \grammarterm{constrained-type-specifier}
designates a placeholder that will be replaced later by deduction
from the \grammarterm{expression} in a 
\grammarterm{compound-requirement} or a
function argument. 
% 
\grammarterm{constrained-type-specifier}s have
the form
% 
\begin{bnf}
\nontermdef{constrained-type-specifier}\br
  nested-name-specifier\opt constrained-type-name

\nontermdef{constrained-type-name}\br
  concept-name\br
  partial-concept-id

\nontermdef{concept-name}\br
  identifier

\nontermdef{partial-concept-id}\br
    concept-name \terminal{<} template-argument-list\opt \terminal{>}
\end{bnf}
% 
A \grammarterm{constrained-type-specifier}
may also designate placeholders for deduced non-type and template 
template arguments.
\enterexample
\begin{codeblock}
template<typename T> concept bool C1 = false;
template<int N> concept bool C2 = false;
template<template<typename> class X> C3 = false;

template<typename T, int N> class Array { };
template<typename T, template<typename> class A> class Stack { };
template<typename T> class Alloc { };

void f1(C1 c);            // C1 designates a placeholder type
void f2(Array<auto, C2>); // C2 designates a placeholder for an integer value
void f3(Stack<auto, C3>); // C3 designates a placeholder for a class template
\end{codeblock}
\exitexample

\pnum
A \grammarterm{constrained-type-specifier}
can appear in many of the same places as the \tcode{auto}
\grammarterm{type-specifier}, is subject
to the same rules, and has equivalent meaning in those contexts. 
In particular, a \grammarterm{constrained-type-specifier} 
can appear in the following contexts with the given meaning:
\begin{itemize}
\item a parameter type of a function declaration, signifying an 
     abbreviated function (\ref{dcl.fct});
\item a parameter of a lambda, signifying a generic lambda 
     (\ref{expr.prim.lambda});
\item the parameter type of a template parameter, signifying a
     constrained template parameter (\ref{temp.param});
\item the \grammarterm{trailing-return-type} of
     a \grammarterm{compound-requirement}
     (\ref{expr.prim.req.compound}),
     signifying an argument deduction constraint 
     (\ref{temp.constr.atom.deduct}). 
\end{itemize}
A program that includes a \grammarterm{constrained-type-specifier}
in any other contexts is ill-formed.
% 
\enterexample
\begin{codeblock}
template<typename T> concept bool C1 = true;
template<typename T, int N> concept bool C2 = true;
template<bool (*)(int)> concept bool C3 = true;

template<typename T> class Vec;

struct N {
  template<typename T> struct Wrap;
}
template<typename T, typename U> struct Pair;
template<bool (*)(int)> struct Pred;

auto gl = [](C1& a, C1* b) { a = *b; }; // OK: a generic lambda
void af(const Vec<C1>& x);              // OK: an abbreviated function

void f1(N::Wrap<C1>);     // OK
void f2(Pair<C1, C2<0>>); // OK
void f3(Pred<C3>);        // OK
void f4(C1::Wrap<C2<1>>); // OK: but deduction of C1 will always fail

template<typename T> concept bool Iter() {
  return requires(T i) {
    {*i} -> const C1&; // OK: an argument deduction constraint
  };
}
\end{codeblock}
The declaration of \tcode{f4} is valid, but a value can never be deduced 
for the placeholder designated by \tcode{C1} since it appears in a
non-deduced context (\cxxref{temp.deduct.type}).
However, a value may be explicitly given as a template argument in a
\grammarterm{template-id}.
\exitexample

\pnum
\enternote
Unlike \tcode{auto}, a 
\grammarterm{constrained-type-specifier} cannot 
be used in the type of a variable declaration or the return type of
a function.
<cxx-example>
\begin{codeblock}
template<typename T> concept bool C = true;
template<typename T, int N> concept bool D = true;

const C* x = 0;  // error: C used in a variable type
D<0> fn(int x);  // error: D<0> used as a return type
\end{codeblock}
\exitexample
\exitnote

\pnum
A \grammarterm{concept-name} refers to a
set of concept definitions (\ref{dcl.spec.concept})
called the \defn{candidate concept set}.
% 
If that set is empty, the program is ill-formed.
% 
\enternote
The candidate concept set has multiple members
only when referring to a set of overloaded function concepts.
There is at most one member of this set when a 
\grammarterm{concept-name}
refers to a variable concept.
\exitnote
% 
\enterexample
\begin{codeblock}
template<typename T> concept bool C() { return true; }             // \#1
template<typename T, typename U> concept bool C() { return true; } // \#2
template<typename T> concept bool D = true;                        // \#3

void f(C); // OK: the concept-name C may refer to both \#1 and \#2
void g(D); // OK: the concept-name D refers only to \#3
\end{codeblock}
\exitexample

\pnum
A \grammarterm{partial-concept-id} is a
\grammarterm{concept-name} followed by a
sequence of \grammarterm{template-argument}s.
% 
\enterexample
\begin{codeblock}
template<typename T, typename U> concept bool C() { return true; }
template<typename T, int N = 0> concept bool Seq = true;

void f(C<int>);
void f(Seq<3>);
void f(Seq<>);
\end{codeblock}
\exitexample

\pnum
The concept designated by a 
\grammarterm{constrained-type-specifier}
is resolved by determining the viability of each concept in the
candidate concept set.
% 
For each candidate concept in that set, \tcode{TT} is a 
\grammarterm{template-id} formed as follows:
% 
let \tcode{C} be the \grammarterm{concept-name}
for the candidate concept set, and let \tcode{X} be a template 
argument that matches
the type and form (\ref{temp.arg}) of the prototype 
parameter (\ref{dcl.spec.concept})
of the candidate concept. 
% 
The template \tcode{X} is called the 
\defn{deduced concept argument}.
% 
If the \grammarterm{constrained-type-name} in
the \grammarterm{constrained-type-specifier} is
a \grammarterm{concept-name}, \tcode{TT} is
formed as \tcode{C<X>}.
% 
Otherwise, the \grammarterm{constrained-type-name} is
a \grammarterm{partial-concept-id} whose
\grammarterm{template-argument-list} is
\tcode{A$_1$, A$_2$, ..., A$_n$}, and
\tcode{TT} is formed as
\tcode{C<X, A$_1$, A$_2$, ..., A$_n$>}.
% 
The candidate concept is a \defn{viable candidate concept} when all
\grammarterm{template-arguments} of \tcode{TT} 
match the template parameters of that candidate 
(\ref{temp.arg}).
% 
If, after determining the viability of each concept, there is a
single viable candidate concept, that is the concept designated by the
\grammarterm{constrained-type-specifier}.
% 
Otherwise, the program is ill-formed.
% 
\enterexample
\begin{codeblock}
template<typename T> concept bool C() { return true; }             // \#1
template<typename T, typename U> concept bool C() { return true; } // \#2
template<typename T> concept bool P() { return true; }
template<int T> concept bool P() { return true; }

void f1(const C*); // OK: C designates \#1
void f2(C<char>);  // OK: C<char> designates \#2
void f3(C<3>);     // error: no matching concept for C<3> (mismatched template arguments)
void g1(P);        // error: resolution of P is ambiguous (P refers to two concepts)
\end{codeblock}
\exitexample


\pnum
The use of a \grammarterm{constrained-type-specifier}
in the type of a \grammarterm{parameter-declaration}
associates a \grammarterm{constraint-expression} 
(\ref{temp.constr.expr}) with the 
entity for which that parameter is declared. 
% 
In the case of a generic lambda, the association is with
the member function call operator of the closure type 
(\ref{expr.prim.lambda}).
% 
For an abbreviated function declaration, the association is with
with that function (\ref{dcl.fct}).
% 
When a \grammarterm{constrained-type-specifier}
appears in the 
\grammarterm{trailing-return-type} of a 
\grammarterm{compound-requirement} it
associates a \grammarterm{constraint-expression} 
with the invented function template of the introduced argument 
deduction constraint (\ref{temp.constr.atom.deduct}).

When a \grammarterm{constrained-type-specifier}
is used in a \grammarterm{template-parameter},
the constrained it associates a 
\grammarterm{constraint-expression} with the
\grammarterm{template-declaration} in which the
\grammarterm{template-parameter} is declared.


\pnum
The \grammarterm{constraint-expression} 
associated by a \grammarterm{constrained-type-specifier}
is derived from the \grammarterm{template-id} 
(called \tcode{TT} above) used to determine the viability of
the designated concept (call it \tcode{D}).
% 
The constraint is formed by replacing the
deduced concept argument \tcode{X} in \tcode{TT} with a 
template argument, \tcode{A}. That template argument is
defined as follows:
% 
\begin{itemize}
\item when the \grammarterm{constrained-type-specifier}
appears in the type of a \grammarterm{parameter-declaration}
of a function or lambda, 
% 
or when the when the 
\grammarterm{constrained-type-specifier}
appears in the \grammarterm{trailing-return-type}
of a \grammarterm{compound-requirement},
% 
\tcode{A} is the name of the
invented template parameter corresponding to the 
placeholder in that (possibly invented) function template 
(\ref{dcl.fct});

\item when the \grammarterm{constrained-type-specifier}
appears in a \grammarterm{template-parameter}
declaration, \tcode{A} is the name of the declared
parameter (\ref{temp.param}). 
\end{itemize}
% 
Let \tcode{TT2} be a \grammarterm{template-id}
formed as follows. 
% 
If the template parameter \tcode{A} declares a template parameter 
pack, and \tcode{D} is a variadic concept 
(\ref{dcl.spec.concept}),
\tcode{TT2} is formed by replacing the deduced concept
argument \tcode{X} in 
\tcode{TT} with the pack expansion \tcode{A...}.
% 
Otherwise \tcode{TT2} is formed by replacing \tcode{X}
with \tcode{A}.
% 
Let \tcode{E} be the \grammarterm{id-expression}
\tcode{TT2} when the \tcode{D} is a variable concept, and
the function call \tcode{TT2()} when the \tcode{D} is a 
function concept.
% 
If the template parameter \tcode{A} declares a template
parameter pack, and \tcode{D} is not a variadic concept,
then the associated constraint is the fold expression 
\tcode{(E \& ...)}
(\cxxref{expr.prim.fold}).
% 
Otherwise, the associated constraint is the expression \tcode{E}.
% 
\enterexample
\begin{codeblock}
template<typename T> concept bool C = true;
template<typename T, typename U> concept bool D() { return true; }
template<int N> concept bool Num = true;

template<int> struct X { };

void f(C&);              // associates C<T1> with f
void g(D<int>);          // associates D<T2, int>() with g
void h(X<Num>);          // associates Num<M> with h
template<C T> struct s1; // associates C<T> with s1
\end{codeblock}
Here, \tcode{T1} and \tcode{T2} are invented type template 
parameters corresponding to the
prototype parameter of their respective designated concepts.
Likewise, \tcode{M} is an invented non-type template parameter
corresponding to the prototype parameter of \tcode{Num}.

\pnum
Two \grammarterm{constrained-type-specifiers}
are said to be equivalent if their associated
\grammarterm{constraint-expression}s are
equivalent according to the rules in 
\ref{temp.constr.expr}.



%%
%% concept specifier
%%
\rSec2[dcl.spec.concept]{\tcode{concept} specifier}

\pnum
The \tcode{concept} specifier shall be applied only to the 
definition of a function template or variable template, declared
in namespace scope (\cxxref{basic.scope.namespace}).
% 
A function template definition having the \tcode{concept}
specifier is called a \defn{function concept}. A function
concept is a non-throwing function 
(\cxxref{except.spec}).
% 
A variable template definition having the \tcode{concept} 
specifier is called a \defn{variable concept}.
% 
A \defn{concept definition} refers to either a function concept 
and its definition or a variable concept and its initializer.
% 
\enterexample
\begin{codeblock}
template<typename T> 
  concept bool F1() { return true; } // OK: declares a function concept
template<typename T> 
  concept bool F2();                 // error: function concept is not a definition
template<typename T> 
  concept bool V1 = true;            // OK: declares a variable concept
template<typename T> 
  concept bool V2;                   // error: variable concept with no initializer
struct S {
  template<typename T> 
    static concept bool C = true;    // error: concept declared in class scope
};
\end{codeblock}
\exitexample

\pnum
Every concept definition is implicitly defined to be a 
\tcode{constexpr} declaration 
(\cxxref{dcl.constexpr}). 

\pnum
No storage specifiers shall appear in a declaration with the 
\tcode{concept} specifier. Additionally, a concept definition
shall not include the \tcode{friend} or \tcode{constexpr}
specifiers. 

\pnum
A concept definition shall be unconstrained.
\enternote
A concept defines a total mapping from its template arguments to the 
values \tcode{true} and \tcode{false}.
\exitnote

\pnum
The first declared template parameter of a concept definition is its
\defn{prototype parameter}. 
% 
A \defn{variadic concept} is a concept whose prototype parameter
is a template parameter pack.


\pnum
A function concept has the following restrictions:
\begin{itemize}
  \item No \grammarterm{function-specifier}s shall
       appear in its declaration (\ref{dcl.fct.spec}). 
  \item The declared return type shall be the non-deduced type \tcode{bool}. 
  \item The declaration shall have a 
  \grammarterm{parameter-declaration-clause}
  equivalent to \tcode{()}. 
  \item The function shall not be recursive. 
  \item The function body shall consist of a single \tcode{return} 
  statement whose \grammarterm{expression} shall be a
  \grammarterm{constraint-expression}
  (\ref{temp.constr.expr}).
\end{itemize}
% 
\enternote
Return type deduction requires the instantiation of the function 
definition, but concept definitions are not instantiated; they
are normalized (\ref{temp.constr.expr}).
\exitnote
% 
\enterexample
\begin{codeblock}
template<typename T> 
  concept int F1() { return 0; }      // error: return type is not bool
template<typename T> 
  concept auto F3(T) { return true; } // error: return type is deduced
template<typename T> 
  concept bool F2(T) { return true; } // error: must have no parameters
\end{codeblock}
\exitexample

\pnum
A variable concept has the following restrictions:
\begin{itemize}
  \item The declared type shall be \tcode{bool}. 
  \item The declaration shall have an initializer. 
  \item The initializer shall be a 
  \grammarterm{constraint-expression}. 
\end{itemize}
\enterexample
\begin{codeblock}
template<typename T> 
  concept bool V2 = 3 + 4; // error: initializer is not a constraint-expression
template<Integral T> 
  concept bool V3 = true;  // error: constrained template declared as a concept
concept bool V4 = 0;       // error: not a template
\end{codeblock}
\exitexample

\pnum
A program that declares an explicit instantiation
(\ref{temp.explicit}), an explicit specialization 
(\ref{temp.expl.spec}), or a partial specialization 
of a concept definition is ill-formed.

\enternote
The prohibitions against overloading and specialization prevent
users from subverting the constraint system by providing a meaning 
for a concept that differs from the one computed by evaluating its 
constraints.
\exitnote

