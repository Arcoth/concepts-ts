
%%
%% Declarations
%%
\setcounter{chapter}{6}
\rSec0[dcl.dcl]{Declarations}


%%
%% Specifiers
%%
\rSec1[dcl.spec]{Specifiers}

Extend the \grammarterm{decl-specifier} production
in paragraph 1 to include the \tcode{concept} specifier.

\begin{quote}
\pnum
The specifiers that can be used in a declaration are

\begin{bnf}
\nontermdef{decl-specifier}\br
    storage-class-specifier\br
    type-specifier\br
    function-specifier\br
    \terminal{friend}\br
    \terminal{typedef}\br
    \terminal{constexpr}\br
    \added{\terminal{concept}}
\end{bnf}
\end{quote}


%%
%% Type specifiers
%%
\setcounter{subsection}{5}
\rSec2[dcl.type]{Type specifiers}


%%
%% Simple type specifiers
%%
\setcounter{subsubsection}{1}
\rSec3[dcl.type.simple]{Simple type specifiers}
        
Add \grammarterm{constrained-type-specifier}
to the grammar for \grammarterm{simple-type-specifier}{}s.

\begin{quote}
\begin{bnf}
\nontermdef{simple-type-specifier}\br
    nested-name-specifier\opt type-name\br
    nested-name-specifier \terminal{template} simple-template-id\br
    \terminal{char}\br
    \terminal{char16_t}\br
    \terminal{char32_t}\br
    \terminal{wchar_t}\br
    \terminal{bool}\br
    \terminal{short}\br
    \terminal{int}\br
    \terminal{long}\br
    \terminal{signed}\br
    \terminal{unsigned}\br
    \terminal{float}\br
    \terminal{double}\br
    \terminal{void}\br
    \terminal{auto}\br
    decltype-specifier\br
    \added{constrained-type-specifier}
\end{bnf}
\end{quote}

Modify paragraph 2 to begin:

\begin{quote}
\pnum
\removed{The \tcode{auto} specifier is a placeholder for a type to be deduced 
(\ref{dcl.spec.auto}).}
\added{The \tcode{auto} specifier and \grammarterm{constrained-type-specifier}{s} 
are placeholders for values (type, non-type, template) to be deduced 
(\ref{dcl.spec.auto}).}
\end{quote}

Add \grammarterm{constrained-type-specifier}{s} to the table of
\grammarterm{simple-type-specifier}{s} in Table~\ref{tab:simple.type.specifiers}.

\renewcommand{\thetable}{\arabic{table}}
\setcounter{table}{9}
\begin{simpletypetable}
{\grammarterm{simple-type-specifier}{s} and the types they specify}
{tab:simple.type.specifiers}
{ll}
\topline
Specifier(s)                                     &   Type                                                                 \\ \capsep
\grammarterm{type-name}                          &   the type named                                                       \\
\grammarterm{simple-template-id}                 &   the type as defined in~\ref{temp.names}                              \\
\multicolumn{2}{|c|}{\vdots}                                                                                              \\
auto                                             & placeholder for a type to be deduced                                   \\
decltype(\grammarterm{expression})               & the type as defined below                                              \\
\added{\grammarterm{constrained-type-specifier}} & \added{placeholder for value (type, non-type, template) to be deduced} \\
\end{simpletypetable}


%%
%% Auto specifier
%%
\setcounter{subsubsection}{3}
\rSec3[dcl.spec.auto]{\tcode{auto} specifier}

Extend this section to allow for \grammarterm{constrained-type-specifier}{s}
as a new syntax for designating placeholders. The section is refactored so
that placeholders are introduced in this section, deduction rules are
defined in subsection \ref{dcl.spec.auto.deduct}, and the meaning of
\grammarterm{constrained-type-specifier}{s} is described in
\ref{dcl.spec.auto.constr}.

Replace paragraph 1 with the text below.

\begin{quote}
\pnum
The \grammarterm{type-specifier}{}s \tcode{auto} and \tcode{decltype(auto)}
and \grammarterm{constrained-type-specifier}{}s designate a placeholder
(type, non-type, or template) that will be replaced later, either through 
deduction or an explicit specification.
%
The \tcode{auto} and \tcode{decltype(auto)} \grammarterm{type-specifier}{}s 
designate placeholder types; a \grammarterm{constrained-type-specifier} can 
also designate placeholders for values and templates. 
%
Placeholders are also used to signify that a lambda is a generic lambda 
(\ref{expr.prim.lambda}), that a function declaration is an
abbreviated function template (\ref{dcl.fct}), or that a 
\grammarterm{trailing-return-type} in a \grammarterm{compound-requirement}
(\ref{expr.prim.req.compound}) introduces an argument deduction constraint 
(\ref{temp.constr.deduct}).
%
\enternote
A \grammarterm{nested-name-specifier} can also include placeholders (\ref{expr.prim}).
Replacements for those placeholders are determined according to the rules
in this section.
\exitnote
% 
\end{quote}

Modify paragraph 2 to allow \grammarterm{constrained-type-specifier}{}s
with function declarators, except in the declared return type.

\begin{quote}
\pnum
\removed{The placeholder type}\added{Placeholders} can appear with a function 
declarator in the \grammarterm{decl-specifier-seq}, \grammarterm{type-specifier-seq},
\grammarterm{conversion-function-id}, or \grammarterm{trailing-return-type}, 
in any context where such a declarator is valid. 
% 
If the function declarator includes a \grammarterm{trailing-return-type} 
(\ref{dcl.fct}), that specifies the declared return type of the function.
% 
If the declared return type of the function contains a placeholder, the 
return type of the function is deduced from return statements in the body of 
the function, if any.
%
\added{
In a function declarator of the form \tcode{auto D -> T} where
\tcode{T} contains placeholders, the initial \tcode{auto} does 
not designate a placeholder.
}
\end{quote}

Modify paragraph 3 to allow the use of \tcode{auto} within the 
parameter type of a lambda or function.

\begin{quote}
\pnum
If \removed{the \tcode{auto} \grammarterm{type-specifier}} \added{a placeholder}
appears \removed{as one of the \grammarterm{decl-specifier}{}s in the 
\grammarterm{decl-specifier-seq} of a \grammarterm{parameter-declaration}} 
\added{in a parameter type} of a \grammarterm{lambda-expression}, the lambda 
is a generic lambda
(\ref{expr.prim.lambda}).
%
\enterexample
\begin{codeblock}
auto glambda = [](int i, auto a) { return i; }; // OK: a generic lambda
\end{codeblock}
\exitexample
%
\begin{addedblock}
Similarly, if a placeholder appears in a parameter type of a function 
declaration, the function declaration declares an abbreviated function 
template (\ref{dcl.fct}).
%
\enterexample
\begin{codeblock}
void f(const auto&, int); // OK: an abbreviated function template
\end{codeblock}
\exitexample
\end{addedblock}
\end{quote}


Add the following after paragraph 3 to allow the use of \tcode{auto} in the
\grammarterm{trailing-return-type} of a \grammarterm{compound-requirement}.
Also, disallow the use of \tcode{decltype(auto)} with function parameters
and deduction constraints.

\begin{quote}
\begin{addedblock}
\pnum
If a placeholder appears in the \grammarterm{trailing-return-type}
of a \grammarterm{compound-requirement} in a \grammarterm{requires-expression} 
(\ref{expr.prim.req.compound}), that return type introduces an argument 
deduction constraint (\ref{temp.constr.deduct}).
% 
\enterexample
\begin{codeblock}
template<typename T> concept bool C() {
  return requires (T i) { 
    {*i} -> const auto&; // OK: introduces an argument deduction constraint
  };
}
\end{codeblock}
\exitexample

\pnum
The \tcode{decltype(auto)} \grammarterm{type-specifier} shall not appear
in the declared type of a \grammarterm{parameter-declaration} or the
\grammarterm{trailing-return-type} of a \grammarterm{compound-requirement}.
\end{addedblock}
\end{quote}


Modify paragraph 4 (paragraph 6, here) to allow multiple placeholders within a 
variable declaration, but disallowing \grammarterm{constrained-type-specifier}{}s.

\begin{quote}
\pnum
The type of a variable declared using 
\added{a placeholder} 
\removed{\tcode{auto} or \tcode{decltype(auto)}} 
is deduced from its initializer.
% 
This use is allowed when declaring variables in a block (\cxxref{stmt.block}),
in namespace scope (\cxxref{basic.scope.namespace}), and in a 
\grammarterm{for-init-statement} (\cxxref{stmt.for}).
%
\removed{\tcode{auto} or \tcode{decltype(auto)} shall appear as one
of the \grammarterm{decl-specifier}{}s in the \grammarterm{decl-specifier-seq} }
%
\added{
A placeholder can appear anywhere in the declared type of the variable, but 
\tcode{decltype(auto)} shall appear only as one of the
\grammarterm{decl-specifier}{}s of the \grammarterm{decl-specifier-seq}.
}
\removed{and the}\added{The} \grammarterm{decl-specifier-seq} \added{of such a
variable} shall be followed by one or more \grammarterm{init-declarator}{}s,
each of which shall have a non-empty initializer.
%
In an initializer of the form
\begin{codeblock}
( expression-list )
\end{codeblock}
the \grammarterm{expression-list} shall be a single 
\grammarterm{assignment-expression}.
% 
\enterexample
\begin{codeblock}
auto x = 5;                // OK: \tcode{x} has type int
const auto *v = &x, u = 6; // OK: \tcode{v} has type \tcode{const int*}, \tcode{u} has type \tcode{const int}
static auto y = 0.0;       // OK: \tcode{y} has type \tcode{double}
auto int r;                // error: \tcode{auto} is not a storage-class-specifier
auto f() -> int;           // OK: \tcode{f} returns \tcode{int}
auto g() { return 0.0; }   // OK: \tcode{g} returns \tcode{double}
auto h();                  // OK: \tcode{h}'s return type will be deduced when it is defined
\end{codeblock}
\exitexample
\end{quote}

Add the following declarations to the example in the previous paragraph.

\begin{quote}
\begin{addedblock}
\begin{codeblock}
struct N {
  template<typename T> struct Wrap;
  template<typename T> static Wrap<T> make_wrap(T);
};
template<typename T, typename U> struct Pair;
template<typename T, typename U> Pair<T, U> make_pair(T, U);
template<int N> struct Size { void f(int) { }  };

void (auto::* p1)(auto) = &Size<0>::f;   // OK: \tcode{p1} has type \tcode{void(Size<0>::*)(int)}
Pair<auto, auto> p2 = make_pair(0, 'a'); // OK: \tcode{p2} has type \tcode{Pair<int, char>}
N::Wrap<auto> a = N::make_wrap(0.0);     // OK: \tcode{a} has type \tcode{Wrap<double>}
auto::Wrap<int> x = N::make_wrap(0);     // error: failed to deduce value for \tcode{auto}
Size<sizeof(auto)> y = Size<0>{};        // error: failed to deduce value for \tcode{auto}

template<typename T> concept bool C = true;
template<typename T> concept bool D = false;
C z1 = 0;                // OK: \tcode{z1} has type \tcode{int}
D z2 = 0;                // error: constraints not satisfied
C cf1() { return 0.0; }; // OK: \tcode{cf1} returns \tcode{double}
D cf2() { return 0.0; }; // error: constraints not satisfied
auto cf3() -> C;         // OK: \tcode{cf3}'s return type will be deduced when it is defined
\end{codeblock}
\end{addedblock}
\end{quote}


Update paragraph 6 (paragraph 8, here) to disallow placeholders in
other contexts.

\begin{quote}
\pnum
A program that uses \removed{\tcode{auto} or \tcode{decltype(auto)}}
\added{placeholders} in a context not explicitly allowed in this section is 
ill-formed.
\end{quote}


% 
% Deducing replacements for variables and return types
% 
\rSec4[dcl.spec.auto.deduct]{Deducing replacements for variables and return types}

Factor the deduction rules for \tcode{auto} into a new subsection.

\begin{quote}
\pnum
When a variable declared using a placeholder is initialized, or a 
\tcode{return} statement occurs in a function declared with a return type
that contains a placeholder, the deduced return type or variable type 
is determined from the type of its initializer.
%
In the case of a return with no operand, the initializer is considered to 
be \tcode{void()}.
%
Let \tcode{T} be the declared type of the variable or return type of the 
function.
%
\removed{If the placeholder is the \tcode{auto} 
\grammarterm{type-specifier},}
\added{If \tcode{T} contains any occurrences of the \tcode{auto}
\grammarterm{type-specifier} or a \grammarterm{constrained-type-specifier},}
the deduced type is determined using the rules for template argument
deduction. 
%
If the deduction is for a return statement and the initializer is a 
\grammarterm{braced-init-list} (\cxxref{dcl.init.list}), the program is
ill-formed. 
%
\removed{
Otherwise, obtain \tcode{P} from \tcode{T} by replacing the occurrences of 
\tcode{auto} with either a new invented type template parameter \tcode{U} or, 
if the initializer is a \grammarterm{braced-init-list}, with 
\tcode{std::initializer_list<U>}.
}

% FIXME: Add the ability to deduce tuple<auto...> = make_tuple(args);
\begin{addedblock}
Otherwise, obtain \tcode{P} from \tcode{T} as follows:
\begin{itemize}
\item when the initializer is a \grammarterm{braced-init-list}
and a placeholder is a \grammarterm{decl-specifier} of the 
\grammarterm{decl-specifier-seq} of the variable declaration, replace that 
occurrence of the placeholder with \tcode{std::initializer_list<U>}
where \tcode{U} is an invented type template parameter;

\item otherwise, replace each occurrence of a placeholder in the
variable or return type with a new invented type template parameter
according to the rules for inventing template parameters
for placeholders in \ref{dcl.fct}.
\end{itemize}
\end{addedblock}
% 
Deduce a value for \removed{\tcode{U}} \added{each invented type template  
parameter in \tcode{P}} using the rules of template argument deduction from 
a function call (\cxxref{temp.deduct.call}), where \tcode{P} is a function 
template parameter type and the initializer is the corresponding argument.
%
If the deduction fails, the declaration is ill-formed.
% 
If any placeholders in the declared type were introduced by a
\grammarterm{constrained-type-specifier}, then let \tcode{C} be the
conjunction of predicate constraints introduced by those placeholders.
The ordering of operands in that conjunction is same as the order in which 
the placeholders appear. If these constraints are not satisfied by the
deduced values, the declaration is ill-formed.
% 
Otherwise, the type deduced for the variable or return type is obtained by 
substituting the deduced \removed{\tcode{U}} \added{values for each invented 
template parameter} into \tcode{P}.
% 
\enterexample
\begin{codeblock}
auto x1 = { 1, 2 };                 // \added{OK:} \tcode{decltype(x1)} is \tcode{std::initializer_list<int>}
auto x2 = { 1, 2.0 };               // error: cannot deduce element type
\end{codeblock}
\exitexample
\end{quote}

Add the following to the first example in paragraph 7 in the \Cpp Standard.

\begin{quote}
\begin{addedblock}
\enterexample
\begin{codeblock}
template<typename T> struct Vec { };
template<typename T> Vec<T> make_vec(std::initializer_list<T>) { return Vec<T>{}; }

template<typename... Ts> struct Tuple { };
template<typename... Ts> auto make_tup(Ts... args) { return Tuple<Ts...>{}; }

auto& x3 = *x1.begin();               // OK: \tcode{decltype(x3)} is \tcode{int\&}
const auto* p = &x3;                  // OK: \tcode{decltype(p)} is \tcode{const int*}
Vec<auto> v1 = make_vec({1, 2, 3});   // OK: \tcode{decltype(v1)} is \tcode{Vec<int>}
Vec<auto> v2 = {1, 2, 3};             // error: type deduction fails
Tuple<auto...> v3 = make_tup(0, 'a'); // OK: \tcode{decltype(v3)} is \tcode{Tuple<int, char>}
\end{codeblock}
\exitexample
\end{addedblock}
\end{quote}

Add the following after the second example in paragraph 7 in the \Cpp Standard.

\begin{quote}
\begin{addedblock}
\enterexample
\begin{codeblock}
template<typename F, typename S> struct Pair;
template<typename T, typename U> Pair<T, U> make_pair(T, U);

struct S { void mfn(bool); } s;
int fn(char, double);

Pair<auto (*)(auto, auto), auto (auto::*)(auto)> p = make_pair(fn, &S::mfn);
\end{codeblock}
The declared type of \tcode{p} is the deduced type of the parameter 
\tcode{x} in the call of \tcode{g(make_pair(fn, \&S::mfn))} of the following 
invented function template:
\begin{codeblock}
template<class T1, class T2, class T3, class T4, class T5, class T6>
void g(Pair< T1(*)(T2, T3), T4 (T5::*)(T6)> x);
\end{codeblock}
\exitexample

\enterexample
\begin{codeblock}
template<typename T> concept bool C = true;

const C* cv = expr;
\end{codeblock}
The type of \tcode{cv} is is deduced form the parameter \tcode{p1} in the
call \tcode{f1(expr)} of the following invented function:
\begin{codeblock}
template<C T> void f1(const T* p1);
\end{codeblock}
\exitexample

\enterexample
\begin{codeblock}
auto cf(int) -> Pair<C, C> { return expr; }
\end{codeblock}
The return type of \tcode{cf} is is deduced form the parameter \tcode{p2} in the
call \tcode{f2(expr)} of the following invented function:
\begin{codeblock}
template<C T1, C T2> void f2(Pair<T1, T2>);
\end{codeblock}
\exitexample
\end{addedblock}
\end{quote}

Copy paragraphs 8-15 from \ref{dcl.spec.auto} in the \Cpp Standard into
this section. Modify paragraph 8 (here, 2) to read:

\begin{quote}
\pnum
If the \grammarterm{init-declarator-list} contains more than one 
\grammarterm{init-declarator}, they shall all form declarations of variables.
The type of each declared variable is determined as described above, and if 
the type that replaces \removed{the placeholder type} 
\added{the declared variable type or return type}
is not the same in each deduction, the program is ill-formed.
\end{quote}

Add add the following examples to that paragraph.

\begin{quote}
\enterexample
\begin{addedblock}
\begin{codeblock}
Pair<auto, auto> p1 = make_pair(0, 0), 
                 *p2 = &p1;              // OK: replacement type is \tcode{Pair<int, int>}
Pair<auto, auto> p3 = make_pair(0, 'a'), 
                 p4 = make_pair('a', 0); // error: different replacement types
\end{codeblock}
\end{addedblock}
\exitexample
\end{quote}

Modify paragraph 9 (here, 3).
\begin{quote}
\pnum
If a function with a declared return type that contains 
\removed{a placeholder type} 
\added{placeholders}
has multiple return statements,
the return type is deduced for each return statement. 
If the type deduced is not the same in each deduction,
the program is ill-formed.
\end{quote}

Modify the text of paragraph 10 (here, 4).

\begin{quote}
\pnum
If a function with a declared return type that uses 
\removed{a placeholder type}
\added{placeholders} 
has no return statements, the return type is deduced as though from 
a return statement with no operand at the closing brace of the function
body. 
\end{quote}

Modify the first sentence of paragraph 11 (here, 5).

\begin{quote}
\pnum
If the type of an entity with an undeduced placeholder \removed{type} 
is needed to determine the type of an expression, the program is 
ill-formed.
\end{quote}

Modify the text of paragraph 13 (here, 7).

\begin{quote}
\setcounter{Paras}{6}
\pnum
Redeclarations or specializations of a function or function template 
with a declared return type that uses 
\removed{a placeholder type}
\added{placeholders} shall also use that placeholder, 
not a deduced type.
% 
\added{If a placeholder is designated by a 
\grammarterm{constrained-type-specifier}, redeclarations or specializations 
shall use the same \grammarterm{constrained-type-specifier}.}
\end{quote}

Add the following examples to that paragraph.

\begin{quote}
\begin{addedblock}
\begin{codeblock}
template<typename T> concept bool C1 = true;
template<typename T> concept bool C2 = true;

template<typename T> auto cf(T) -> C1; // \#1
template<typename T> C1 cf(T);         // \#2, redeclaration of \#1
template<typename T> C2 cf(T);         // error: redeclared with different placeholder
\end{codeblock}
\end{addedblock}
\end{quote}

Modify paragraph 14 (here, 8).

\begin{quote}
\pnum
A function declared with a return type that uses 
\removed{a placeholder type}
\added{placeholders}
shall not be virtual (10.3)
\end{quote}


Modify paragraph 15 (paragraph 8, here) to read:

\begin{quote}
\setcounter{Paras}{7}
\pnum
An explicit instantiation declaration (\ref{temp.explicit}) does not cause the 
instantiation of an entity declared using 
\removed{a placeholder type}
\added{placeholders},
but it also does not prevent that entity from being instantiated as needed to
determine its type. 
\end{quote}

%%
%% Constrained type specifiers
%%
\rSec4[dcl.spec.auto.constr]{Constrained type specifiers}

Add this section to \ref{dcl.spec.auto}.

\begin{quote}
\pnum
A \grammarterm{constrained-type-specifier} designates a placeholder
(type, non-type, or template) and introduces an associated constraint
(\ref{temp.constr.decl}).

\begin{bnf}
\nontermdef{constrained-type-specifier}\br
	qualified-concept-name

\nontermdef{qualified-concept-name}\br
	nested-name-specifier\opt constrained-type-name

\nontermdef{constrained-type-name}\br
  concept-name\br
  partial-concept-id

\nontermdef{concept-name}\br
  identifier

\nontermdef{partial-concept-id}\br
		concept-name \terminal{<} template-argument-list\opt \terminal{>}
\end{bnf}

\enterexample
\begin{codeblock}
template<typename T> concept bool C1 = false;
template<int N> concept bool C2 = false;
template<template<typename> class X> concept bool C3 = false;

template<typename T, int N> class Array { };
template<typename T, template<typename> class A> class Stack { };
template<typename T> class Alloc { };

void f1(C1);              // \tcode{C1} designates a placeholder type
void f2(Array<auto, C2>); // \tcode{C2} designates a placeholder for an integer value
void f3(Stack<auto, C3>); // \tcode{C3} designates a placeholder for a class template
\end{codeblock}
\exitexample

\pnum
An identifier is a \grammarterm{concept-name} if it refers to a set of 
concept definitions (\ref{dcl.spec.concept}).
%
\enternote
The set of concepts has multiple members only when referring to a set of 
overloaded function concepts. There is at most one member of this set when a
\grammarterm{concept-name} refers to a variable concept.
\exitnote
%
\enterexample
\begin{codeblock}
template<typename T> concept bool C() { return true; }             // \#1
template<typename T, typename U> concept bool C() { return true; } // \#2
template<typename T> concept bool D = true;                        // \#3

void f(C); // OK: the set of concepts referred to by C includes both \#1 and \#2;
           // concept resolution (\ref{temp.constr.resolve}) selects \#1.
void g(D); // OK: the concept-name \tcode{D} refers only to \#3
\end{codeblock}
\exitexample

\pnum
A \grammarterm{partial-concept-id} is a \grammarterm{concept-name} followed 
by a sequence of \grammarterm{template-argument}{}s.
%
\enterexample
\begin{codeblock}
template<typename T, int N = 0> concept bool Seq = true;

void f1(Seq<3>); // OK
void f2(Seq<>);  // OK
\end{codeblock}
\exitexample

\pnum
The concept designated by a \grammarterm{constrained-type-specifier}
is the one selected according to the rules for concept resolution in
\ref{temp.constr.resolve}.

\pnum
\enternote
The constraint introduced by a \grammarterm{constrained-type-name} is
introduced by the invention of a template parameter. 
% 
The rules for inventing template parameters corresponding to placeholders
in the \grammarterm{parameter-declaration-clause} of a
\grammarterm{lambda-expression} (\ref{expr.prim.lambda})
or function declaration (\ref{dcl.fct}) are described in
\ref{dcl.fct}.
% 
The rules for inventing a template parameter corresponding to placeholders
in the \grammarterm{trailing-return-type} of a 
\grammarterm{compound-requirement} are described in 
\ref{temp.constr.deduct}.
\exitnote
\end{quote}


%%
%% concept specifier
%%
\rSec2[dcl.spec.concept]{\tcode{concept} specifier}

Add this section to \ref{dcl.spec}.

\begin{quote}

\pnum
The \tcode{concept} specifier shall be applied only to the 
definition of a function template or variable template, declared
in namespace scope (\cxxref{basic.scope.namespace}).
%
A function template definition having the \tcode{concept}
specifier is called a \defn{function concept}. 
% 
A function concept shall have no \grammarterm{exception-specification} 
and is treated as if it were specified with \tcode{noexcept(true)} 
(\cxxref{except.spec}).
%
When a function is declared to be a concept, it shall be the only
declaration of that function.
%
A variable template definition having the \tcode{concept} 
specifier is called a \defn{variable concept}.
%
A \defn{concept definition} refers to either a function concept 
and its definition or a variable concept and its initializer.
%
\enterexample
\begin{codeblock}
template<typename T> 
  concept bool F1() { return true; } // OK: declares a function concept
template<typename T> 
  concept bool F2();                 // error: function concept is not a definition
template<typename T> 
  constexpr bool F3();
template<typename T>
  concept bool F3() { return true; } // error: redeclaration of a function as a concept
template<typename T> 
  concept bool V1 = true;            // OK: declares a variable concept
template<typename T> 
  concept bool V2;                   // error: variable concept with no initializer
struct S {
  template<typename T> 
    static concept bool C = true;    // error: concept declared in class scope
};
\end{codeblock}
\exitexample

\pnum
Every concept definition is implicitly defined to be a 
\tcode{constexpr} declaration (\cxxref{dcl.constexpr}).
% 
A concept definition shall not be declared with the 
\tcode{thread_local}, \tcode{inline}, \tcode{friend}, or 
\tcode{constexpr} specifiers, nor shall a concept definition have associated 
constraints (\ref{temp.constr.decl}).

\pnum
The definition of a function concept or the initializer of
a variable concept shall not include a reference to the concept being
declared.
%
\enterexample
\begin{codeblock}
template<typename T>
  concept bool F() { return F<typename T::type>(); } // error
template<typename T>
  concept bool V = V<T*>;                            // error
\end{codeblock}
\exitexample

\pnum
The first declared template parameter of a concept definition is its
\defn{prototype parameter}. 
%
A \defn{variadic concept} is a concept whose prototype parameter
is a template parameter pack.

\pnum
A function concept has the following restrictions:
\begin{itemize}
\item No \grammarterm{function-specifier}{}s shall
     appear in its declaration (\cxxref{dcl.fct.spec}).

\item The declared return type shall have the type \tcode{bool}.

\item The declaration's parameter list shall be equivalent to an empty 
      parameter list.

\item The declaration shall have a \grammarterm{function-body} equivalent
to \tcode{\{ return E; \}} where \tcode{E} is a 
\grammarterm{constraint-expression} (\ref{temp.constr.expr}).
\end{itemize}
%
\enternote
Return type deduction requires the instantiation of the function 
definition, but concept definitions are not instantiated; they
are normalized (\ref{temp.constr.decl}).
\exitnote
%
\enterexample
\begin{codeblock}
template<typename T> 
  concept int F1() { return 0; }      // error: return type is not bool
template<typename T> 
  concept auto F2() { return true; }  // error: return type is deduced
template<typename T> 
  concept bool F3(T) { return true; } // error: not an empty parameter list
\end{codeblock}
\exitexample

\pnum
A variable concept has the following restrictions:
\begin{itemize}
\item The declared type shall have the type \tcode{bool}.
\item The declaration shall have an initializer.
\item The initializer shall be a \grammarterm{constraint-expression}.
\end{itemize}
%
\enterexample
\begin{codeblock}
template<typename T> 
  concept bool V1 = 3 + 4; // error: initializer is not a constraint-expression
concept bool V2 = 0;       // error: not a template

template<typename T> concept bool C = true;

template<C T> 
  concept bool V3 = true;  // error: constrained template declared as a concept
\end{codeblock}
\exitexample

\pnum
A program shall not declare an explicit instantiation (\ref{temp.explicit}), 
an explicit specialization (\ref{temp.expl.spec}), or a partial specialization
of a concept definition.
%
\enternote
This prevents users from subverting the constraint system by providing a 
meaning for a concept that differs from its original definition.
\exitnote

\end{quote}
