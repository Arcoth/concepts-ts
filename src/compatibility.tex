%!TEX root = std.tex
\infannex{diff}{Compatibility}

\rSec1[diff.iso]{\Cpp extensions for Concepts and ISO \Cpp 2014}

\pnum
This subclause lists the differences between \Cpp with Concepts
and ISO \Cpp, by the chapters of this document.

\rSec2[diff.lex]{Clause~\ref{lex}: lexical conventions}

\ref{lex.key}
\change New Keywords\\
New keywords are added to \Cpp extensions for Concepts;
see \ref{lex.key}.
\rationale
These keywords were added in order to implement the semantics of the
new features. In particular, the \tcode{requires} keyword is added
to introduce constraints through a \grammarterm{requires-clause} or
a \grammarterm{requires-expression}. The \tcode{concept} keyword is
added to enable the definition of concepts (\ref{dcl.spec.concept}),
the normalization of constraints (\ref{temp.constr.decl}), and the 
semantic differentiation of \grammarterm{concept-name}{s} from other
\grammarterm{identifier}{s}.
\effect
Change to semantics of well-defined feature.
Any ISO \Cpp programs that used any of these keywords as identifiers
are not valid \Cpp programs with Concepts.
\difficulty
Syntactic transformation.
Converting one specific program is easy.
Converting a large collection
of related programs takes more work.
\howwide
Seldom.

