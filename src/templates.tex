
%%
%% Templates
%%
\setcounter{chapter}{13}
\rSec0[temp]{Templates}

Modify the \grammarterm{template-declaration} grammar in paragraph 1 to 
allow a template declaration introduced by a concept.

\begin{quote}
\pnum

\begin{bnf}
\nontermdef{template-declaration}\br
  \terminal{template} \terminal{<} template-parameter-list \terminal{>}
    \added{requires-clause\opt} declaration\br
  \added{template-introduction declaration}

\begin{addedblock}
\nontermdef{requires-clause}\br
  \terminal{requires} constraint-expression
\end{addedblock}
\end{bnf}

\end{quote}
  
Add the following paragraphs after paragraph 6.

\begin{quote}
\setcounter{Paras}{6}
\pnum
A \grammarterm{template-declaration} is written in terms of its template 
parameters. These parameters are declared explicitly in a 
\grammarterm{template-parameter-list} (\ref{temp.param}), or they are
introduced by a \grammarterm{template-introduction} (\ref{temp.intro}).
%
The optional \grammarterm{requires-clause} following a
\grammarterm{template-parameter-list} allows the specification of
constraints (\ref{temp.constr.decl}) on template arguments (\ref{temp.arg}).
\end{quote}

%%
%% Template parameters
%%
\rSec1[temp.param]{Template parameters}

In paragraph 1, extend the grammar for template parameters to 
constrained template parameters.

\begin{quote}
\pnum

\begin{bnf}
\nontermdef{template-parameter}\br
  \added{constrained-parameter}

\begin{addedblock}
\nontermdef{constrained-parameter}\br
  nested-name-specifier\opt constrained-type-name ...\opt identifier\opt default-template-argument\opt\br

\nontermdef{default-template-argument}\br
  \terminal{=} type-id\br
  \terminal{=} id-expression\br
  \terminal{=} initializer-clause\br
\end{addedblock}
\end{bnf}
\end{quote}

Insert a new paragraph after paragraph 1.

\begin{quote}
\pnum
There is an ambiguity in the syntax of a template parameter between the
declaration of a \grammarterm{constrained-parameter} and a
\grammarterm{parameter-declaration}.
% 
If the \grammarterm{type-specifier-seq} of a \grammarterm{parameter-declaration} 
is a \grammarterm{constrained-type-specifier} (\ref{dcl.spec.auto.constr}), 
then the \grammarterm{template-parameter} is a \grammarterm{constrained-parameter}.
\end{quote}

Insert the following paragraphs after paragraph 8. These paragraphs
define the meaning of a constrained template parameter.

\begin{quote}
\setcounter{Paras}{8}
\pnum
A \grammarterm{constrained-parameter} declares a template parameter whose 
kind (type, non-type, template) and type match that of the prototype parameter 
of the concept designated by the \grammarterm{concept-name} of the
\grammarterm{constrained-type-specifier}.
% 
The designated concept is selected by the rules for concept resolution 
described in \ref{temp.constr.resolve}.
% 
Let \tcode{X} be the prototype parameter of the designated concept.
\enternote
The prototype parameter is matched by the wildcard in the matching argument 
list for the \grammarterm{constrained-type-name} of 
the \grammarterm{constrained-parameter}.
\exitnote
% 
The declared template parameter is determined by the kind of \tcode{X} 
(type, non-type, template) and the optional ellipsis in the
\grammarterm{constrained-parameter} as follows.
% 
\begin{itemize}
\item If \tcode{X} is a type \grammarterm{template-parameter} the declared
parameter is a type \grammarterm{template-parameter}. 

\item If \tcode{X} is a non-type \grammarterm{template-parameter}, the declared
parameter is a non-type \grammarterm{template-parameter} having the same 
type as \tcode{X}.

\item If \tcode{X} is a template \grammarterm{template-parameter}, the declared
parameter is a template \grammarterm{template-parameter} having the same 
\grammarterm{template-parameter-list} as \tcode{X}, excluding default template 
arguments.
\end{itemize}
% 
\enterexample
\begin{codeblock}
template<typename T> concept bool C1 = true;
template<template<typename> class X> concept bool C2 = true;
template<int N> concept bool C3 = true;
template<typename... Ts> concept bool C4 = true;
template<char... Cs> concept bool C5 = true;

template<C1 T> void f1();     // OK: \tcode{T} is a type template-parameter
template<C2 X> void f2();     // OK: \tcode{X} is a template with one type-parameter
template<C3 N> void f3();     // OK: \tcode{N} has type int
template<C4... Ts> void f4(); // OK: \tcode{Ts} is a template parameter pack of types
template<C4 T> void f5();     // OK: \tcode{T} is a type template-parameter
template<C5... Cs> f6();      // OK: \tcode{Cs} is a template parameter pack of \tcode{char}{}s
\end{codeblock}
\exitexample

\pnum
A \grammarterm{constrained-parameter} associates a predicate
constraint with its \grammarterm{template-declaration}. The constraint
is derived from the \grammarterm{constrained-type-name} \tcode{CC} in the
\grammarterm{constrained-parameter}, its designated concept \tcode{C},
and the declared template parameter \tcode{P}.
% 
\begin{itemize}
\item First, form a template argument \tcode{A} from \tcode{P}. If \tcode{P} 
declares a template parameter pack (\ref{temp.variadic}),
and \tcode{C} is a variadic concept (\ref{dcl.spec.concept}), then \tcode{A} is 
the pack expansion \tcode{P...}. Otherwise, \tcode{A} is the 
\grammarterm{id-expression} \tcode{P}.

\item Then, form a \grammarterm{template-id} \tcode{TT} based on the 
\grammarterm{constrained-type-name} \tcode{CC} of the 
\grammarterm{constrained-parameter}. If \tcode{CC} is 
a \grammarterm{concept-name}, then \tcode{TT} is \tcode{C<A>}. Otherwise, 
\tcode{CC} is a \grammarterm{partial-concept-id} of the form
\tcode{C<A1, A2, ..., A$N$>}, and \tcode{TT} is \tcode{C<A, A1, A2, ..., A$N$}>.

\item Then, form an \grammarterm{expression} \tcode{E} as follows. 
If \tcode{C} is variable concept (\ref{dcl.spec.concept}), then \tcode{E} is the 
\grammarterm{id-expression} \tcode{TT}. Otherwise, \tcode{C} is a function 
concept and \tcode{E} is the function call \tcode{TT()}.

\item Finally, If \tcode{P} declares a template parameter pack and 
\tcode{C} is not a variadic concept, \tcode{E} is adjusted to be the
\grammarterm{fold-expression} \tcode{E \&\& ...} (\ref{expr.prim.fold}).
\end{itemize}
% 
\tcode{E} is the expression of the associated predicate constraint.
% 
\enterexample
\begin{codeblock}
template<typename T> concept bool C1 = true;
template<typename... Ts> concept bool C2() { return true; }
template<typename T, typename U> concept bool C3 = true;

template<C1 T> struct s1;      // associates \tcode{C1<T>}
template<C1... T> struct s2;   // associates \tcode{C1<T>...}
template<C2... T> struct s3;   // associates \tcode{C2<T...>()}
template<C3<int> T> struct s4; // associates \tcode{C3<T, int>}
\end{codeblock}
\exitexample

\end{quote}

Insert the following paragraph after paragraph 9 to require that the
kind of a \grammarterm{default-argument} matches the kind of its
\grammarterm{constrained-parameter}.

\begin{quote}
\setcounter{Paras}{11}
\pnum
The default \grammarterm{template-argument} of
a \grammarterm{constrained-parameter} shall match
the kind (type, non-type, template) of the declared parameter.
% 
\enterexample
\begin{codeblock}
template<typename T> concept bool C1 = true;
template<int N> concept bool C2 = true;
template<template<typename> class X> concept bool C3 = true;

template<typename T> struct S0;

template<C1 T = int> struct S1; // OK
template<C2 N = 0> struct S2;   // OK
template<C3 X = S0> struct S3;  // OK
template<C1 T = 0> struct S4;   // error: default argument is not a type
\end{codeblock}
\exitexample
\end{quote}


%%
%% Introduction of template parameters
%%
\rSec1[temp.intro]{Introduction of template parameters}

Add this section after \ref{temp.param}.

\begin{quote}

\pnum
A \grammarterm{template-introduction} provides a concise way of declaring
templates.

\begin{bnf}
\nontermdef{template-introduction}\br
  nested-name-specifier\opt concept-name \terminal{\{} introduction-list \terminal{\}}

\nontermdef{introduction-list}\br
  introduced-parameter\br
  introduction-list \terminal{,} introduced-parameter

\nontermdef{introduced-parameter}\br
    \terminal{...}\opt identifier
\end{bnf}

A \grammarterm{template-introduction} declares a sequence of 
\grammarterm{template-parameter}{s}, which are derived from a 
\grammarterm{concept-name} and the sequence of \grammarterm{introduced-parameter}{s} 
in its \grammarterm{introduction-list}.

\pnum
The concept designated by the \grammarterm{concept-name} is selected by the
concept resolution rules described in \ref{temp.constr.resolve}. Let
\tcode{C} be the designated concept.

\pnum
The template parameters declared by a \grammarterm{template-introduction}
are derived from its \grammarterm{introduced-parameter}{s} and the
template parameter declarations in \tcode{C} to which those
\grammarterm{introduced-parameter}{s} are matched as wildcards according to 
the rules in \ref{temp.constr.resolve}.
% 
For each \grammarterm{introduced-parameter} \tcode{I}, declare a template
parameter using the following rules:
\begin{itemize}
\item Let \tcode{P} be the template parameter declaration in \tcode{C} 
      corresponding to \tcode{I}. If \tcode{P} does not declare a template 
      parameter pack (\ref{temp.variadic}), \tcode{I} shall not include 
      an ellipsis.

\item If \tcode{P} declares a template parameter pack, adjust \tcode{P} 
      to be the pattern of that pack.

\item Declare a template parameter according to the rules for declaring a
      \grammarterm{constrained-parameter} in \ref{temp.param}, using
      \tcode{P} as the prototype parameter.

\item If \tcode{I} includes an ellipsis, then the declared template parameter 
      is a template parameter pack.
\end{itemize}
% 
\enterexample
\begin{codeblock}
template<typename T, int N, typename... Xs> concept bool C1 = true;
template<template<typename> class X> concept bool C2 = true;
template<typename... Ts> concept bool C3 = true;

C1{A, B, ...C} // OK: \tcode{A} is declared as \tcode{typename A},
  struct S1;   // \tcode{B} is declared as \tcode{int B}, and
               // \tcode{C} is declared as \tcode{typename ... C}

C2{T} void f();    // OK: \tcode{T} is declared as \tcode{template<typename> class T}
C2{...Ts} void g(); // error: the template parameter corresponding to \tcode{Ts}
                    // is not a template parameter pack

C3{T} struct S2;     // OK: \tcode{T} is declared as \tcode{typename T}
C3{...Ts} struct S2; // OK: \tcode{T} is declared as \tcode{typename ... T}
\end{codeblock}
\exitexample


\pnum
A concept referred to by a \grammarterm{concept-name} may have template 
parameters with default template arguments. An \grammarterm{introduction-list} 
may omit \grammarterm{identifier}{}s for a corresponding template
parameter if it has a default argument. 
% 
Only the \grammarterm{introduced-parameter}{}s are declared as template 
parameters.
% 
\enterexample
\begin{codeblock}
template<typename A, typename B = bool> concept bool C() { return true; }

C{T} void f(T); // OK: \tcode{f(T)} is a function template with
                // a single template type parameter \tcode{T}
\end{codeblock}
\exitexample

\pnum
An introduced template parameter does not have a default template argument 
even if its corresponding template parameter does.
% 
\enterexample
\begin{codeblock}
template<typename T, int N = -1> concept bool P() { return true; }

P{T, N} struct Array { };

Array<double, 0> s1; // OK
Array<double> s2;    // error: \tcode{Array} takes two template arguments
\end{codeblock}
\exitexample

%  TODO: Unify with temp.constr.form

\pnum
A \grammarterm{template-introduction} associates a predicate constraint with 
its \grammarterm{template-declaration}. This constraint is derived from
the \grammarterm{concept-name} \tcode{C} in the 
\grammarterm{template-introduction} and the sequence of 
\grammarterm{introduced-parameter}{s}.

\begin{itemize}
\item First, form a sequence of template arguments \tcode{A1, A2, ..., A$N$} 
corresponding to the \grammarterm{introduced-parameter}{s}
\tcode{P1, P2, ..., P$N$}.
% 
For each \grammarterm{introduced-parameter} \tcode{P}, form a corresponding 
template argument \tcode{A} as follows. If \tcode{P} includes an ellipsis,
then \tcode{A} is the pack expansion \tcode{P...} (\ref{temp.variadic}). 
Otherwise, \tcode{A} is the \grammarterm{id-expression} \tcode{P}.

\item Then, form an expression \tcode{E} as follows. If \tcode{C} is
a variable concept (\ref{dcl.spec.concept}), then \tcode{E} is the
\grammarterm{id-expression}
\tcode{C<A1, ..., A$N$>}. Otherwise, \tcode{C} is a function concept and
\tcode{E} is the function call \tcode{C<A1, ..., A$N$>}.
\end{itemize}
% 
\tcode{E} is the expression of the associated predicate constraint.
% 
\enterexample
\begin{codeblock}
template<typename T, typename U> concept bool C1 = true;
template<typename T, typename U> concept bool C2() { return true; }
template<typename... Ts> concept bool C3 = true;

C1{A, B} struct s1;    // associates \tcode{C1<A, B>}
C2{A, B} struct s2;    // associates \tcode{C2<A, B>()}
C3{...Ts} struct s3;   // associates \tcode{C3<Ts...>}
C3{X, ...Y} struct s4; // associates \tcode{C3<X, Y...>}
\end{codeblock}
\exitexample

\pnum
A template declared by a \grammarterm{template-introduction} can also be 
an abbreviated function template (\ref{dcl.fct}). 
% 
The invented template parameters introduced by the placeholders in the 
abbreviated function template are appended to the list of template parameters 
declared by the \grammarterm{template-introduction}.
% 
\enterexample
\begin{codeblock}
template<typename T> concept bool C1 = true;

C1{T} void f(T, auto);
template<C1 T, typename U> void f(T, U); // OK: redeclaration of \tcode{f(T, auto)}
\end{codeblock}
% 
\exitexample

\end{quote}

%%
%% Names of template specializations
%%
\rSec1[temp.names]{Names of template specializations}

Add this paragraph at the end of the section to require the satisfaction of 
associated constraints on the formation of the \grammarterm{simple-template-id}.

\begin{quote}
\setcounter{Paras}{7}
\pnum
When the \grammarterm{template-name} of a \grammarterm{simple-template-id} names
a constrained non-function template or a template template parameter,
but not a member template that is a member of an unknown specialization
\ref{temp.res}, and all \grammarterm{template-argument}{}s in the
\grammarterm{simple-template-id} are non-dependent \cxxref{temp.dep.temp}, the 
template arguments are substituted into the associated constraints
(\ref{temp.constr.decl}). 
% 
If, as a result of substitution, the associated constraints are not 
satisfied (\ref{temp.constr}), the \grammarterm{simple-template-id} is ill-formed.
% 
\enterexample
\begin{codeblock}
template<typename T> concept bool C = false;

template<C T> struct S1 { };
template<C T> using Ptr = T*;

S1<int>* p; // error: constraints not satisfied
Ptr<int> p; // error: constraints not satisfied

template<typename T>
  struct S2 { Ptr<int> x; }; // error: constraints not satisfied

template<typename T>
  struct S3 { Ptr<T> x; };   // OK: satisfaction is not required

S3<int> x;                   // error: constraints not satisfied

template<template<C T> class X>
  struct S4 {
    X<int> x; // error: constraints not satisfied (\#1)
  };

template<typename T>
  struct S5 {
    using Type = typename T::template MT<char>; // \#2
  };
\end{codeblock}
In \#1, the error is caused by the substitution of \tcode{int} into 
the associated constraints of the template parameter \tcode{X}.
% 
In \#2, there are no constraints can be checked for 
\tcode{typename Y::template MT<char>} because \tcode{MT} is a member
of an unknown specialization.
\exitexample
\end{quote}


%%
%% Template arguments
%%
\rSec1[temp.arg]{Template arguments}

%%
%% Template template arguments
%%
\rSec2[temp.arg.template]{Template template arguments}

Modify paragraph 3 to include rules for matching constrained template 
\grammarterm{template parameter}{}s. Note that the examples following this 
paragraph in the \Cpp Standard are omitted.

\begin{quote}
\setcounter{Paras}{2}
\pnum
A \grammarterm{template-argument} matches  a template 
\grammarterm{template-parameter}  (call it \tcode{P}) when each of the 
template parameters in the \grammarterm{template-parameter-list} of the 
\grammarterm{template-argument}'s corresponding class template or alias 
template (call it  \tcode{A}) matches the corresponding template parameter in 
the \grammarterm{template-parameter-list} of \tcode{P}\added{, and \tcode{P} is 
at least as constrained as \tcode{A} according to the rules in 
\ref{temp.constr.order}}.
% 
Two template parameters match if they are of the same kind (type, non-type, 
template), for non-type \grammarterm{template-parameter}{}s, their types are 
equivalent (\ref{temp.over.link}), and for template 
\grammarterm{template-parameter}{}s, each of their corresponding
\grammarterm{template-parameter}{}s matches, recursively. 
% 
When \tcode{P}'s \grammarterm{template-parameter-list} 
contains a template parameter pack (\ref{temp.variadic}), the template 
parameter pack will match zero or more template parameters or template 
parameter packs in the \grammarterm{template-parameter-list} of
\tcode{A} with the same kind (type, non-type, template) and type as the template 
parameter pack in \tcode{P} (ignoring whether those template parameters are
template parameter packs).
\end{quote}

Add the following example to the end of paragraph 3, after the
examples given in the \Cpp Standard.

\begin{quote}
\begin{addedblock}
\enterexample
\begin{codeblock}
template<typename T> concept bool C = requires (T t) { t.f(); };
template<typename T> concept bool D = C<T> && requires (T t) { t.g(); };

template<template<C> class P>
  struct S { };

template<C> struct X { };
template<D> struct Y { };
template<typename T> struct Z { };

S<X> s1; // OK: \tcode{X} and \tcode{P} have equivalent constraints
S<Y> s2; // error: \tcode{P} is not at least as constrained \tcode{Y} (\tcode{Y} is more constrained than \tcode{P})
S<Z> s3; // OK: \tcode{P} is at least as constrained as \tcode{Z}
\end{codeblock}
\exitexample
\end{addedblock}
\end{quote}


%%
%% Template declarations
%%
\setcounter{section}{5}
\rSec1[temp.decls]{Template declarations}

Modify paragraph 2 to indicate that associated constraints are
instantiated separately from the template they are associated with.

\begin{quote}
\setcounter{Paras}{1}
For purposes of name lookup and instantiation, default 
arguments\added{, associated constraints (\ref{temp.constr.decl}),} and
\grammarterm{exception-specification}{}s of function templates and default
arguments\added{, associated constraints,} and 
\grammarterm{exception-specification}{}s of member functions of class
templates are considered definitions; each default 
argument\added{, associated constraint,} or 
\grammarterm{exception-specification} is a separate definition which is 
unrelated to the function template definition or to any other default 
arguments\added{, associated constraints,} or 
\grammarterm{exception-specification}{}s.
\end{quote}


%%
%% Class templates
%%
\rSec2[temp.class]{Class templates}

Modify paragraph 3 to require template constraints for out-of-class
definitions of members of constrained templates. 

\begin{quote}
\setcounter{Paras}{2}
\pnum
When a member function, a member class, a member enumeration, a static 
data member or a member template of a class template is defined outside 
of the class template definition, the member definition is defined as a 
template definition in which the \grammarterm{template-parameter}{s}
\added{and associated constraints} (\ref{temp.constr.decl}) are those of 
the class template.
% 
The names of the template parameters used in the definition of the 
member may be different from the template parameter names used in the 
class template definition. The template argument list following the class
template name in the member definition shall name the parameters in the 
same order as the one used in the template parameter list of the member. 
% 
Each template parameter pack shall be expanded with an ellipsis in the 
template argument list.
\end{quote}

Add the following example at the end of paragraph 3.

\begin{quote}
\begin{addedblock}
\enterexample
\begin{codeblock}
template<typename T> concept bool C = true;
template<typename T> concept bool D = true;

template<C T> struct S {
    void f();
    void g();
    template<D U> struct Inner;
  }

template<typename T> requires C<T> void S<T>::f() { } // OK: parameters and constraints match
template<typename T> void S<T>::g() { } // error: no matching declaration for \tcode{S<T>}

template<C T> D{U} struct S<T>::Inner { }; // OK
\end{codeblock}
\exitexample
\end{addedblock}
\end{quote}


%%
%% Member functions of class templates
%%
\rSec3[temp.mem.func]{Member functions of class templates}

Add the following example to the end of paragraph 1.

\begin{quote}
\begin{addedblock}
\enterexample
\begin{codeblock}
template<typename T> struct S {
  void f() requires true;
  void g() requires true;
};

template<typename T> 
  void S<T>::f() requires true { } // OK
template<typename T> 
  void S<T>::g() { }               // error: no matching function in \tcode{S<T>}
\end{codeblock}
\exitexample
\end{addedblock}
\end{quote}


%%
%% Member templates
%%
\rSec2[temp.mem]{Member templates}


Modify paragraph 1 in order to account for constrained member templates
of (possibly) constrained class templates. 

\begin{quote}
\pnum
A template can be declared within a class or class template; such a 
template is called a member template. 
% 
A member template can be defined within or outside its class definition 
or class template definition. 
% 
A member template of a class template that is defined outside of its 
class template definition shall be specified with the 
\grammarterm{template-parameter}{s} \added{and associated constraints}
(\ref{temp.constr.decl}) of the class template followed by the 
\grammarterm{template-parameter}{s}
\added{and associated constraints} of the member template.
\end{quote}


Add the following example at the end of paragraph 1.

\begin{quote}
\begin{addedblock}
\enterexample
\begin{codeblock}
template<typename T> concept bool C1 = true;
template<typename T> concept bool C2 = sizeof(T) <= 4;

template<C1 T>
  struct S {
    template<C2 U> void f(U);
    template<C2 U> void g(U);
  };

template<C1 T> template<typename U> 
  void S<T>::f(U) requires C2<U> { } // OK
template<C1 T> template<typename U> 
  void S<T>::g(U) { }                // error: no matching function in \tcode{S<T>}
\end{codeblock}
\exitexample
\end{addedblock}
\end{quote}


%%
%% Variadic templates
%%
\rSec2[temp.variadic]{Variadic templates}

Add \grammarterm{fold-expression}{s} to the list of contexts in which
pack expansion can occur. 

\begin{quote}
\begin{itemize}
\item ...

\item \added{In a \grammarterm{fold-expression} (\ref{expr.prim.fold});
the pattern is the \grammarterm{cast-expression}
that contains an unexpanded parameter pack.}
\end{itemize}
\end{quote}

Modify paragraph 7 to exclude \grammarterm{fold-expression}{s} from
producing a comma-separated list of elements.

\begin{quote}
The instantiation of a pack expansion
\removed{that is not a \tcode{sizeof...} expression}
\added{that is neither a \tcode{sizeof...} expression nor a \grammarterm{fold-expression}}
produces a list
$\mathtt{E}_1, \mathtt{E}_2, ..., \mathtt{E}_N$,
where $N$ is the number of 
elements in the pack expansion parameters. 
% 
Each \tcode{E$_i$} is generated by instantiating 
the pattern and replacing each pack expansion 
parameter with its $i$th element.
\end{quote}


Add the following paragraphs at the end of this section.

\begin{quote}
\begin{addedblock}
\setcounter{Paras}{8}
\pnum
The instantiation of a \grammarterm{fold-expression} produces:

\begin{itemize}
\item
\tcode{((}$\mathtt{E}_1$
           \placeholder{op} $\mathtt{E}_2$\tcode{)}
           \placeholder{op} $\cdots$\tcode{)}
           \placeholder{op} $\mathtt{E}_N$
for a unary left fold,
\item
         $\mathtt{E}_1$     \placeholder{op}
\tcode{(}$\cdots$           \placeholder{op}
\tcode{(}$\mathtt{E}_{N-1}$ \placeholder{op}
         $\mathtt{E}_N$\tcode{))}
for a unary right fold,
\item
\tcode{(((}$\mathtt{E}$
            \placeholder{op} $\mathtt{E}_1$\tcode{)}
            \placeholder{op} $\mathtt{E}_2$\tcode{)}
            \placeholder{op} $\cdots$\tcode{)}
            \placeholder{op} $\mathtt{E}_N$
for a binary left fold, and
\item
         $\mathtt{E}_1$     \placeholder{op}
\tcode{(}$\cdots$           \placeholder{op}
\tcode{(}$\mathtt{E}_{N-1}$ \placeholder{op}
\tcode{(}$\mathtt{E}_{N}$   \placeholder{op}
         $\mathtt{E}$\tcode{)))}
for a binary right fold.
\end{itemize}

In each case,
\placeholder{op} is the \grammarterm{fold-operator},
$N$ is the number of elements in the pack expansion parameters,
and each $\mathtt{E}_i$ is generated by instantiating the pattern
and replacing each pack expansion parameter with its $i$th element.
For a binary fold-expression,
$\mathtt{E}$ is generated
by instantiating the \grammarterm{cast-expression}
that did not contain an unexpanded parameter pack.
\enterexample
\begin{codeblock}
template<typename ...Args>
  bool all(Args ...args) { return (... && args); }

bool b = all(true, true, true, false);
\end{codeblock}
Within the instantiation of \tcode{all},
the returned expression expands to
\tcode{((true \&\& true) \&\& true) \&\& false},
which evalutes to \tcode{false}.
\exitexample
If $N$ is zero for a unary fold-expression,
the value of the expression is shown in Table~\ref{tab:fold.empty};
if the operator is not listed in Table~\ref{tab:fold.empty},
the instantiation is ill-formed.

\begin{floattable}{Value of folding empty sequences}{tab:fold.empty}
{ll}
\topline
\lhdr{Operator} & \rhdr{Value when parameter pack is empty} \\
\capsep
\tcode{*}       & \tcode{1}      \\
\tcode{+}       & \tcode{int()}  \\
\tcode{\&}      & \tcode{-1}     \\
\tcode{|}       & \tcode{int()}  \\
\tcode{\&\&}    & \tcode{true}   \\
\tcode{||}      & \tcode{false}  \\
\tcode{,}       & \tcode{void()} \\
\end{floattable}
\end{addedblock}
\end{quote}


%%
%% Friends
%%
\rSec2[temp.friend]{Friends}

Modify paragraph 9 to restrict constrained friend declarations.

\begin{quote}
\setcounter{Paras}{8}
\pnum
When a friend declaration refers to a specialization of a function
template, the function parameter declarations shall not include
default arguments, \added{the declaration shall not have associated constraints
(\ref{temp.constr.decl}),} nor shall the inline specifier be used in such a
declaration.
\end{quote}

Add examples following that paragraph.

\begin{quote}
\begin{addedblock}
\pnum
\enternote
Other friend declarations can be constrained. In a constrained friend 
declaration that is not a definition, the constraints are used for declaration
\exitnote
\enterexample
\begin{codeblock}
template<typename T> concept bool C1 = true;
template<typename T> concept bool C2 = false;

template<C1 T> g0(T);
template<C1 T> g1(T);
template<C2 T> g2(T);

template<typename T>
  struct S {
    friend void f1() requires true;      // OK
    friend void f2() requires C1<T>;     // OK
    friend void g0<T>(T) requires C1<T<; // error: constrained friend specialization
    friend void g1<T>(T);                // OK
    friend void g2<T>(T);                // error: constraint can never be satisfied, no diagnostic required
  };

void f1() requires true;    // friend of all S<T>
void f2() requires C1<int>; // friend of S<int>
\end{codeblock}
The friend declaration of \tcode{g2} is ill-formed, no
diagnostic required, because no valid specialization of \tcode{S}
can be generated: the constraint on \tcode{g2} can never
be satisfied, so template argument deduction
(\ref{temp.deduct.decl}) will always fail.
\exitexample

\pnum
\enternote
Within a class template, a friend may define a non-template function
whose constraints specify requirements on template arguments.
\enterexample
\begin{codeblock}
template<typename T> concept bool Eq = requires (T t) { t == t; };

template<typename T>
  struct S {
    friend bool operator==(S a, S b) requires Eq<T> { return a == b; } // OK
  };
\end{codeblock}
\exitexample
In the instantiation of such a class template, the template
arguments are substituted into the constraints but not evaluated.
Constraints are checked (\ref{temp.constr}) only when
that function is considered as a viable candidate for overload resolution
(\ref{over.match.viable}).
\exitnote
\end{addedblock}
\end{quote}


%%
%% Class template partial specializations
%%
\rSec2[temp.class.spec]{Class template partial specialization}

After paragraph 3, insert the following, which explains constrained partial 
specializations.

\begin{quote}
\begin{addedblock}
\setcounter{Paras}{3}
\pnum
A class template partial specialization may be constrained
(Clause~\ref{temp}).
\enterexample
\begin{codeblock}
template<typename T> concept bool C = requires (T t) { t.f(); };
template<int I> concept bool N = I > 0;

template<C T1, C T2, N I> class A<T1, T2, I>;  // \#6
template<C T, N I>        class A<int, T*, I>; // \#7
\end{codeblock}
\exitexample
\end{addedblock}
\end{quote}

Remove the 3rd item in the list of paragraph 8 to allow constrained class 
template partial specializations like \#6, and because it is redundant with 
the 4th item. Note that all other items in that list are elided.

\begin{quote}
\setcounter{Paras}{7}
Within the argument list of a class template partial specialization, 
the following restrictions apply:
\begin{itemize}
\item ...

\item \removed{The argument list of the specialization shall
not be identical to the implicit argument list of the
primary template.}

\item The specialization shall be more specialized than the primary
template (\ref{temp.class.order}).

\item ...
\end{itemize}
\end{quote}
 
%%
%% Matching of class template partial specializations
%%
\rSec3[temp.class.spec.match]{Matching of class template partial specializations}

Modify paragraph 2; constraints must be satisfied in order
to match a partial specialization. 

\begin{quote}
\setcounter{Paras}{1}
\pnum
A partial specialization matches a given actual template argument list if 
the template arguments of the partial specialization can be deduced from the 
actual template argument list (\ref{temp.deduct}) \added{, and the deduced 
template arguments satisfy the constraints of the partial specialization, if 
any (\ref{temp.constr})}.
\end{quote}

Add the following example to the end of paragraph 2.

\begin{quote}
\begin{addedblock}
\enterexample
\begin{codeblock}
struct S { void f(); };

A<S, S, 1>    a6; // uses \#6
A<S, int, 2>  a7; // error: constraints not satisfied
A<int, S*, 3> a8; // uses \#7
\end{codeblock}
\exitexample
\end{addedblock}
\end{quote}


%%
%% Partial ordering of class template specializations
%%
\rSec3[temp.class.order]{Partial ordering of class template specializations}

Modify paragraph 1 so that constraints are considered in the
partial ordering of class template specializations. 

\begin{quote}
\pnum
For two class template partial specializations, the first is 
at least as specialized as the second if, given the following 
rewrite to two function templates, the first function template 
is at least as specialized as the second according to the ordering 
rules for function templates 
(\ref{temp.func.order}):
% 
\begin{itemize}
\item the first function template has the same template 
parameters \added{and associated constraints (\ref{temp.constr.decl})} 
as the first partial specialization, and has a single function parameter 
whose type is a class template specialization with the template
arguments of the first partial specialization, and

\item the second function template has the same template 
parameters \added{and associated constraints (\ref{temp.constr.decl})} 
as the second partial specialization, and has a single function parameter 
whose type is a class template specialization with the template
arguments of the second partial specialization.
\end{itemize}
\end{quote}

Add the following example to the end of paragraph 1.

\begin{quote}
\begin{addedblock}
\enterexample
\begin{codeblock}
template<typename T> concept bool C = requires (T t) { t.f(); };
template<typename T> concept bool D = C<T> && requires (T t) { t.f(); };


template<typename T> class S { };
template<C T> class S<T> { }; // \#1
template<D T> class S<T> { }; // \#2

template<C T> void f(S<T>); // A
template<D T> void f(S<T>); // B
\end{codeblock}
The partial specialization \#2 is more specialized than 
\#1 for template arguments that satisfy both constraints because 
\tcode{B} is more specialized than \tcode{A}.
\exitexample
\end{addedblock}
\end{quote}


%%
%% Function templates
%%
\rSec2[temp.fct]{Function templates}

%%
%% Function template overloading
%%
\rSec3[temp.over.link]{Function template overloading}

Modify paragraph 6 to account for constraints on function templates.

\begin{quote}
\setcounter{Paras}{5}
\pnum

\removed{ Two function templates are \defn{equivalent} if they are 
declared in the same scope, have the same name, have identical template 
parameter lists, and have return types and parameter lists that are 
equivalent using the rules described above to compare expressions 
involving template parameters.}
% 
\begin{addedblock}
Two function templates are \defn{equivalent} if they:
\begin{itemize}
\item are declared in the same scope,

\item have the same name,

\item have identical template parameter lists,

\item have return types and parameter lists that are equivalent using the 
rules described above to compare expressions involving template parameters, and

\item have associated constraints that are equivalent using the rules 
described in \ref{temp.constr.decl} to compare constraints.
\end{itemize}
\end{addedblock}
% 
Two function templates are \defn{functionally equivalent} if they 
are equivalent except that \removed{one or more expressions that involve 
template parameters in the return types and parameter lists are functionally 
equivalent using the rules described above to compare expressions involving 
template parameters.}
\begin{addedblock}
\begin{itemize}
\item one or more expressions that involve template parameters in the return 
      types and parameter lists are functionally equivalent using the rules 
      described above to compare expressions involving template parameters, or
\item the associated constraints are functionally equivalent using the rules 
      described in \ref{temp.constr.decl} to compare constraints.
\end{itemize}
\end{addedblock}
% 
If a program contains declarations of function templates that are 
functionally equivalent but not equivalent, the program is ill-formed; 
no diagnostic is required.
\end{quote}


%%
%% Partial ordering of function templates
%%
\rSec3[temp.func.order]{Partial ordering of function templates}

Modify paragraph 2 to include constraints in the partial ordering
of function templates.

\begin{quote}
\setcounter{Paras}{1}
\pnum
Partial ordering selects which of two function templates is 
more specialized than the other by transforming each template 
in turn (see next paragraph) and performing template argument 
deduction using the function type. The deduction process 
determines whether one of the templates is more specialized 
than the other.
% 
If so, the more specialized template is the one chosen by the 
partial ordering process. 
% 
\added{If both deductions succeed, the partial ordering selects
the more constrained template as described by the rules in
\ref{temp.constr.order}.}
\end{quote}

%%
%% Name resolution
%%
\rSec1[temp.res]{Name resolution}

Modify paragraph 8.

\begin{quote}
\setcounter{Paras}{7}
\pnum
Knowing which names are type names allows the syntax of every
template to be checked. No diagnostic shall be issued for a template
for which a valid specialization can be generated. If no valid
specialization can be generated for a template, and that template is
not instantiated, the template is ill-formed, no diagnostic
required. If every valid specialization of a variadic template
requires an empty template parameter pack, the template is
ill-formed, no diagnostic required. 
% 
% TODO: "would result in a valid expression" should probably be
% "would always result in substitution failure".
% 
\added{If no instantiation of
the associated constraints (\ref{temp.constr.decl}) of a template would result 
in a valid expression, the template is ill-formed, no diagnostic required.}
% 
If a hypothetical instantiation of a template immediately following
its definition would be ill-formed due to a construct that does not
depend on a template parameter, the program is ill-formed; no
diagnostic is required. If the interpretation of such a construct in
the hypothetical instantiation is different from the interpretation
of the corresponding construct in any actual instantiation of the
template, the program is ill-formed; no diagnostic is required.
\end{quote}

%%
%% Dependent names
%%
\setcounter{subsection}{1}
\rSec2[temp.dep]{Dependent names}

%%
%% Type-dependent expressions
%%
\setcounter{subsubsection}{1}
\rSec3[temp.dep.expr]{Type-dependent expressions}

Add the following paragraph to this section.

\begin{quote}
\setcounter{Paras}{6}
\pnum
\added{A \grammarterm{fold-expression} is type-dependent.}
\end{quote}

%%
%% Value-dependent expressions
%%
\rSec3[temp.dep.constexpr]{Value-dependent expressions}

Modify paragraph 4 to include \grammarterm{fold-expression}{s} in
the set of value-dependent expressions.

\begin{quote}
\setcounter{Paras}{3}
\pnum
Expressions of the following form are value-dependent:

\begin{ncbnftab}
\terminal{sizeof} \terminal{...} \terminal{(} identifier \terminal{)}\br
\added{fold-expression}
\end{ncbnftab}
\end{quote}


%%
%% Dependent name resolution
%% 
\setcounter{subsection}{3}
\rSec2[temp.dep.res]{Dependent name resolution}

%%
%% Point of instantiation
%%
\rSec3[temp.point]{Point of instantiation}

Add a new paragraph after paragraph 4.

\begin{quote}
\setcounter{Paras}{4}
\pnum
The point of instantiation of a \grammarterm{constraint-expression} of a
specialization immediately precedes the point of instantiation of
the specialization.
\end{quote}


%%
%% Template instantiation and specialization
%%
\rSec1[temp.spec]{Template instantiation and specialization}

%%
%% Implicit instantiation
\rSec2[temp.inst]{Implicit Instantiation}
    
Change paragraph 1 to include associated constraints.

\begin{quote}
Unless a class template specialization has been explicitly
instantiated \ref{temp.explicit} or explicitly specialized \ref{temp.expl.spec}, 
the class template specialization is implicitly instantiated when the
specialization is referenced in a context that requires a
completely-defined object type or when the completeness of the
class type affects the semantics of the program. 
% 
\enternote
Within a template declaration, a local class or enumeration and the members
of a local class are never considered to be entities that can be
separately instantiated (this includes their default arguments,
exception-specifications, and non-static data member initializers, if any). 
% 
As a result, the dependent names are looked up, the semantic constraints 
are checked, and any templates used are instantiated as part of the
instantiation of the entity within which the local class or enumeration is 
declared.
\exitnote
% 
The implicit instantiation of a class template specialization causes the 
implicit instantiation of the declarations, but not of the definitions, default
arguments, \added{associated constraints (\ref{temp.constr.decl}),} or
\grammarterm{exception-specification}{}s of the class member functions,
member classes, scoped member enumerations, static data members and
member templates; and it causes the implicit instantiation of the
definitions of unscoped member enumerations and member anonymous
unions.
\end{quote}


Add a new paragraph after paragraph 15 to describe how associated
constraints are instantiated.

\begin{quote}
\begin{addedblock}
\setcounter{Paras}{15}
\pnum
The associated constraints of a template specialization are not
instantiated along with the specialization itself; they are
instantiated only to determine if they are satisfied
(\ref{temp.constr}).
% 
\enternote
The satisfaction of constraints is determined during lookup or overload
resolution (\ref{over.match}). Preserving the spelling
of the substituted constraint also allows constrained member function
to be partially ordered by those constraints according to the rules
in \ref{temp.constr.order}.
\exitnote
% 
\enterexample
\begin{codeblock}
template<typename T> concept bool C = sizeof(T) > 2;
template<typename T> concept bool D = C<T> && sizeof(T) > 4;

template<typename T> struct S {
  S() requires C<T> { } // \#1
  S() requires D<T> { } // \#2
};

S<char> s1;    // error: no matching constructor
S<char[8]> s2; // OK: calls \#2
\end{codeblock}

Even though neither constructor for \tcode{S<char>} will be selected by
overload resolution, they remain a part of the class template specialization. 
% 
This also has the effect of suppressing the implicit generation of a default
constructor (\cxxref{class.ctor}).
\exitexample

\enterexample
\begin{codeblock}
template<typename T> struct S1 {
  template<typename U> requires false struct Inner1; // OK
};

template<typename T> struct S2 {
  template<typename U> 
    requires sizeof(T[(int)-sizeof(T)]) > 1 // error: ill-formed, no diagnostic required
      struct Inner2;
};
\end{codeblock}
\exitexample
Every instantiation of \tcode{S1} results in a valid type, although any use 
of its nested \tcode{Inner1} template is invalid.
% 
\tcode{S2} is ill-formed, no diagnostic required, since no substitution into 
the constraints of its \tcode{Inner2} template would result in a valid 
expression.
\end{addedblock}
\end{quote}


%%
%% Explicit instantiation
%%
\rSec2[temp.explicit]{Explicit instantiation}

Modify paragraph 8 to ensure that only members whose constraints are 
satisfied are explicitly instantiated during class template 
specialization. The note in the \Cpp Standard is omitted.

\begin{quote}
\setcounter{Paras}{7}
\pnum
An explicit instantiation that names a class template specialization is 
also an explicit instantiation of the same kind (declaration or 
definition) of each of its members (not including members inherited from 
base classes and members that are templates) that has not been previously 
explicitly specialized in the translation unit containing the explicit 
instantiation, \added{and provided that the associated constraints 
(\ref{temp.constr.decl}), if any, of that member are satisfied 
(\ref{temp.constr}) by the template arguments of the explicit instantiation, } 
except as described below.
\end{quote}

Add the following paragraphs to this section. These require an explicit
instantiation of a constrained template to satisfy the template's
associated constraints.

\begin{quote}
\begin{addedblock}
\setcounter{Paras}{13}
\pnum
If the explicit instantiation names a class template specialization
or variable template specialization of a constrained template, then
the \grammarterm{template-arguments} in the
\grammarterm{template-id} of the explicit
instantiation shall satisfy the template's associated constraints
(\ref{temp.constr}).
\enterexample
\begin{codeblock}
template<typename T> concept bool C = sizeof(T) == 1;

template<C T> struct S { };

template struct S<char>;    // OK
template struct S<char[2]>; // error: constraints not satisfied
\end{codeblock}
\exitexample

\pnum
When an explicit instantiation refers to a specialization of a
function template (\ref{temp.deduct.decl}), that 
template's associated constraints shall be satisfied by the template 
arguments of the explicit instantiation.

\enterexample
\begin{codeblock}
template<typename T> concept bool C = requires (T t) { -t; };

template<C T>        void f(T) { } // \#1
template<typename T> void g(T) { } // \#2
template<C T>        void g(T) { } // \#3

template void f(int);   // OK: refers to \#1
template void f(void*); // error: no matching template
template void g(int);   // OK: refers to \#3
template void g(void*); // OK: refers to \#2
\end{codeblock}
\exitexample
\end{addedblock}
\end{quote}


%%
%% Explicit specialization
%%
\rSec2[temp.expl.spec]{Explicit specialization}

Insert the following paragraphs after paragraph 12. These require
an explicit specialization to satisfy the constraints of the primary
template.

\begin{quote}
\begin{addedblock}
\setcounter{Paras}{12}
\pnum
The \grammarterm{template-argument}{s} in the
\grammarterm{template-id} of an explicit 
specialization of a constrained non-function template shall satisfy the 
associated constraints of that template, if any 
(\ref{temp.constr}).
% 
\enterexample
\begin{codeblock}
template<typename T> concept bool C = sizeof(T) == 1;

template<C T> struct S { };

template<> struct S<char> { };    // OK
template<> struct S<char[2]> { }; // error: constraints not satisfied
\end{codeblock}
\exitexample

\pnum
When determining the function template referred to by an explicit 
specialization of a function template (\ref{temp.deduct.decl}),
the associated constraints of that template (if any) shall be satisfied 
(\ref{temp.constr}) by the template arguments of the explicit specialization.

\enterexample
\begin{codeblock}
template<typename T> concept bool C = requires (T t) { -t; };

template<C T>        void f(T); // \#1
template<typename T> void g(T); // \#2
template<C T>        void g(T); // \#3

template<> void f(int);   // OK: refers to \#1
template<> void f(void*); // error: no matching template
template<> void g(int);   // OK: refers to \#3
template<> void g(void*); // OK: refers to \#2
\end{codeblock}
\exitexample
\end{addedblock}
\end{quote}


%%
%% Function template specializations
%%
\rSec1[temp.fct.spec]{Function template specializations}

%%
%% Template argument deduction
%%
\rSec2[temp.deduct]{Template argument deduction}

Add the following sentences to the end of paragraph 5. This defines
the substitution of template arguments into a function template's
associated constraints. Note that the last part of paragraph 5
has been duplicated in order to provide context for the addition.

\begin{quote}
\setcounter{Paras}{4}
\pnum
When all template arguments have been deduced or obtained from default 
template arguments, all uses of template parameters in the template 
parameter list of the template and the function type are replaced with
the corresponding deduced or default argument values. 

If the substitution results in an invalid type, as described above, type 
deduction fails.

\added{If the function template has associated constraints (\ref{temp.constr.decl}),
the template arguments are substituted into the associated constraints
without evaluating the resulting expression. If this substitution
results in an invalid type or expression, type deduction fails.
% 
\enternote
The satisfaction of constraints (\ref{temp.constr})
associated with the function template specialization is determined during 
overload resolution (\ref{over.match}), and not at 
the point of substitution.
\exitnote}
\end{quote}

\setcounter{subsection}{5}
\rSec3[temp.deduct.decl]{Deducing template arguments from a function declaration}

Add the following after paragraph 1 in order to require the
satisfaction of constraints when matching a specialization to a
template.

\begin{quote}
\begin{addedblock}
\setcounter{Paras}{2}
\pnum
Remove from the set of function templates considered all those
whose associated constraints (if any) are not satisfied by the deduced
template arguments (\ref{temp.constr}).
\end{addedblock}
\end{quote}

Update paragraph 2 (now paragraph 3) to accommodate the new wording.

\begin{quote}
\pnum
If, 
\removed{for the set of function templates so considered}
\added{for the remaining function templates},
there is either no match or more than one match after partial ordering 
has been considered (\ref{temp.func.order}), deduction fails 
and, in the declaration cases, the program is ill-formed.
\end{quote}


%%
%% Template constraints
%%
\rSec1[temp.constr]{Template constraints}

Add this section after 14.9.

\begin{quote}

\pnum
\enternote
This section defines the meaning of constraints on template arguments.
% 
The abstract syntax, satisfaction rules, and equivalence rules are defined
in \ref{temp.constr.constr}. 
% 
Constraints are associated with declarations in \ref{temp.constr.decl}.
% 
Declarations are partially ordered by their associated constraints 
(\ref{temp.constr.order}).
\exitnote


%%
%% Constraints
%%
\rSec2[temp.constr.constr]{Constraints}

\pnum
A \defn{constraint} is a sequence of logical operations and 
operands that specifies requirements on template arguments.
\enternote The operands of a logical operation are constraints. \exitnote
% 
There are several different kinds of constraints:
\begin{itemize}
\item conjunctions (\ref{temp.constr.op}),
\item disjunctions (\ref{temp.constr.op}),
\item predicate constraints (\ref{temp.constr.pred}),
\item expression constraints (\ref{temp.constr.expr}),
\item type constraints (\ref{temp.constr.type}),
\item implicit conversion constraints (\ref{temp.constr.conv}),
\item argument deduction constraints (\ref{temp.constr.deduct}),
\item exception constraints (\ref{temp.constr.noexcept}), and
\item parameterized constraints (\ref{temp.constr.param})
\end{itemize}

\pnum
In order for a constrained template to be instantiated (\ref{temp.spec}), its 
associated constraints (\ref{temp.constr.decl}) shall be \defn{satisfied}.
% 
\enternote
The satisfaction of constraints on class templates, alias templates, 
and variable templates is required when referring a template specialization 
(\ref{temp.names}). The satisfaction of constraints on functions and
function templates is required during overload resolution (\ref{over.match.viable}).
\exitnote
% 
Determining if a constraint is satisfied entails the the substitution 
of template arguments into that constraint.
% 
The rules for determining the satisfaction of different kinds of 
constraints are defined in the following subsections.

\pnum 
In certain contexts, it is necessary to know when two constraints are equivalent
(\ref{temp.constr.decl}). 
% 
The rules for determining the equivalence of different kinds of
constraints are defined in the following subsections.
% 
Two constraints that are not equivalent are \defn{functionally equivalent} if,
for any given set of template arguments, either both constraints are satisfied
or both constraints are unsatisfied.


%%
%% Logical operations
%%
\rSec3[temp.constr.op]{Logical operations}

\pnum
There are two binary logical operations on constraints: conjunction
and disjunction.
% 
\enternote 
These logical operations have no corresponding \Cpp syntax.
For the purpose of exposition, conjunction is spelled
using the symbol $\land$ and disjunction is spelled using the 
symbol $\lor$. 
% 
The operands of these operations are called the left 
and right operands. In the constraint \tcode{P $\land$ Q},
\tcode{P} is the left operand and \tcode{Q} is the right operand.
% 
Grouping of constraints is shown using parentheses.
\exitnote

\pnum
A \defn{conjunction} is a constraint taking two 
operands. A conjunction of constraints is satisfied if and only 
if both operands are satisfied. 
% 
The satisfaction of a conjunction's operands are evaluated left-to-right; 
if the left operand is not satisfied, template arguments are not 
substituted into the right operand, and the constraint is not satisfied.
% 
If the left and right operands of a conjunction are predicate constraints
(\ref{temp.constr.pred}), let \tcode{P} and \tcode{Q} be the expressions
of those constraints resulting from substitution. If the expression
\tcode{P \&\& Q} results in a call to a user-declared \tcode{operator\&\&},
the program is ill-formed.
% 
\enterexample
\begin{codeblock}
template<typename T>
  constexpr bool fail() { return T::value; }

template<typename T>
  requires sizeof(T) > 1 && get_value<T>()
    void f(T);   // has associated constraint \tcode{sizeof(T) > 1 $\land$ get_value<T>()}

void f(int);

f('a'); // OK: calls \tcode{f(int)}
\end{codeblock}
In the satisfaction of the associated constraints (\ref{temp.constr.decl}) 
of \tcode{f}, the constraint \tcode{sizeof(char) > 1} is not satisfied; 
arguments are not substituted into the right operand of the conjunction.
% 
Such a substitution would cause this program to be ill-formed since 
\tcode{get_value<char>()} produces an invalid expression that is not in
the immediate context (\ref{temp.deduct}.
\exitexample


% \pnum
A conjunction \tcode{P} is equivalent to another conjunction \tcode{Q}
if and only if the left operands of \tcode{P} and \tcode{Q} are equivalent
and the right operands of \tcode{P} and \tcode{Q} are equivalent.

\pnum
A \defn{disjunction} is a constraint taking two 
operands. A disjunction of constraints is satisfied if and only 
if either operand is satisfied or both operands are satisfied.
% 
The satisfaction of a disjunction's operands are evaluated left-to-right; 
if the left operand is satisfied, template arguments are not 
substituted into the right operand, and the constraint is satisfied.
%
If the left and right operands of a conjunction are predicate constraints
(\ref{temp.constr.pred}), let \tcode{P} and \tcode{Q} be the expressions
of those constraints resulting from substitution. If the expression
\tcode{P || Q} results in a call to a user-declared \tcode{operator||},
the program is ill-formed.

% \pnum
A disjunction \tcode{P} is equivalent to another disjunction \tcode{Q}
if and only if the left operands of \tcode{P} and \tcode{Q} are equivalent
and the right operands of \tcode{P} and \tcode{Q} are equivalent.

\pnum
\enternote
The prohibition against user-declared logical operators disallows
\grammarterm{constraint-expression}{s} (\ref{temp.constr.decl}) whose 
evaluation disagrees with the satisfaction of its derived constraint.
That is, for any atomic predicate constraints \tcode{P} and \tcode{Q},
the conjunction \tcode{P $\land$ Q} is satisfied if and only if
the \grammarterm{constraint-expression} \tcode{P \&\& Q} evaluates to
\tcode{true}. Likewise, the disjunction \tcode{P $\lor$ Q} is satisfied 
if and only if the \grammarterm{constraint-expression} \tcode{P || Q}
evaluates to \tcode{true}.
\exitnote


%%
%% Predicate constraints
%%
\rSec3[temp.constr.pred]{Predicate constraints}

\pnum
A \defn{predicate constraint} is a constraint that evaluates a constant 
expression \tcode{E} (\cxxref{expr.const}).
% 
\enternote
Predicate constraints allow the definition of template requirements
in terms of constant expressions. This allows the specification 
constraints on non-type template arguments and template template 
arguments.
\exitnote
% 
\enternote
A predicate constraint is introduced by the \grammarterm{constraint-expression}
of a 
\grammarterm{requires-clause} (\ref{temp.constr.decl}), 
or as the associated constraint of a
\grammarterm{constrained-parameter} (\ref{temp.param}) or
\grammarterm{template-introduction} (\ref{temp.intro}).
\exitnote
% 
After substitution, \tcode{E} shall have type \tcode{bool}.
% 
The constraint is satisfied if and only if \tcode{E} evaluates to 
\tcode{true}.
% 
\enterexample
\begin{codeblock}
template<typename T> 
  concept bool C = sizeof(T) == 4 && !true; // requires predicate constraints
                                            // \tcode{sizeof(T) == 4} and \tcode{!t}

template<typename T>
  struct S {
    constexpr explicit operator bool() const { return true; }
  };

template<typename T>
  requires S<T>{}
    void f(T);

f(0); // error: constraints cannot be satisfied because the
      // expression \tcode{S<int>\{\}} does not have type \tcode{bool}
\end{codeblock}
No conversions are applied to predicate constraints.
\exitexample

\pnum
A predicate constraint \tcode{P} is equivalent to another predicate
\tcode{Q} if and only if the expressions of \tcode{P} and \tcode{Q}
are equivalent using the rules described in \ref{temp.over.link} to compare
expressions.


%%
%% Expression constraints
%%
\rSec3[temp.constr.expr]{Expression constraints}

\pnum
An \defn{expression constraint} is a constraint
that specifies a requirement on the formation of an
\grammarterm{expression} \tcode{E}
through substitution of template arguments.
% 
An expression constraint is satisfied if substitution 
yielding \tcode{E} did not fail. 
% 
Within an expression constraint, \tcode{E} is an unevaluated 
operand (Clause \ref{expr}).
% 
\enternote
An expression constraint is introduced by the \grammarterm{expression} in 
either a \grammarterm{simple-requirement} (\ref{expr.prim.req.simple})
or \grammarterm{compound-requirement} (\ref{expr.prim.req.compound})
of a \grammarterm{requires-expression}.
\exitnote
% 
\enterexample
\begin{codeblock}
template<typename T> concept bool C = requires (T t) { ++t; };
\end{codeblock}
The concept \tcode{C} introduces an expression constraint for 
the expression \tcode{++t}.
% 
The type argument \tcode{int} satisfies this constraint because the
the expression \tcode{++t} is valid after substituting \tcode{int}
for \tcode{T}.
\exitexample

\pnum
An expression constraint \tcode{P} is equivalent to another expression
constraint \tcode{Q} if and only if the \tcode{expression}{}s of
\tcode{P} and \tcode{Q} are equivalent using the rules described 
in \ref{temp.over.link} to compare expressions.


%%
%% Type constraints
%%
\rSec3[temp.constr.type]{Type constraints}

\pnum
A \defn{type constraint} is a constraint that specifies a requirement 
on the formation of a type \tcode{T} through the substitution of template 
arguments.
% 
A type constraint is satisfied if and only \tcode{T} is not ill-formed, meaning 
that the substitution yielding \tcode{T} did not fail.
% 
\enternote
A type constraint is introduced by the \grammarterm{typename-specifier} in a
\grammarterm{type-requirement} of a \grammarterm{requires-expression}
(\ref{expr.prim.req.type}).
\exitnote
% 
\enterexample
\begin{codeblock}
template<typename T> concept bool C = requires () { typename T::type; };
\end{codeblock}
The concept \tcode{C} introduces a type constraint for the 
type name \tcode{T::type}.
% 
The type \tcode{int} does not satisfy this constraint
because substitution of that type into the constraint results in a
substitution failure; \tcode{typename int::type} is ill-formed.
\exitexample

\pnum
A type constraint that names a class template specialization 
does not require that type to be complete 
(\cxxref{basic.types}).

\pnum
A type constraint \tcode{P} is equivalent to another type
constraint \tcode{Q} if and only if the types in \tcode{P}
and \tcode{Q} are equivalent according to the rules in
\cxxref{temp.type}.


%%
%% Implicit conversion constraints
%%
\rSec3[temp.constr.conv]{Implicit conversion constraints}

\pnum
An \defn{implicit conversion constraint} is a constraint that 
specifies a requirement on the implicit conversion of an \grammarterm{expression}
\tcode{E} to a type \tcode{T}. 
% 
The constraint is satisfied if and only if \tcode{E} is implicitly convertible 
to \tcode{T} (Clause~\cxxref{conv}).
% 
\enternote
A conversion constraint is introduced by a \grammarterm{trailing-return-type} 
in a \grammarterm{compound-requirement} when the 
\grammarterm{trailing-return-type} contains no placeholders 
(\ref{expr.prim.req.compound}).
\exitnote
% 
\enterexample
\begin{codeblock}
template<typename T> concept bool C = 
  requires (T a, T b) {
    { a == b } -> bool;
  };
\end{codeblock}
The \grammarterm{compound-requirement} in the
\grammarterm{requires-expression} of \tcode{C} introduces two atomic 
constraints: an expression constraint for \tcode{a == b}, and the implicit 
conversion constraint that the expression \tcode{a == b} is implicitly 
convertible to \tcode{bool}.
\exitexample

\pnum
An implicit conversion constraint \tcode{P} is equivalent to another implicit 
conversion constraint \tcode{Q} if and only if the \grammarterm{expression}{}s 
of \tcode{P} and \tcode{Q} are equivalent using the rules in
\ref{temp.over.link} to compare expressions, and the the types
of \tcode{P} and \tcode{Q} are equivalent according to the rules in
\cxxref{temp.type}.


%%
%% Argument deduction constraints
%%
\rSec3[temp.constr.deduct]{Argument deduction constraints}

\pnum
An \defn{argument deduction constraint} is a constraint that specifies 
a requirement that the type of an \grammarterm{expression} \tcode{E}
can be deduced from a type \tcode{T}, when \tcode{T} includes one or more 
placeholders (\ref{dcl.spec.auto}).
% 
\enternote
An argument deduction constraint is introduced by a
\grammarterm{compound-requirement} (\ref{expr.prim.req.compound}) having a
\grammarterm{trailing-return-type} that contains one ore more placeholders.
% 
In such a constraint, \tcode{E} is the \grammarterm{expression} of the 
\grammarterm{compound-requirement}, and \tcode{T} is the type specified
by the \grammarterm{trailing-return-type}.
\exitnote

\pnum
To determine if an argument deduction constraint is satisfied, invent
an abbreviated function template \tcode{f} with one parameter whose
type is \tcode{T} (\ref{dcl.fct}). 
% 
The constraint is satisfied if the resolution of the function call 
\tcode{f(E)} succeeds (\ref{over.match}).
% 
\enternote
Overload resolution succeeds when values are deduced for all invented
template parameters in \tcode{f} that correspond to the placeholders in 
\tcode{T}, and the constraints associated by any 
\grammarterm{constrained-type-specifier}{s} are satisfied.
\exitnote
% 
\enterexample
\begin{codeblock}
template<typename T, typename U> struct Pair;

template<typename T>
  concept bool C1() { return true; }

template<typename T>
  concept bool C2() { return requires(T t) { {*t} -> Pair<C1&, auto>; }; }

template<C2 T> void g(T);

g((int*)nullptr); // error: constraints not satisfied.
\end{codeblock}
The invented abbreviated function template \tcode{f} for the 
\grammarterm{compound-requirement} in \tcode{C2} is:
\begin{codeblock}
void f(Pair<C1&, auto>);
\end{codeblock}
In the call \tcode{g((int*)nullptr)}, the constraints are not satisfied 
because no values can be deduced for the placeholders \tcode{C1} and 
\tcode{auto} from the expression \tcode{*t} when \tcode{t} has type
``pointer-to-\tcode{int}''.
\exitexample

\pnum
An argument deduction constraint \tcode{P} is equivalent to another 
argument deduction constraint \tcode{Q} if and only if the 
\grammarterm{expression}{s} of \tcode{P} and \tcode{Q} are equivalent
using the rules in \ref{temp.over.link} to compare expressions, and the types
of \tcode{P} and \tcode{Q} are equivalent (\cxxref{temp.type}).


%%
%% Exception constraints
%%
\rSec3[temp.constr.noexcept]{Exception constraints}

\pnum
An \defn{exception constraint} is a constraint
for an expression \tcode{E} that is satisfied if and only
if the expression \tcode{noexcept(E)} is \tcode{true}
(\cxxref{expr.unary.noexcept}).
% 
\enternote
Exception constraints are introduced by a \grammarterm{compound-requirement} 
that includes the \tcode{noexcept} specifier (\ref{expr.prim.req.compound}).
\exitnote

\pnum
An exception constraint \tcode{P} is equivalent to another predicate
\tcode{Q} if and only if the expressions of \tcode{P} and \tcode{Q}
are equivalent using the rules described in \ref{temp.over.link} to compare
expressions.


%%
%% Parameterized constraints
%%
\rSec3[temp.constr.param]{Parameterized constraints}

\pnum
A \defn{parameterized constraint} is a constraint that declares a sequence
of parameters (\ref{dcl.fct}), called \defn{constraint variables}, and has a 
single operand. 
% 
\enternote
Parameterized constraints are introduced by 
\grammarterm{requires-expression}{s} (\ref{expr.prim.req}). The constraint
variables of a parameterized constraint correspond to the 
parameters declared in the \grammarterm{requirement-parameter-list} of a
\grammarterm{requires-expression}, and the operand of the constraint
is the conjunction of constraints.
\exitnote
% 
\enternote 
Parameterized constraints have no corresponding C++ syntax. For the purpose of 
exposition, a parameterized constraint is written as, e.g.,
\tcode{$\lambda$(T x, U y) P(x, y)}, where \tcode{x} and \tcode{y} are
constraint variables, and \tcode{P(x, y)} is that constraint's operand
written in terms of \tcode{x} and \tcode{y}.
\exitnote
% 
\enterexample
\begin{codeblock}
template<typename T>
  concept bool Eq = requires (T a, T b) {
    a == b;
    a != b;
  };
\end{codeblock}
The concept \tcode{Eq} defines the parameterized constraint
\tcode{$\lambda$(T a, T b) P(a, b)} where \tcode{P(a, b)} is the conjunction
of two expression constraints: \tcode{a == b} and \tcode{a != b} must be
valid expressions (\ref{temp.constr.expr}).
\exitexample

\pnum
A parameterized constraint is satisfied if and only substitution into 
the types of its constraint variables does not result in an invalid type, and its
operand is satisfied. Template arguments are substituted into the declared
constraint variables in the order in which they are declared. If substitution 
into a constraint variable fails, no more substitutions are performed, and
the constraint is not satisfied.

\pnum
Two parameterized constraints \tcode{P} and \tcode{Q} are equivalent 
if and only if their operands are equivalent.
% 
Two expressions involving constraint variables are equivalent if they
are equivalent according to the rules for expressions described in 
\ref{temp.over.link}, except that any \grammarterm{identifier}{s} referring to 
constraint variables are equivalent if and only if the types of their 
corresponding declarations are equivalent (\cxxref{temp.type}).

\pnum
A constraint variable shall not appear as an evaluated operand 
(\ref{expr}) of a predicate constraint (\ref{temp.constr.pred}).
\enterexample
\begin{codeblock}
template<typename T> 
  concept bool C = requires (T a) {
    requires sizeof(a) == 4; // OK
    requires a == 0;         // error: evaluation of a constraint variable
  }
\end{codeblock}
\exitexample


%%
%% Constrained declarations
%%
\rSec2[temp.constr.decl]{Constrained declarations}

\pnum
A template declaration (Clause~\ref{temp}) or function declaration 
(\ref{dcl.fct}) can be constrained by the use of a 
\grammarterm{requires-clause}. 
% 
This allows the specification of constraints for that declaration as
an expression:

\begin{bnf}
\nontermdef{constraint-expression}\br
    logical-or-expression
\end{bnf}

A \grammarterm{constraint-expression} introduces a predicate constraint
(\ref{temp.constr.pred}) for its \grammarterm{logical-or-expression}
(\cxxref{expr.log.or}).

\pnum
Constraints can also be associated through the use of 
\grammarterm{template-introduction}{}s, 
\grammarterm{constrained-parameter}{}s in a 
\grammarterm{template-parameter-list}, or 
\grammarterm{constrained-type-specifier}{}s in the parameter-type-list
of a function template.
% 
A template's \defn{associated constraints} are the conjunction of 
constraints introduced in its \grammarterm{template-declaration}. 
% 
The ordering of operands in the that conjunction is:
% 
\begin{itemize}
\item a \grammarterm{template-introduction} (\ref{temp.intro}), and

\item all \grammarterm{constrained-parameter}{s} 
      (\ref{temp.param}) in the declaration's 
      \grammarterm{template-parameter-list}, in 
      order of appearance, and

\item the predicate constraint (\ref{temp.constr.pred}) 
      of a \grammarterm{requires-clause} following a 
      \grammarterm{template-parameter-list} (Clause~\ref{temp}), and

\item all \grammarterm{constrained-type-specifier}{s} 
      (\ref{dcl.spec.auto.constr}) in the type of a 
      \grammarterm{parameter-declaration} in a function declaration
      (\ref{dcl.fct}), in order of appearance, and

\item the predicate constraint of a trailing 
      \grammarterm{requires-clause} (Clause~\ref{dcl.decl}) 
      of a function declaration (\ref{dcl.fct}).
\end{itemize}
% 
The formation of the associated constraints for a template declaration
defines the order in which constraints are compared for equivalence
(to determine when one template redeclares another), and the order in
which template arguments are substituted when checking for satisfaction
(\ref{temp.constr.constr}).
% 
A program containing two declarations whose associated constraints are 
functionally equivalent but not equivalent (\ref{temp.constr.constr}) is 
ill-formed, no diagnostic required.
% 
\enterexample
\begin{codeblock}
template<typename T> concept bool C = true;

void f1(C);
template<C T> void f1(T);
C{T} void f1(T);
template<typename T> requires C<T> void f1(T);
template<typename T> void f1(T) requires C<T>;
\end{codeblock}
All declarations of \tcode{f1} declare the same function.
% 
\begin{codeblock}
template<typename T> concept bool C1 = true;
template<typename T> concept bool C2 = sizeof(T) > 0;

template<C1 T> void f2(T) requires C2<T>;                // \#1
template<typename T> requires C1<T> && C2<T> void f2(T); // \#2, redeclaration of \#1
\end{codeblock}
The associated constraints of \#1 are \tcode{C1<T> $\land$ C2<T>}, and
those of \#2 are also \tcode{C1<T> $\land$ C2<T>}.
% 
\begin{codeblock}
template<C1 T> requires C2<T> void f3();
template<C2 T> requires C1<T> void f3(); // error: constraints are functionally
                                         // equivalent but not equivalent
\end{codeblock}
% 
The associated constraints of the first declaration are
\tcode{C1<T> $\land$ C2<T>}, and those of the second are
\tcode{C2<T> $\land$ C1<T>}.
\exitexample

\pnum
Determining if a declaration's associated constraints are satisfied
and partially ordering declarations by their associated constraints
requires the \defn{normalization} of the associated constraints. 
% 
Normalization transforms a constraint into a
sequence of conjunctions and disjunctions of \defn{atomic constraints}.
% 
An atomic constraint (as defined below) is one that cannot be expressed
as a conjunction or disjunction of its operands.
% 
The \defn{normal form} of a constraint is defined as follows:
% 
\begin{itemize}
\item The normal form of the disjunction \tcode{P} $\lor$ \tcode{Q} is the 
disjunction of the normal form of \tcode{P} and the normal form \tcode{Q}.

\item The normal form of the conjunction \tcode{P} $\land$ \tcode{Q} is the 
conjunction of the normal form of \tcode{P} and the normal form \tcode{Q}.

\item The normal form of a predicate constraint \tcode{P} is formed by
first creating a new expression \tcode{E} from the expression of \tcode{P} by 
replacing all subexpressions in \tcode{P} that refer to concepts with their 
corresponding definitions. In particular,
% 
\begin{itemize}

\item replace all function calls of the form 
\tcode{C<A$1$, A$2$, ..., A$N$>()}, where \tcode{C<A$1$, A$2$, ..., A$N$>} 
names the specialization of a function concept \tcode{D} (\ref{dcl.spec.concept}),
with the result of substituting \tcode{A$1$, A$2$, ..., A$N$} into the 
expression returned by \tcode{D}, and

\item replace all \grammarterm{id-expression}{s} of the form 
\tcode{C<A$1$, A$2$, ..., A$N$>}, where \tcode{C<A$1$, A$2$, ..., A$N$>} 
names the specialization of a variable concept \tcode{D} (\ref{dcl.spec.concept}),
with the result of substituting \tcode{A$1$, A$2$, ..., A$N$} 
into the initializer of \tcode{D}.
\end{itemize}
% 
If any such substitution fails, the program is ill-formed.
% 
Second, transform the expression \tcode{E} into a constraint as follows:
\begin{itemize}
\item An expression \tcode{(P)} is transformed into the normalized predicate 
constraint \tcode{P}.

\item An expression \tcode{P || Q} is transformed into the normalized 
disjunction of the predicate constraints \tcode{P} and \tcode{Q}.

\item An expression \tcode{P \&\& Q} is transformed into the normalized
conjunction of the predicate constraints \tcode{P} and \tcode{Q}.

\item A \grammarterm{requires-expression}, (\ref{expr.prim.req}) having the
form
 
\begin{ncsimplebnf}
\terminal{requires} \terminal{(} parameter-declaration-clause \terminal{)} requirement-body
\end{ncsimplebnf}
% 
where the \grammarterm{parameter-declaration-clause} is not equivalent to
an empty parameter list, 
is transformed into a parameterized constraint (\ref{temp.constr.param})
with the same parameters as those in the \grammarterm{parameter-declaration-clause} 
and whose operand is the normal form of conjunction of constraints introduced
by \grammarterm{requirement}{s} in the \grammarterm{requirement-body}.

\item A \grammarterm{requires-expression} having one of the following forms

\begin{ncsimplebnf}
\terminal{requires} \terminal{( void )} requirement-body\br
\terminal{requires} \terminal{()} requirement-body\br
\terminal{requires} requirement-body\br
\end{ncsimplebnf}
% 
is transformed into the normal form of conjunction of constraints introduced
by \grammarterm{requirement}{s} in the \grammarterm{requirement-body}.

\item Otherwise, the expression \tcode{E} is an atomic predicate constraint.
\end{itemize}

\item Expression constraints (\ref{temp.constr.expr}), 
type constraints (\ref{temp.constr.type}), 
implicit conversion constraints (\ref{temp.constr.conv}), 
argument deduction constraints (\ref{temp.constr.deduct}), and 
exception constraints (\ref{temp.constr.expr})
are all atomic constraints and in normal form.
\end{itemize}
% 
\enterexample
\begin{codeblock}
template<typename T> concept bool C1() { return sizeof(T) == 1; }
template<typename T> concept bool C2 = C1<T>() && 1 == 2;
template<typename T> concept bool C3 = requires { typename T::type; };
template<typename T> concept bool C4 = requires (T x) { ++x; }

template<C2 T> void f1(T);                            // \#1
template<C3 T> void f2(T);                            // \#2
template<C4 T> void f3(T);                            // \#3
template<typename T> requires (bool)3 + 4 void f4(T); // error: invalid constraints (\#4)
\end{codeblock}
The normalized associated constraints of \#1 are 
\tcode{sizeof(T) == 1 $\land$ 1 == 2},
% 
those of \#2 are the type constraint for \tcode{T::type},
%
those of \#3 are are the parameterized constraint
\tcode{$\lambda$(T x)} the expression constraint \tcode{++x}.
% 
Note that the normalized constraints of \#2 includes two atomic 
constraints: \tcode{sizeof(char) == 1} and \tcode{1 == 2}.
% 
In \#4, the \grammarterm{constraint-expression} \tcode{(bool)3 + 4}
is not a valid predicate constraint because it does not have type \tcode{bool}.
\exitexample


%%
%% Partial ordering by constraints
%%
\rSec2[temp.constr.order]{Partial ordering by constraints}

\pnum
A constraint \tcode{P} is said to \defn{subsume} another constraint \tcode{Q} 
if, informally, it can be determined that \tcode{P} implies \tcode{Q}, up to 
the equivalence of types and expressions in \tcode{P} and \tcode{Q}.
% 
\enterexample
Subsumption does not determine if the predicate constraint 
\tcode{N >= 0} (\ref{temp.constr.pred}) subsumes \tcode{N > 0} for some 
integral template argument \tcode{N}.
\exitexample

% TODO: If I add a quantified constraint (which I will), then this
% rewriting needs to lift its conjunctions out of it.

\pnum
In order to determine if a constraint \tcode{P} subsumes a constraint
\tcode{Q}, transform \tcode{P} into disjunctive normal form, 
and transform \tcode{Q} into conjunctive normal form\footnote{
A constraint is in disjunctive normal form when it is a disjunction of
clauses where each clause is a conjunction of atomic constraints. 
% 
Similarly, a constraint is in conjunctive normal form when it is a conjunction 
of clauses where each each clause is disjunction of atomic constraints.
% 
\enterexample
Let \tcode{A}, \tcode{B}, and \tcode{C} be atomic constraints.
% 
The constraint \tcode{A} $\land$ (\tcode{B} $\lor$ \tcode{C}) is in 
conjunctive normal form.
% 
Its conjunctive clauses are \tcode{A} and (\tcode{B} $\lor$ \tcode{C}).
% 
The disjunctive normal form of the constraint
\tcode{A} $\land$ (\tcode{B} $\lor$ \tcode{C}) 
is
(\tcode{A} $\land$ \tcode{B}) $\lor$ (\tcode{A} $\land$ \tcode{C}).
% 
Its disjunctive clauses are (\tcode{A} $\land$ \tcode{B}) and 
(\tcode{A} $\land$ \tcode{C}).
\exitexample
}.
% 
Parameterized constraints do not appear in conjunctive or disjunctive normal
forms. For the purpose of this transformation, the constraint
\tcode{$\lambda$(T x) P(x)} is equivalent to the constraint \tcode{P(x)}.
% 
Then, \tcode{P} subsumes \tcode{Q} if and only if
\begin{itemize}
\item for every disjunctive clause \tcode{P$i$} in the disjunctive normal 
form of \tcode{P}, \tcode{P$i$} subsumes every conjunctive clause \tcode{Q$j$} 
in the conjuctive normal form of \tcode{Q}, where

\item a disjunctive clause \tcode{P$i$} subsumes a conjunctive clause
\tcode{Q$j$} if and only if each atomic constraint in \tcode{P$i$} subsumes 
any atomic constraint \tcode{Q$j$}, where

\item an atomic constraint \tcode{A} subsumes another atomic constraint
\tcode{B} if and only if the \tcode{A} and \tcode{B} are equivalent using the
rules described in \ref{temp.constr.constr} to compare constraints.
\end{itemize}
% 
\enterexample
Let \tcode{A} and \tcode{B} be atomic constraints (\ref{temp.constr.pred}).
% 
The constraint \tcode{A $\land$ B} subsumes \tcode{A}, 
but \tcode{A} does not subsume \tcode{A $\land$ B}. 
% 
The constraint \tcode{A} subsumes \tcode{A $\lor$ B}, but
\tcode{A $\lor$ B} does not subsume \tcode{A}. 
% 
Also note that every constraint subsumes itself.
\exitexample


\pnum
The subsumption relation defines a partial ordering on constraints. 
This partial ordering is used to determine
% 
\begin{itemize}
\item the best viable candidate of non-template functions
     (\ref{over.match.best}), 
\item the address of a non-template function
     (\ref{over.over}), 
\item the matching of template template arguments
     (\ref{temp.arg.template}), 
\item the partial ordering of class template specializations
     (\ref{temp.class.order}), and
\item the partial ordering of function templates
     (\ref{temp.func.order}).
\end{itemize}

\pnum
When two declarations \tcode{D1} and \tcode{D2} are
partially ordered by their normalized constraints, \tcode{D1} is 
\defn{at least as constrained} as \tcode{D2} if
% 
\begin{itemize}
\item \tcode{D1} and \tcode{D2} are both constrained
declarations and \tcode{D1}'s associated constraints subsume
those of \tcode{D2}; or

\item \tcode{D2} is
unconstrained. 
\end{itemize}
% 
\pnum
A declaration \tcode{D1} is \defn{more constrained}
than another declaration \tcode{D2} when \tcode{D1} is at least as
constrained as \tcode{D2}, and \tcode{D2} is not at least as
constrained as \tcode{D1}.

\enterexample
\begin{codeblock}
template<typename T> concept bool C1 = requires(T t) { --t; };
template<typename T> concept bool C2 = C1<T> && requires(T t) { *t; };

template<C1 T> void f(T);       // \#1
template<C2 T> void f(T);       // \#2
template<typename T> void g(T); // \#3
template<C1 T> void g(T);       // \#4

f(0);       // selects \#1
f((int*)0); // selects \#2
g(true);    // selects \#3 because \tcode{C1<bool>} is not satisfied
g(0);       // selects \#4
\end{codeblock}
\exitexample


%%
%% Resolution of constrained-type-specifiers
%%
\rSec2[temp.constr.resolve]{Resolution of \grammarterm{qualified-concept-name}{s}}

\pnum
\defn{Concept resolution} is the process of selecting a concept 
from a set of concept definitions referred to by a 
\grammarterm{qualified-concept-name}.
% 
Concept resolution is performed when a \grammarterm{qualified-concept-name} 
appears
\begin{itemize}
\item as a \grammarterm{constrained-type-specifier} (\ref{dcl.spec.auto.constr}), 
\item in a \grammarterm{constrained-parameter} (\ref{temp.param}), or 
\item in a \grammarterm{template-introduction} (\ref{temp.intro}).
\end{itemize}

\pnum
Concept resolution selects a concept from a set referred to by an
optional \grammarterm{nested-name-specifier} and the \grammarterm{concept-name} 
of a \grammarterm{qualified-concept-name} by matching the template parameters 
of each concept in that set to a sequence of template arguments and 
\defn{wildcard}{s}.
% 
This sequence is called the \defn{concept argument list}, and its elements
are called \defn{concept argument}{s}.
% 
For the purpose this matching, a wildcard can match a template 
parameter of any kind (type, non-type, template) as described below.

\pnum
The method for determining the concept argument list depends on the
context in which \grammarterm{concept-name} \tcode{C} appears.
% 
\begin{itemize}
\item If \tcode{C} is part of a \grammarterm{constrained-type-specifier} or 
\grammarterm{constrained-parameter},
then
  \begin{itemize}
  \item if \tcode{C} is a \grammarterm{constrained-type-name}, the concept 
  argument list is comprised of a single wildcard, or
  \item if \tcode{C} is the \grammarterm{concept-name} of a 
  \grammarterm{partial-concept-id}, the concept argument list is comprised of a
  single wildcard followed by the \grammarterm{template-argument}{s} of that 
  \grammarterm{partial-concept-id}.
  \end{itemize}

\item If \tcode{C} is the \grammarterm{concept-name} in a
\grammarterm{template-introduction}. the concept argument list is a sequence 
wildcards of the same length as the \grammarterm{introduction-list} of
the \grammarterm{template-introduction}.

\item If \tcode{C} appears as a \grammarterm{template-name} of a
\grammarterm{template-id}, the concept argument list is the sequence of
\grammarterm{template-argument}{s} of the \grammarterm{template-id}.
\end{itemize}


\pnum
The selection of a concept from the set referred to by the 
\grammarterm{concept-name} \tcode{C} is done by matching the
concept argument list against the template parameter lists of each
concept in that set.
% 
For a concept \tcode{CC} in that set to be a viable selection, each 
argument in the concept argument list is matched against the corresponding 
template parameters of \tcode{CC}.
% 
Default template arguments (if present) are instantiated for each template 
parameter that does not correspond to a concept argument. Instantiated
default arguments are appended to the concept argument list.
% 
If the last declared template parameter of \tcode{CC} is not a parameter pack
and the number of template parameters of \tcode{CC} is greater than the
number of concept arguments, \tcode{CC} is not a viable selection.
% 
Otherwise, concept arguments are matched to template parameters using the 
following rules:
% 
\begin{itemize}
\item a template argument matches a template parameter if and only if
it matches in kind (type, non-type, template) and type according to the
rules in \ref{temp.arg};

\item a wildcard matches a template parameter of any kind;

\item a template parameter pack (\ref{temp.variadic}), matches zero or more 
concept arguments, provided that each of those arguments matches the pattern 
of the template parameter pack using the rules above for matching matching 
concept arguments and template parameters.
\end{itemize}
% 
If any concept arguments do not match a corresponding template parameter,
the concept \tcode{CC} is not a viable selection.
% 
The concept selected by concept resolution shall be the single viable selection
in the set of concepts referred by \tcode{C}.
% 
\enterexample
\begin{codeblock}
template<typename T> concept bool C1() { return true; }             // \#1
template<typename T, typename U> concept bool C1() { return true; } // \#2
template<typename T> concept bool C2() { return true; }
template<int T> concept bool C2() { return true; }
template<typename... Ts> concept bool C3 = true;

void f1(const C1*); // OK: \tcode{C1} selects \#1
void f2(C1<char>);  // OK: \tcode{C1<char>} selects \#2

template<C2<0> T> struct S1; // error: no matching concept for \tcode{C<0>},
                             // mismatched template arguments
template<C2 T> struct S2;    // error: resolution of \tcode{C2} is ambiguous,
                             // both concepts are viable

Q{...Ts} void q1(); // OK: selects \tcode{Q}
Q{T} void q2();     // OK: selects \tcode{Q}
\end{codeblock}
\exitexample

\end{quote}


