
%%
%% Template constraints
%%
\rSec1[temp.constr]{Template constraints}

Add this section after 14.9.

\begin{quote}

\pnum
\enternote
This section defines the meaning of constraints on template arguments.
% 
The abstract syntax, satisfaction rules, and equivalence rules are defined
in \ref{temp.constr.constr}. 
% 
Constraints are associated with declarations in \ref{temp.constr.decl}.
% 
Declarations are partially ordered by their associated constraints 
(\ref{temp.constr.order}).
\exitnote


%%
%% Constraints
%%
\rSec2[temp.constr.constr]{Constraints}

\pnum
A \defn{constraint} is a sequence of logical operations and 
operands that specifies requirements on template arguments.
\enternote The operands of a logical operation are constraints. \exitnote
% 
There are several different kinds of constraints:
\begin{itemize}
\item conjunctions (\ref{temp.constr.op}),
\item disjunctions (\ref{temp.constr.op}),
\item predicate constraints (\ref{temp.constr.pred}),
\item expression constraints (\ref{temp.constr.expr}),
\item type constraints (\ref{temp.constr.type}),
\item implicit conversion constraints (\ref{temp.constr.conv}),
\item argument deduction constraints (\ref{temp.constr.deduct}),
\item exception constraints (\ref{temp.constr.noexcept}), and
\item parameterized constraints (\ref{temp.constr.param})
\end{itemize}

\pnum
In order for a constrained template to be instantiated (\ref{temp.spec}), its 
associated constraints (\ref{temp.constr.decl}) shall be \defn{satisfied}.
% 
\enternote
The satisfaction of constraints on class templates, alias templates, 
and variable templates is required when referring to a template specialization 
(\ref{temp.names}). The satisfaction of constraints on functions and
function templates is required during overload resolution 
(\ref{over.match.viable}). The satisfaction of constraints introduced by
\grammarterm{constrained-type-specifier}{}s (\ref{dcl.spec.auto.constr})
in the type of a variable or declared return type of a function is 
required when values are deduced for their corresponding placeholders 
(\ref{dcl.spec.auto}).
\exitnote
% 
Determining if a constraint is satisfied entails the substitution 
of template arguments into that constraint.
% 
The rules for determining the satisfaction of different kinds of 
constraints are defined in the following subsections.

\pnum 
In certain contexts, it is necessary to know when two constraints are equivalent
(\ref{temp.constr.decl}). 
% 
The rules for determining the equivalence of different kinds of
constraints are defined in the following subsections.
% 
Two constraints that are not equivalent are \defn{functionally equivalent} if,
for any given set of template arguments, either both constraints are satisfied
or both constraints are unsatisfied.


%%
%% Logical operations
%%
\rSec3[temp.constr.op]{Logical operations}

\pnum
There are two binary logical operations on constraints: conjunction
and disjunction.
% 
\enternote 
These logical operations have no corresponding \Cpp syntax.
For the purpose of exposition, conjunction is spelled
using the symbol $\land$ and disjunction is spelled using the 
symbol $\lor$. 
% 
The operands of these operations are called the left 
and right operands. In the constraint $A \land B$,
$A$ is the left operand, and $B$ is the right operand.
% 
Grouping of constraints is shown using parentheses.
\exitnote

\pnum
A \defn{conjunction} is a constraint taking two 
operands. A conjunction of constraints is satisfied if and only 
if both operands are satisfied. 
% 
The satisfaction of a conjunction's operands are evaluated left-to-right; 
if the left operand is not satisfied, template arguments are not 
substituted into the right operand, and the constraint is not satisfied.
% 
If the left and right operands of a conjunction are predicate constraints
(\ref{temp.constr.pred}), let \tcode{P} and \tcode{Q} be the expressions
of those constraints resulting from substitution. If the expression
\tcode{P \&\& Q} results in a call to a user-declared \tcode{operator\&\&},
the program is ill-formed.
% 
\enterexample
\begin{codeblock}
template<typename T>
  constexpr bool get_value() { return T::value; }

template<typename T>
  requires sizeof(T) > 1 && get_value<T>()
    void f(T);   // has associated constraint \tcode{sizeof(T) > 1 $\land$ get_value<T>()}

void f(int);

f('a'); // OK: calls \tcode{f(int)}
\end{codeblock}
In the satisfaction of the associated constraints (\ref{temp.constr.decl}) 
of \tcode{f}, the constraint \tcode{sizeof(char) > 1} is not satisfied; 
arguments are not substituted into the right operand of the conjunction.
% 
Such a substitution would cause this program to be ill-formed since 
\tcode{get_value<char>()} produces an invalid expression that is not in
the immediate context (\ref{temp.deduct}.
\exitexample

\pnum
A conjunction $C$ is equivalent to another conjunction $D$
if and only if the left operands of $C$ and $D$ are equivalent
and the right operands of $C$ and $D$ are equivalent.

\pnum
A \defn{disjunction} is a constraint taking two 
operands. A disjunction of constraints is satisfied if and only 
if either operand is satisfied or both operands are satisfied.
% 
The satisfaction of a disjunction's operands are evaluated left-to-right; 
if the left operand is satisfied, template arguments are not 
substituted into the right operand, and the constraint is satisfied.
%
If the left and right operands of a conjunction are predicate constraints
(\ref{temp.constr.pred}), let \tcode{P} and \tcode{Q} be the expressions
of those constraints resulting from substitution. If the expression
\tcode{P || Q} results in a call to a user-declared \tcode{operator||},
the program is ill-formed.

\pnum
A disjunction $C$ is equivalent to another disjunction $D$
if and only if the left operands of $C$ and $D$ are equivalent
and the right operands of $C$ and $D$ are equivalent.

\pnum
\enternote
The prohibition against user-declared logical operators disallows
\grammarterm{constraint-expression}{s} (\ref{temp.constr.decl}) whose 
evaluation disagrees with the satisfaction of its derived constraint.
That is, for any atomic predicate constraints $A$ and $B$, whose
respective expressions are \tcode{PA} and \tcode{PB},
the conjunction $A \land B$ is satisfied if and only if
the \grammarterm{constraint-expression} \tcode{PA \&\& PB} evaluates to
\tcode{true}. Likewise, the disjunction $A \lor B$ is satisfied 
if and only if the \grammarterm{constraint-expression} \tcode{PA || PB}
evaluates to \tcode{true}.
\exitnote


%%
%% Predicate constraints
%%
\rSec3[temp.constr.pred]{Predicate constraints}

\pnum
A \defn{predicate constraint} is a constraint that evaluates a constant 
expression \tcode{E} (\cxxref{expr.const}).
% 
\enternote
Predicate constraints allow the definition of template requirements
in terms of constant expressions. This allows the specification of
constraints on non-type template arguments and template template 
arguments.
\exitnote
% 
\enternote
A predicate constraint is introduced by the \grammarterm{constraint-expression}
of a 
\grammarterm{requires-clause} (\ref{temp.constr.decl}), 
or as the associated constraint of a
\grammarterm{constrained-parameter} (\ref{temp.param}) or
\grammarterm{template-introduction} (\ref{temp.intro}).
\exitnote
% 
After substitution, \tcode{E} shall have type \tcode{bool}.
% 
The constraint is satisfied if and only if \tcode{E} evaluates to 
\tcode{true}.
% 
\enterexample
\begin{codeblock}
template<typename T> 
  concept bool C = sizeof(T) == 4 && !true; // requires predicate constraints
                                            // \tcode{sizeof(T) == 4} and \tcode{!true}

template<typename T>
  struct S {
    constexpr explicit operator bool() const { return true; }
  };

template<typename T>
  requires S<T>{}
    void f(T);

f(0); // error: constraints cannot be satisfied because the
      // expression \tcode{S<int>\{\}} does not have type \tcode{bool}
\end{codeblock}
No conversions are applied to predicate constraints.
\exitexample

\pnum
A predicate constraint $A$ is equivalent to another predicate
$B$ if and only if the expressions of $A$ and $B$ are equivalent using the 
rules described in \ref{temp.over.link} to compare
expressions.


%%
%% Expression constraints
%%
\rSec3[temp.constr.expr]{Expression constraints}

\pnum
An \defn{expression constraint} is a constraint
that specifies a requirement on the formation of an
\grammarterm{expression} \tcode{E}
through substitution of template arguments.
% 
An expression constraint is satisfied if substitution 
yielding \tcode{E} did not fail. 
% 
Within an expression constraint, \tcode{E} is an unevaluated 
operand (Clause \ref{expr}).
% 
\enternote
An expression constraint is introduced by the \grammarterm{expression} in 
either a \grammarterm{simple-requirement} (\ref{expr.prim.req.simple})
or \grammarterm{compound-requirement} (\ref{expr.prim.req.compound})
of a \grammarterm{requires-expression}.
\exitnote
% 
\enterexample
\begin{codeblock}
template<typename T> concept bool C = requires (T t) { ++t; };
\end{codeblock}
The concept \tcode{C} introduces an expression constraint for 
the expression \tcode{++t}.
% 
The type argument \tcode{int} satisfies this constraint because the 
expression \tcode{++t} is valid after substituting \tcode{int} for 
\tcode{T}.
\exitexample

\pnum
An expression constraint $A$ is equivalent to another expression
constraint $B$ if and only if the \grammarterm{expression}{}s of
$A$ and $B$ are equivalent using the rules described 
in \ref{temp.over.link} to compare expressions.


%%
%% Type constraints
%%
\rSec3[temp.constr.type]{Type constraints}

\pnum
A \defn{type constraint} is a constraint that specifies a requirement 
on the formation of a type \tcode{T} through the substitution of template 
arguments.
% 
A type constraint is satisfied if and only \tcode{T} is not ill-formed, meaning 
that the substitution yielding \tcode{T} did not fail.
% 
\enternote
A type constraint is introduced by the \grammarterm{typename-specifier} in a
\grammarterm{type-requirement} of a \grammarterm{requires-expression}
(\ref{expr.prim.req.type}).
\exitnote
% 
\enterexample
\begin{codeblock}
template<typename T> concept bool C = requires () { typename T::type; };
\end{codeblock}
The concept \tcode{C} introduces a type constraint for the 
type name \tcode{T::type}.
% 
The type \tcode{int} does not satisfy this constraint
because substitution of that type into the constraint results in a
substitution failure; \tcode{typename int::type} is ill-formed.
\exitexample

\pnum
A type constraint that names a class template specialization 
does not require that type to be complete 
(\cxxref{basic.types}).

\pnum
A type constraint $A$ is equivalent to another type constraint $B$ if and 
only if the types in $A$ and $B$ are equivalent according to the rules in
\cxxref{temp.type}.


%%
%% Implicit conversion constraints
%%
\rSec3[temp.constr.conv]{Implicit conversion constraints}

\pnum
An \defn{implicit conversion constraint} is a constraint that 
specifies a requirement on the implicit conversion of an \grammarterm{expression}
\tcode{E} to a type \tcode{T}. 
% 
The constraint is satisfied if and only if \tcode{E} is implicitly convertible 
to \tcode{T} (Clause~\cxxref{conv}).
% 
\enternote
A conversion constraint is introduced by a \grammarterm{trailing-return-type} 
in a \grammarterm{compound-requirement} when the 
\grammarterm{trailing-return-type} contains no placeholders 
(\ref{expr.prim.req.compound}).
\exitnote
% 
\enterexample
\begin{codeblock}
template<typename T> concept bool C = 
  requires (T a, T b) {
    { a == b } -> bool;
  };

template <typename T> concept bool D =
  requires (T a) {
    { a } -> int;
  };
\end{codeblock}
The \grammarterm{compound-requirement} in the
\grammarterm{requires-expression} of \tcode{C} introduces two atomic 
constraints: an expression constraint for \tcode{a == b}, and the implicit 
conversion constraint that the expression \tcode{a == b} is implicitly 
convertible to \tcode{bool}.
% 
The implicit conversion constraint in \tcode{D} is satisfied whenever
the \tcode{std::is_convertible<T, int>::value} is \tcode{true} (\cxxref{meta.rel}).
\exitexample


\pnum
An implicit conversion constraint $A$ is equivalent to another implicit 
conversion constraint $B$ if and only if the \grammarterm{expression}{}s 
of $A$ and $B$ are equivalent using the rules in
\ref{temp.over.link} to compare expressions, and the types
of $A$ and $B$ are equivalent according to the rules in
\cxxref{temp.type}.


%%
%% Argument deduction constraints
%%
\rSec3[temp.constr.deduct]{Argument deduction constraints}

\pnum
An \defn{argument deduction constraint} is a constraint that specifies 
a requirement that the type of an \grammarterm{expression} \tcode{E}
can be deduced from a type \tcode{T}, when \tcode{T} includes one or more 
placeholders (\ref{dcl.spec.auto}).
% 
\enternote
An argument deduction constraint is introduced by a
\grammarterm{compound-requirement} (\ref{expr.prim.req.compound}) having a
\grammarterm{trailing-return-type} that contains one or more placeholders.
% 
In such a constraint, \tcode{E} is the \grammarterm{expression} of the 
\grammarterm{compound-requirement}, and \tcode{T} is the type specified
by the \grammarterm{trailing-return-type}.
\exitnote

\pnum
To determine if an argument deduction constraint is satisfied, invent
an abbreviated function template \tcode{f} with one parameter whose
type is \tcode{T} (\ref{dcl.fct}). 
% 
The constraint is satisfied if the resolution of the function call 
\tcode{f(E)} succeeds (\ref{over.match}).
% 
\enternote
Overload resolution succeeds when values are deduced for all invented
template parameters in \tcode{f} that correspond to the placeholders in 
\tcode{T}, and the constraints associated by any 
\grammarterm{constrained-type-specifier}{s} are satisfied.
\exitnote
% 
\enterexample
\begin{codeblock}
template<typename T, typename U> struct Pair;

template<typename T>
  concept bool C1() { return true; }

template<typename T>
  concept bool C2() { return requires(T t) { {*t} -> Pair<C1&, auto>; }; }

template<C2 T> void g(T);

g((int*)nullptr); // error: constraints not satisfied.
\end{codeblock}
The invented abbreviated function template \tcode{f} for the 
\grammarterm{compound-requirement} in \tcode{C2} is:
\begin{codeblock}
void f(Pair<C1&, auto>);
\end{codeblock}
In the call \tcode{g((int*)nullptr)}, the constraints are not satisfied 
because no values can be deduced for the placeholders \tcode{C1} and 
\tcode{auto} from the expression \tcode{*t} when \tcode{t} has type
``pointer-to-\tcode{int}''.
\exitexample

\pnum
An argument deduction constraint $A$ is equivalent to another 
argument deduction constraint $B$ if and only if the 
\grammarterm{expression}{s} of $A$ and $B$ are equivalent
using the rules in \ref{temp.over.link} to compare expressions, and the types
of $A$ and $B$ are equivalent (\cxxref{temp.type}).


%%
%% Exception constraints
%%
\rSec3[temp.constr.noexcept]{Exception constraints}

\pnum
An \defn{exception constraint} is a constraint
for an expression \tcode{E} that is satisfied if and only
if the expression \tcode{noexcept(E)} is \tcode{true}
(\cxxref{expr.unary.noexcept}).
% 
\enternote
Exception constraints are introduced by a \grammarterm{compound-requirement} 
that includes the \tcode{noexcept} specifier (\ref{expr.prim.req.compound}).
\exitnote

\pnum
An exception constraint $A$ is equivalent to another exception constraint
$B$ if and only if the expressions of $A$ and $B$ are equivalent using the 
rules described in \ref{temp.over.link} to compare expressions.


%%
%% Parameterized constraints
%%
\rSec3[temp.constr.param]{Parameterized constraints}

\pnum
A \defn{parameterized constraint} is a constraint that declares a sequence
of parameters (\ref{dcl.fct}), called \defn{constraint variables}, and has a 
single operand. 
% 
\enternote
Parameterized constraints are introduced by 
\grammarterm{requires-expression}{s} (\ref{expr.prim.req}). The constraint
variables of a parameterized constraint correspond to the 
parameters declared in the \grammarterm{requirement-parameter-list} of a
\grammarterm{requires-expression}, and the operand of the constraint
is the conjunction of constraints.
\exitnote
% 
\enternote 
Parameterized constraints have no corresponding C++ syntax. For the purpose of 
exposition, a parameterized constraint is written as, e.g.,
$\lambda(\mathtt{T~x}, \mathtt{U~y})~P(\mathtt{x}, \mathtt{y})$, where 
\tcode{x} and \tcode{y} are constraint variables, and $P(\mathtt{x}, \mathtt{y})$ 
is that constraint's operand written in terms of \tcode{x} and \tcode{y}.
\exitnote
% 
\enterexample
\begin{codeblock}
template<typename T>
  concept bool Eq = requires (T a, T b) {
    a == b;
    a != b;
  };
\end{codeblock}
The concept \tcode{Eq} defines the parameterized constraint
$\lambda(\mathtt{T~a}, \mathtt{T~b})~P(\mathtt{a}, \mathtt{b})$ where 
$P(\mathtt{a}, \mathtt{b})$ is the conjunction
of two expression constraints: \tcode{a == b} and \tcode{a != b} must be
valid expressions (\ref{temp.constr.expr}).
\exitexample

\pnum
A parameterized constraint is satisfied if and only if substitution into 
the types of its constraint variables does not result in an invalid type, and its
operand is satisfied. Template arguments are substituted into the declared
constraint variables in the order in which they are declared. If substitution 
into a constraint variable fails, no more substitutions are performed, and
the constraint is not satisfied.

\pnum
Two parameterized constraints $A$ and $B$ are equivalent 
if and only if their operands are equivalent.
% 
Two expressions involving constraint variables are equivalent if they
are equivalent according to the rules for expressions described in 
\ref{temp.over.link}, except that any \grammarterm{identifier}{}s referring to 
constraint variables are equivalent if and only if the types of their 
corresponding declarations are equivalent (\cxxref{temp.type}).

\pnum
A constraint variable shall not appear as an evaluated operand 
(\ref{expr}) of a predicate constraint (\ref{temp.constr.pred}).
\enterexample
\begin{codeblock}
template<typename T> 
  concept bool C = requires (T a) {
    requires sizeof(a) == 4; // OK
    requires a == 0;         // error: evaluation of a constraint variable
  }
\end{codeblock}
\exitexample


%%
%% Constrained declarations
%%
\rSec2[temp.constr.decl]{Constrained declarations}

\pnum
A template declaration (Clause~\ref{temp}) or function declaration 
(\ref{dcl.fct}) can be constrained by the use of a 
\grammarterm{requires-clause}. 
% 
This allows the specification of constraints for that declaration as
an expression:

\begin{bnf}
\nontermdef{constraint-expression}\br
    logical-or-expression
\end{bnf}

A \grammarterm{constraint-expression} introduces a predicate constraint
(\ref{temp.constr.pred}) for its \grammarterm{logical-or-expression}
(\cxxref{expr.log.or}).

\pnum
Constraints can also be associated with a declaration through the use of 
\grammarterm{template-introduction}{}s, 
\grammarterm{constrained-parameter}{}s in a 
\grammarterm{template-parameter-list},
\grammarterm{constrained-type-specifier}{}s in the parameter-type-list
of a function template.
% 
A template's \defn{associated constraints} are the conjunction of 
constraints introduced in its \grammarterm{template-declaration}. 
% 
The ordering of operands in that conjunction is:
% 
\begin{itemize}
\item a \grammarterm{template-introduction} (\ref{temp.intro}), and

\item all \grammarterm{constrained-parameter}{s} 
      (\ref{temp.param}) in the declaration's 
      \grammarterm{template-parameter-list}, in 
      order of appearance, and

\item the predicate constraint (\ref{temp.constr.pred}) 
      of a \grammarterm{requires-clause} following a 
      \grammarterm{template-parameter-list} (Clause~\ref{temp}), and

\item all \grammarterm{constrained-type-specifier}{s} 
      (\ref{dcl.spec.auto.constr}) in the type of a 
      \grammarterm{parameter-declaration} in a function declaration
      (\ref{dcl.fct}), in order of appearance, and

\item the predicate constraint of a trailing 
      \grammarterm{requires-clause} (Clause~\ref{dcl.decl}) 
      of a function declaration (\ref{dcl.fct}).
\end{itemize}
% 
The formation of the associated constraints for a template declaration
defines the order in which constraints are compared for equivalence
(to determine when one template redeclares another), and the order in
which constraints are instantiated when checking for satisfaction
(\ref{temp.constr.constr}).
% 
The constraints associated by \grammarterm{constrained-type-specifier}{}s 
appearing in a variable type or in the declared return type of a function are 
not included in the associated constraints of a template declaration.
\enternote
These constraints are checked during the instantiation of the declaration.
\exitnote
% 
A program containing two declarations whose associated constraints are 
functionally equivalent but not equivalent (\ref{temp.constr.constr}) is 
ill-formed, no diagnostic required.
% 
\enterexample
\begin{codeblock}
template<typename T> concept bool C = true;

void f1(C);
template<C T> void f1(T);
C{T} void f1(T);
template<typename T> requires C<T> void f1(T);
template<typename T> void f1(T) requires C<T>;
\end{codeblock}
All declarations of \tcode{f1} declare the same function.
% 
\begin{codeblock}
template<typename T> concept bool C1 = true;
template<typename T> concept bool C2 = sizeof(T) > 0;

template<C1 T> void f2(T) requires C2<T>;                // \#1
template<typename T> requires C1<T> && C2<T> void f2(T); // \#2, redeclaration of \#1
\end{codeblock}
The associated constraints of \#1 are \tcode{C1<T> $\land$ C2<T>}, and
those of \#2 are also \tcode{C1<T> $\land$ C2<T>}.
% 
\begin{codeblock}
template<C1 T> requires C2<T> void f3();
template<C2 T> requires C1<T> void f3(); // error: constraints are functionally
                                         // equivalent but not equivalent
\end{codeblock}
% 
The associated constraints of the first declaration are
\tcode{C1<T> $\land$ C2<T>}, and those of the second are
\tcode{C2<T> $\land$ C1<T>}.
\exitexample

\pnum
Determining 
the satisfaction of constraints (\ref{temp.constr.constr}) and 
the partial ordering of constraints (\ref{temp.constr.order}) 
requires that they are first \defn{normalized}.
% 
Normalization transforms a constraint into a
sequence of conjunctions and disjunctions of \defn{atomic constraints}.
% 
An atomic constraint (as defined below) is one that cannot be expressed
as a conjunction or disjunction of its operands.
% 
The \defn{normal form} of a constraint is defined as follows:
% 
\begin{itemize}
\item The normal form of the disjunction $P \lor Q$ is the 
disjunction of the normal form of $P$ and the normal form $Q$.

\item The normal form of the conjunction $P \land Q$ is the 
conjunction of the normal form of $P$ and the normal form $Q$.

\item The normal form of a predicate constraint $P$ is formed by
first creating a new expression \tcode{E} from the expression of $P$ by 
replacing all subexpressions in $P$ that refer to concepts with their 
corresponding definitions. In particular,
% 
\begin{itemize}

\item replace all function calls of the form 
\tcode{C<A$1$, A$2$, ..., A$N$>()}, where \tcode{C<A$1$, A$2$, ..., A$N$>} 
names the specialization of a function concept \tcode{D} (\ref{dcl.spec.concept}),
with the result of substituting \tcode{A$1$, A$2$, ..., A$N$} into the 
expression returned by \tcode{D}, and

\item replace all \grammarterm{id-expression}{s} of the form 
\tcode{C<A$1$, A$2$, ..., A$N$>}, where \tcode{C<A$1$, A$2$, ..., A$N$>} 
names the specialization of a variable concept \tcode{D} (\ref{dcl.spec.concept}),
with the result of substituting \tcode{A$1$, A$2$, ..., A$N$} 
into the initializer of \tcode{D}.
\end{itemize}
% 
If any such substitution fails, the program is ill-formed.
% 
Second, transform the expression \tcode{E} into a constraint as follows:
\begin{itemize}
\item An expression \tcode{(E)} is transformed into the normalized predicate 
constraint for \tcode{E}.

\item An expression \tcode{E1 || E2} is transformed into the normalized 
disjunction of the predicate constraints for \tcode{E1} and \tcode{E2}.

\item An expression \tcode{E1 \&\& E2} is transformed into the normalized
conjunction of the predicate constraints for \tcode{E1} and \tcode{E2}.

\item A \grammarterm{requires-expression}, (\ref{expr.prim.req}) having the
form
 
\begin{ncsimplebnf}
\terminal{requires} \terminal{(} parameter-declaration-clause \terminal{)} requirement-body
\end{ncsimplebnf}
% 
where the \grammarterm{parameter-declaration-clause} is not equivalent to
an empty parameter list, 
is transformed into a parameterized constraint (\ref{temp.constr.param})
with the same parameters as those in the \grammarterm{parameter-declaration-clause} 
and whose operand is the normal form of conjunction of constraints introduced
by \grammarterm{requirement}{s} in the \grammarterm{requirement-body}.

\item A \grammarterm{requires-expression} having one of the following forms

\begin{ncsimplebnf}
\terminal{requires} \terminal{( void )} requirement-body\br
\terminal{requires} \terminal{()} requirement-body\br
\terminal{requires} requirement-body\br
\end{ncsimplebnf}
% 
is transformed into the normal form of conjunction of constraints introduced
by \grammarterm{requirement}{s} in the \grammarterm{requirement-body}.

\item Otherwise, the expression \tcode{E} is an atomic predicate constraint.
\end{itemize}

\item Expression constraints (\ref{temp.constr.expr}), 
type constraints (\ref{temp.constr.type}), 
implicit conversion constraints (\ref{temp.constr.conv}), 
argument deduction constraints (\ref{temp.constr.deduct}), and 
exception constraints (\ref{temp.constr.expr})
are all atomic constraints and in normal form.
\end{itemize}
% 
\enterexample
\begin{codeblock}
template<typename T> concept bool C1() { return sizeof(T) == 1; }
template<typename T> concept bool C2 = C1<T>() && 1 == 2;
template<typename T> concept bool C3 = requires { typename T::type; };
template<typename T> concept bool C4 = requires (T x) { ++x; }

template<C2 T> void f1(T);                            // \#1
template<C3 T> void f2(T);                            // \#2
template<C4 T> void f3(T);                            // \#3
template<typename T> requires (bool)3 + 4 void f4(T); // error: invalid constraints (\#4)
\end{codeblock}
The normalized associated constraints of \#1 are 
\tcode{sizeof(T) == 1 $\land$ 1 == 2},
% 
those of \#2 are the type constraint for \tcode{T::type},
%
those of \#3 are the parameterized constraint
$\lambda(\mathtt{T~x})$ the expression constraint \tcode{++x}.
% 
In \#4, the \grammarterm{constraint-expression} \tcode{(bool)3 + 4}
is not a valid predicate constraint because it does not have type \tcode{bool}.
\exitexample


%%
%% Partial ordering by constraints
%%
\rSec2[temp.constr.order]{Partial ordering by constraints}

\pnum
A constraint $P$ is said to \defn{subsume} another constraint $Q$ 
if, informally, it can be determined that $P$ implies $Q$, up to 
the equivalence of types and expressions in $P$ and $Q$.
% 
\enterexample
Subsumption does not determine if the predicate constraint 
\tcode{N >= 0} (\ref{temp.constr.pred}) subsumes \tcode{N > 0} for some 
integral template argument \tcode{N}.
\exitexample

\pnum
In order to determine if a constraint $P$ subsumes a constraint
$Q$, transform $P$ into disjunctive normal form, 
and transform $Q$ into conjunctive normal form\footnote{
A constraint is in disjunctive normal form when it is a disjunction of
clauses where each clause is a conjunction of atomic constraints. 
% 
Similarly, a constraint is in conjunctive normal form when it is a conjunction 
of clauses where each each clause is disjunction of atomic constraints.
% 
\enterexample
Let $A$, $B$, and $C$ be atomic constraints.
% 
The constraint $A \land$ ($B \lor C$) is in 
conjunctive normal form.
% 
Its conjunctive clauses are $A$ and ($B \lor C$).
% 
The disjunctive normal form of the constraint
$A \land$ ($B \lor C$) 
is
($A \land B$) $\lor$ ($A \land C$).
% 
Its disjunctive clauses are ($A \land B$) and 
($A \land C$).
\exitexample
}.
% 
Parameterized constraints do not appear in conjunctive or disjunctive normal
forms. For the purpose of this transformation, the constraint
$\lambda(\mathtt{T~x})~P(\mathtt{x})$ is equivalent to the constraint 
$P(\mathtt{x})$.
% 
Then, $P$ subsumes $Q$ if and only if
\begin{itemize}
\item for every disjunctive clause $P_i$ in the disjunctive normal 
form of $P$, $P_i$ subsumes every conjunctive clause $Q_j$ 
in the conjuctive normal form of \tcode{Q}, where

\item a disjunctive clause $P_i$ subsumes a conjunctive clause
$Q_j$ if and only if each atomic constraint in $P_i$ subsumes 
any atomic constraint $Q_j$, where

\item an atomic constraint $A$ subsumes another atomic constraint
$B$ if and only if the $A$ and $B$ are equivalent using the
rules described in \ref{temp.constr.constr} to compare constraints.
\end{itemize}
% 
\enterexample
Let $A$ and $B$ be atomic constraints (\ref{temp.constr.pred}).
% 
The constraint $A \land B$ subsumes $A$, but $A$ does not subsume $A \land B$.
% 
The constraint $A$ subsumes $A \lor B$, but $A \lor B$ does not subsume $A$.
% 
Also note that every constraint subsumes itself.
\exitexample


\pnum
The subsumption relation defines a partial ordering on constraints. 
This partial ordering is used to determine
% 
\begin{itemize}
\item the best viable candidate of non-template functions
     (\ref{over.match.best}), 
\item the address of a non-template function
     (\ref{over.over}), 
\item the matching of template template arguments
     (\ref{temp.arg.template}), 
\item the partial ordering of class template specializations
     (\ref{temp.class.order}), and
\item the partial ordering of function templates
     (\ref{temp.func.order}).
\end{itemize}

\pnum
When two declarations \tcode{D1} and \tcode{D2} are
partially ordered by their normalized constraints, \tcode{D1} is 
\defn{at least as constrained} as \tcode{D2} if
% 
\begin{itemize}
\item \tcode{D1} and \tcode{D2} are both constrained declarations and 
\tcode{D1}'s associated constraints subsume those of \tcode{D2}; or

\item \tcode{D2} is unconstrained.
\end{itemize}
% 
\pnum
A declaration \tcode{D1} is \defn{more constrained}
than another declaration \tcode{D2} when \tcode{D1} is at least as
constrained as \tcode{D2}, and \tcode{D2} is not at least as
constrained as \tcode{D1}.

\enterexample
\begin{codeblock}
template<typename T> concept bool C1 = requires(T t) { --t; };
template<typename T> concept bool C2 = C1<T> && requires(T t) { *t; };

template<C1 T> void f(T);       // \#1
template<C2 T> void f(T);       // \#2
template<typename T> void g(T); // \#3
template<C1 T> void g(T);       // \#4

f(0);       // selects \#1
f((int*)0); // selects \#2
g(true);    // selects \#3 because \tcode{C1<bool>} is not satisfied
g(0);       // selects \#4
\end{codeblock}
\exitexample


%%
%% Resolution of constrained-type-specifiers
%%
\rSec2[temp.constr.resolve]{Resolution of \grammarterm{qualified-concept-name}{s}}

\pnum
\defn{Concept resolution} is the process of selecting a concept 
from a set of concept definitions referred to by a 
\grammarterm{qualified-concept-name}.
% 
Concept resolution is performed when a \grammarterm{qualified-concept-name} 
appears
\begin{itemize}
\item as a \grammarterm{constrained-type-specifier} (\ref{dcl.spec.auto.constr}), 
\item in a \grammarterm{constrained-parameter} (\ref{temp.param}), or 
\item in a \grammarterm{template-introduction} (\ref{temp.intro}).
\end{itemize}

\pnum
Concept resolution selects a concept from a set referred to by an
optional \grammarterm{nested-name-specifier} and the \grammarterm{concept-name} 
of a \grammarterm{qualified-concept-name} by matching the template parameters 
of each concept in that set to a sequence of template arguments and 
\defn{wildcard}{s}.
% 
This sequence is called the \defn{concept argument list}, and its elements
are called \defn{concept argument}{s}.
% 
For the purpose this matching, a wildcard can match a template 
parameter of any kind (type, non-type, template) as described below.

\pnum
The method for determining the concept argument list depends on the
context in which \grammarterm{concept-name} \tcode{C} appears.
% 
\begin{itemize}
\item If \tcode{C} is part of a \grammarterm{constrained-type-specifier} or 
\grammarterm{constrained-parameter},
then
  \begin{itemize}
  \item if \tcode{C} is a \grammarterm{constrained-type-name}, the concept 
  argument list is comprised of a single wildcard, or
  \item if \tcode{C} is the \grammarterm{concept-name} of a 
  \grammarterm{partial-concept-id}, the concept argument list is comprised of a
  single wildcard followed by the \grammarterm{template-argument}{s} of that 
  \grammarterm{partial-concept-id}.
  \end{itemize}

\item If \tcode{C} is the \grammarterm{concept-name} in a
\grammarterm{template-introduction}. the concept argument list is a sequence 
of wildcards of the same length as the \grammarterm{introduction-list} of
the \grammarterm{template-introduction}.

\item If \tcode{C} appears as a \grammarterm{template-name} of a
\grammarterm{template-id}, the concept argument list is the sequence of
\grammarterm{template-argument}{s} of the \grammarterm{template-id}.
\end{itemize}


\pnum
The selection of a concept from the set referred to by the 
\grammarterm{concept-name} \tcode{C} is done by matching the
concept argument list against the template parameter lists of each
concept in that set.
% 
For a concept \tcode{CC} in that set to be a viable selection, each 
argument in the concept argument list is matched against the corresponding 
template parameters of \tcode{CC}.
% 
Default template arguments (if present) are instantiated for each template 
parameter that does not correspond to a concept argument. Instantiated
default arguments are appended to the concept argument list.
% 
If the last declared template parameter of \tcode{CC} is not a parameter pack
and the number of template parameters of \tcode{CC} is greater than the
number of concept arguments, \tcode{CC} is not a viable selection.
% 
Otherwise, concept arguments are matched to template parameters using the 
following rules:
% 
\begin{itemize}
\item a template argument matches a template parameter if and only if
it matches in kind (type, non-type, template) and type according to the
rules in \ref{temp.arg};

\item a wildcard matches a template parameter of any kind;

\item a template parameter pack (\ref{temp.variadic}), matches zero or more 
concept arguments, provided that each of those arguments matches the pattern 
of the template parameter pack using the rules above for matching matching 
concept arguments and template parameters.
\end{itemize}
% 
If any concept arguments do not match a corresponding template parameter,
the concept \tcode{CC} is not a viable selection.
% 
The concept selected by concept resolution shall be the single viable selection
in the set of concepts referred by \tcode{C}.
% 
\enterexample
\begin{codeblock}
template<typename T> concept bool C1() { return true; }             // \#1
template<typename T, typename U> concept bool C1() { return true; } // \#2
template<typename T> concept bool C2() { return true; }
template<int T> concept bool C2() { return true; }
template<typename... Ts> concept bool C3 = true;

void f1(const C1*); // OK: \tcode{C1} selects \#1
void f2(C1<char>);  // OK: \tcode{C1<char>} selects \#2

template<C2<0> T> struct S1; // error: no matching concept for \tcode{C2<0>},
                             // mismatched template arguments
template<C2 T> struct S2;    // error: resolution of \tcode{C2} is ambiguous,
                             // both concepts are viable

C3{...Ts} void q1(); // OK: selects \tcode{C3}
C3{T} void q2();     // OK: selects \tcode{C3}
\end{codeblock}
\exitexample

\end{quote}
