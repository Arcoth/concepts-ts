
\setcounter{chapter}{8}
\rSec0[class]{Classes}

\setcounter{section}{1}
\rSec1[class.mem]{Class members}

Modify the grammar in 9.2 to allow a 
\grammarterm{requires-clause} for member declarations.

\begin{quote}
\pnum
\begin{bnf}
\nontermdef{member-declarator}
  declarator virt-specifier-seq\opt \added{requires-clause\opt} pure-specifier-seq\opt
\end{bnf}
\end{quote}

Insert the following after paragraph 8, explaining where a
\grammarterm{requires-clause} can appear
and that its \grammarterm{constraint-expression}
can refer to parameters.

\begin{quote}
\setcounter{Paras}{8}
\pnum
A \grammarterm{requires-clause} shall only appear
in a \grammarterm{member-declarator} if its
\grammarterm{declarator} is a function declarator.

The \grammarterm{requires-clause} associates its 
\grammarterm{constraint-expression} with the 
member function.

\enterexample
\begin{codeblock}
  struct A {
    A(int*) requires true;  // <i>OK: constrained constructor</i>
    ~A() requires true;     // <i>OK: constrained destructor</i>
    void f() requires true; // <i>OK: constrained member function</i>
    int x requires true;    // <i>error: constrained member variable</i>
  };
\end{codeblock}
\exitexample
\end{quote}

\begin{quote}
\pnum
The names of parameters in a function declarator are visible in the
\grammarterm{constraint-expression} of the
\grammarterm{requires-clause}. 
\end{quote}
