
\rSec0[intro]{General}

\rSec1[intro.scope]{Scope}

\pnum
This Technical Specification describes extensions to the \Cpp
Programming Language (\ref{intro.refs}) that
enable the specification and checking of constraints on template 
arguments, and the ability to overload functions and specialize
class templates based on those constraints. These extensions include 
new syntactic forms and modifications to existing language semantics.

\pnum
The International Standard, ISO/IEC 14882, provides important context
and specification for this Technical Specification. This document is 
written as a set of changes against that specification. Instructions
to modify or add paragraphs are written as explicit instructions. 
Modifications made directly to existing text from the International
Standard use \added{underlining} to represent added text and
\removed{strikethrough} to represent deleted text.

\pnum
WG21 paper N4191 defines ``fold expressions'', which are used to define 
constraint expressions resulting from the use of 
\grammarterm{constrained-parameter}{s} that declare template parameter
packs. This feature is not present in ISO/IEC 14882:2014, but it is planned 
to be included in the next revision of that International Standard. The
specification of that feature is included in this document.


\rSec1[intro.refs]{Normative references}

\pnum
The following referenced document is indispensable for the
application of this document. For dated references, only the
edition cited applies. For undated references, the latest edition
of the referenced document (including any amendments) applies.

\begin{itemize}
\item ISO/IEC 14882:2014, \doccite{Programming Languages -- \Cpp}
\end{itemize}

ISO/IEC 14882:2014 is hereafter called the \defn{\Cpp Standard}.
%
The numbering of Clauses, sections, and paragraphs in this document
reflects the numbering in the \Cpp Standard. References to Clauses
and sections not appearing in this Technical Specification refer to
the original, unmodified text in the \Cpp Standard.

\rSec1[intro.defs]{Terms and definitions}

Modify the definitions of ``signature'' to include associated
constraints (\ref{temp.constr.decl}). This allows different translation units
to contain definitions of functions with the same signature, excluding 
associated constraints, without violating the one definition rule 
(\cxxref{basic.def.odr}). That is, without incorporating the constraints
in the signature, such functions would have the same mangled name, thus
appearing as multiple definitions of the same function.

\begin{quote}
\indexdefn{signature}%
\definition{signature}{defns.signature}
<function> name, parameter type list~(\ref{dcl.fct}), \removed{and} enclosing 
namespace (if any)\added{, and any associated constraints~(\ref{temp.constr.decl})}\\
\enternote Signatures are used as a basis for
name mangling and linking.\exitnote

\indexdefn{signature}%
\definition{signature}{defns.signature.templ}
<function template> name, parameter type list~(\ref{dcl.fct}), enclosing namespace (if any),
return type, \removed{and} template parameter list\added{, and any associated
constraints~(\ref{temp.constr.decl})}

\indexdefn{signature}%
\definition{signature}{defns.signature.member}
<class member function> name, parameter type list~(\ref{dcl.fct}), class of which the
function is a member, \cv-qualifiers (if any),
\removed{and} \grammarterm{ref-qualifier} (if any)\added{, and any associated
constraints~(\ref{temp.constr.decl})}

\indexdefn{signature}%
\definition{signature}{defns.signature.member.templ}
<class member function template> name, parameter type list~(\ref{dcl.fct}), class of which the
function is a member, \cv-qualifiers (if any),
\grammarterm{ref-qualifier} (if any), return type, \removed{and} template 
parameter list\added{, and any associated constraints~(\ref{temp.constr.decl})}
\end{quote}

%%
%% Implementation compliance
%%
\rSec1[intro.compliance]{Implementation compliance}

\pnum
Conformance requirements for this specification are the same as those 
defined in \ref{intro.compliance} in the \Cpp Standard.
\enternote 
Conformance is defined in terms of the behavior of programs.
\exitnote

%%
%% Feature-testing recommendations
%%
\rSec1[intro.features]{Feature-testing recommendations (Informative)}

\pnum
For the sake of improved portability between partial implementations of various
C++ standards, WG21 (the ISO Technical Committee for the \Cpp Programming
Language) recommends that implementers and programmers follow the guidelines in
this section concerning feature-test macros. 
\enternote 
WG21's SD-6 makes similar recommendations for the \Cpp Standard.
\exitnote

\pnum
Implementers who provide a new language feature should define a macro with the
recommended name, in the same circumstances under which the feature is available
(for example, taking into account relevant command-line options), to indicate
the presence of support for that feature. Implementers should define that macro
with the value specified in the most recent version of this Technical
Specification that they have implemented. The macro name for this Technical 
Specification is \tcode{__cpp_experimental_concepts}, and its value is 
\tcode{201501}.

\pnum
No header files should be required to test macros describing the presence
of support for language features.

\rSec1[intro.ack]{Acknowledgments}

\pnum
The design of this specification is based, in part, on a concept 
specification of the algorithms part of the C++ standard library, known 
as ``The Palo Alto'' report (WG21 N3351), which was developed by a large 
group of experts as a test of the expressive power of the idea of 
concepts. Despite syntactic differences between the notation of the 
Palo Alto report and this Technical Specification, the report can be seen as a 
large-scale test of the expressiveness of this Technical Specification.

\pnum
This work was funded by NSF grant ACI-1148461.
