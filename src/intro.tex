
\rSec0[intro]{General}

\rSec1[intro.scope]{Scope}

\pnum
This technical specification describes extensions to the C++ 
Programming Language (\ref{intro.refs}) that
enable the specification and checking of constraints on template 
arguments, and the ability to overload functions and specialize
class templates based on those constraints. These extensions include 
new syntactic forms and modifications to existing language semantics.

\pnum
International Standard, ISO/IEC 14882, provides important context
and specification for this Technical Specification. This document as 
written as a set of changes against that specification. Instructions
to modify or add paragraphs are written as explicit instructions. 
Modifications made directly to existing text from the International
Standard use \added{underlining} to represent added text and
\removed{strikethrough} to represent deleted text.


\rSec1[intro.refs]{Normative references}

\pnum
The following referenced document is indispensable for the
application of this document. For dated references, only the
edition cited applies. For undated references, the latest edition
of the referenced document (including any amendments) applies.
\begin{itemize}
\item ISO/IEC 14882:2014, \doccite{Programming Languages -- \Cpp}
\end{itemize}
% 
ISO/IEC 14882:2014 is herein after called the \defn{C++ Standard}.
References to clauses within the C++ Standard are written as ``\Cpp \S 3.2''.

\rSec1[intro.compliance]{Implementation compliance}

\pnum
Conformance requirements for this specification are the same as those 
defined in \ref{intro.compliance} in the \Cpp Standard.
\enternote 
Conformance is defined in terms of the behavior of programs.
\exitnote


\rSec1[intro.ack]{Acknowledgments}

\pnum
The design of this specification is based, in part, on a concept 
specification of the algorithms part of the \Cpp standard library, known 
as ``The Palo Alto'' report (WG21 N3351), which was developed by a large 
group of experts as a test of the expressive power of the idea of 
concepts. Despite syntactic differences between the notation of the 
Palo Alto report and this Technical Specification, the report can be seen as a 
large-scale test of the expressiveness of this Technical Specification.

\pnum
This work was funded by NSF grant ACI-1148461.
